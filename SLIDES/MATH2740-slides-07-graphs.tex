%\PassOptionsToPackage{landscape}{geometry}
\documentclass[aspectratio=43]{beamer}

\input{slides_setup_nonLightBoard_whiteBG.tex}

\title{Graphs -- Introduction (theory)}
\date{}

\begin{document}
\DeclareFontShape{OT1}{cmss}{b}{n}{<->ssub * cmss/bx/n}{} 
\begin{frame}
	\titlepage
\end{frame}

%\stretchon
%%%%%%%%%%%%%%%%%%%%%
%%%%%%%%%%%%%%%%%%%%%
%%%%%%%%%%%%%%%%%%%%%
%%%%%%%%%%%%%%%%%%%%%



\begin{frame}{Graphs versus networks}
	Mostly a terminology difference:
	\begin{itemize}
	\item graphs in the mathematical world
	\item networks elsewhere
	\end{itemize}
	\vfill
	I will mostly say \emph{graphs} (this is a math course) but might oscillate
	\vfill
	Beware: language is not consistent, so make sure you read the definitions at the start of whatever source you are using
\end{frame}

\begin{frame}{Graphs vs digraphs vs multigraphs vs multidigraphs vs ...}
	Name-wise and notation-wise, this domain is a bit of a mess. We see definitions later, but as far as possible, in these notes:
	\begin{itemize}
		\item The vertex set $V$ is essentially the only constant in what follows
		\item An undirected graph is denoted $G=(V,E)$, where $E$ are the \emph{edges}
		\item An undirected multigraph is denoted $G_M=(V,E)$. We will not be using these much
		\item A \emph{directed graph} (or \emph{digraph}) is denoted $G=(V,A)$, where $A$ are the \emph{arcs}
		\item A \emph{directed multigraph} (or \emph{multidigraph}) is denoted $G_M=(V,A)$
		\item Any of the above is called a \emph{graph} and is denoted $G=(V,X)$, when we seek generality
	\end{itemize}
And just to confuse the whole thing more: we often say \emph{graph} for \emph{unoriented graph}.
\end{frame}

\begin{frame}{The bridges of K\"onigsberg}
	A ``real life'' problem from Euler: is it possible to cross the seven bridges in a continuous walk without recrossing any of them?
	\begin{center}
	\includegraphics[angle=90,width=.49\textwidth]{FIGS_slides/bridge_color}
	\includegraphics[angle=90,width=.49\textwidth]{FIGS_slides/graph_bridge}
	\end{center}
	\begin{block}{Mathematical problem}
	Is it possible to find a trail containing all edges of the graph?
	\end{block}
\end{frame}


\begin{frame}{Finding a cycle with all vertices}
	A salesperson must visit a couple of cities for their job. Is it possible for them to plan a round trip using highways enabling tehm to visit each specified city exactly once?
	\begin{center}
	\includegraphics[angle=90,width=.35\textwidth]{FIGS_slides/florida}
	\end{center}
	\begin{itemize}
	\item vertices correspond to cities
	\item two vertices are connected iff a highway connects the corresponding cities and does not pass through any other city.
	\end{itemize}
	\begin{block}{Mathematical problem}
	Is it possible to find a cycle containing all graph vertices?
	\end{block}
\end{frame}

\begin{frame}{How far is it to drive through $n$ cities?}
	What is the minimal length of driving needed to drive through $n$ cities?
	\begin{center}
	\includegraphics[angle=90,width=.4\textwidth]{FIGS_slides/florida}
	\end{center}
	\begin{itemize}
	\item vertices correspond to the cities
	\item all cities are connected; each edge has a value assigned to it
	\end{itemize}
	\begin{block}{Mathematical problem}
	What is the minimal spanning tree associated to the graph?
	\end{block}
\end{frame}


%%%%%%%%%%%%%%%%%%%%
%%%%%%%%%%%%%%%%%%%%
%%%%%%%%%%%%%%%%%%%%
%%%%%%%%%%%%%%%%%%%%
\section{Binary relations}

\begin{frame}\frametitle{Binary relation}
	\begin{definition}[Binary relation]
	\begin{itemize}
	\item A \defword{binary relation} is an arbitrary association of elements of one set with elements of another (maybe the same) set
	\item  A binary relation over the sets $X$ and $Y$ is defined as a subset of the Cartesian product $X\times Y =\{(x,y)| x\in X , y\in Y\}$
	\item $(x,y)\in R$ is read ``$x$ is $R$-related to $y$'' and is denoted $xRy$
	\item If $(x,y)\not\in R$, we write not $x R y$
	\end{itemize}
	\end{definition}
\end{frame}
	
\begin{frame}
	\begin{definition}[Properties of binary relations]
		A binary relation $R$ over a set $X$ is
	\begin{itemize}
	\item \defword{Reflexive} if $\forall x\in X$, $xRx$
	\item \defword{Irreflexive} if there does not exist $x\in X$ such that $xRx$
	\item \defword{Symmetric} if $xRy \Rightarrow yRx$
	\item \defword{Asymmetric} if $xRy \Rightarrow $ not $y R x$
	\item \defword{Antisymmetric} if $xRy$ and $yRx$ $\Rightarrow$ $x=y$
	\item \defword{Transitive} if $xRy$ and $yRz$ $\Rightarrow$  $xRz$
	\item \defword{Total} (or \defword{complete}) if $\forall x, y\in X$, $x R y$ or $y R x$
	\end{itemize}
	\end{definition}
\end{frame}
	
	
\begin{frame} 
	\begin{definition}[Equivalence relation]
	 A relation which is reflexive, symmetric and transitive is called an \defword{equivalence relation}
	\end{definition}
	\vfill
	\begin{definition}[Partial order]
	 A relation which is reflexive, antisymmetric and transitive is called a \defword{partial order}
	\end{definition}
	\vfill
	\begin{definition}[Total order]
	A partial order which is total is called a \defword{total order}
	\end{definition}
\end{frame}
	



%%%%%%%%%%%%%%%%%%%%
%%%%%%%%%%%%%%%%%%%%
%%%%%%%%%%%%%%%%%%%%
%%%%%%%%%%%%%%%%%%%%
\section{Undirected graphs}


\subsection{Undirected graph}


\begin{frame}{Graph}
	Intuitively: a graph is a set of points, and a set of relations between the points
	\vfill
	The points are called the \emph{vertices} of the graph and the relations are the \emph{edges} of the graph
	\vfill
	We can also think of the relations as being one directional, in which case the relations are the \emph{arcs} of the digraph (a contraction of ``directed graph'')
\end{frame}


\begin{frame}\frametitle{Graph, vertex and edge} 
	\begin{definition}[Graph]
	An \defword{undirected graph} is a pair $G=(V,E)$ of sets such that
	\begin{itemize}
	\item $V$ is a set of points:  $V=\{v_1,v_2,v_3,..,v_p\}$
	\item $E$ is a set of 2-element subsets of $V$: $E=\{\{v_i,v_j\},\{v_i,v_k\},\dots,\{v_n,v_p\}\}$, also noted $E=\{v_iv_j,v_iv_k,\dots,v_nv_p\}$
	\end{itemize}
	\end{definition}
	\begin{definition}[Vertex]
	The elements of $V$ are the \defword{vertices} (or nodes, or points) of the graph $G$.
	$V$ (or $V(G)$) is the vertex set of the graph $G$
	\end{definition}
	\begin{definition}[Edge]
	The elements of $E$ are the \defword{edges} (or lines) of the graph $G$.
	$E$ (or $E(G)$) is the edge set of the graph $G$
	\end{definition}
\end{frame}

\begin{frame}\frametitle{Order and Size}
	\begin{definition}[Order of a graph]
	The number of vertices in $G$ is the \defword{order} of $G$. Using the notation $|V(G)|$ for the \emph{cardinality} of $V(G)$,
	$$|V(G)|=\textrm{order of G}$$
	\end{definition}
	\vfill
	\begin{definition}[Size of a graph]
	The number of edges in $G$ is the \defword{size} of $G$,
	$$|E(G)|=\textrm{size of G}$$
	\end{definition}
	\vfill
	\begin{itemize}
	\item A graph having order $p$ and size $q$ is called a $(p,q)-$graph
	\item A graph is finite if $|V(G)|<\infty$
	\end{itemize}
\end{frame}

\begin{frame}\frametitle{Adjacent - Incident}
	\begin{definition}[Incident]
	\begin{itemize}
	\item A vertex $v$ is \defword{incident} with an edge $e$ if $v\in e$; then $e$ is an edge at $v$
	\item If $e=uv\in E(G)$, then $u$ and $v$ are each incident with $e$
	\item The two vertices incident with an edge are its ends
	\item An edge $e=uv$ is incident with both vertices $u$ and $v$
	\end{itemize}
	\end{definition}
	\vfill
	\begin{definition}[Adjacent]
	\begin{itemize}
	\item Two vertices $u$ and $v$ are \defword{adjacent} in a graph $G$ if $uv\in E(G)$
	\item If $uv$ and $uw$ are distinct edges (i.e. $v\not=w$) of a graph $G$, then $uv$ and $uw$ are adjacent edges
	\end{itemize}
	\end{definition}
\end{frame}


\begin{frame}
	\begin{definition}[Multiple edge]
	\defword{Multiple edges} are two or more edges connecting the same two vertices within a multigraph
	\end{definition}
	\vfill
	\begin{definition}[Loop]
	A \defword{loop} is an edge with both the same ends; \emph{e.g.} $\{u,u\}$ is a loop
	\end{definition}
	\vfill
	\begin{definition}[Simple graph]
		A \defword{simple graph} is a graph which contains no loops or multiple edges
	\end{definition}
	\vfill
	\begin{definition}[Multigraph]
		A \defword{multigraph} is a graph which can contain multiple edges or loops
	\end{definition}
\end{frame}


\begin{frame}\frametitle{Graph and binary relations}
	A (simple) graph $G$ can be defined in term of a vertex set $V$ and an irreflexive and symmetric binary relation over $V$
	\vfill
	$R$ is symmetric if $(u,v)\in R$ $\Rightarrow$ $(v,u)\in R$  (or, in other words, $uRv\implies vRu$)
	\vfill
	Hence, $\{(u,v),(v,u)\}\in E(G)$ ($\{(u,v),(v,u)\}$ is an edge)
	\vfill
	The set of edges $E(G)$ is the set of symmetric pairs in $R$
\end{frame}


\subsection{Degree of a vertex}

\begin{frame}
	\begin{definition}[Degree of a vertex]
	Let $v$ be a vertex of $G=(V,E)$.
	\begin{itemize}
	\item The number of edges of $G$ incident with $v$ is the \defword{degree} of $v$ in $G$
	\item The number of edges of $G$ at $v$ is the \defword{degree} of $v$ in $G$
	\item The degree of $v$ in $G$ is noted $d_G(v)$ or $deg_G(v)$
	\end{itemize}
	\end{definition}
	\vfill
	\begin{theorem}
	Let $G$ be a $(p,q)-$graph with vertices $v_1$, $\dots$, $v_p$, then
	\[
		\sum_{i=1}^{p}d_G(v_i)=2q
	\]
	\end{theorem}
\end{frame}

\begin{frame}
\begin{definition}[{Odd vertex}]
A vertex is an \defword{odd vertex} is its degree is odd.
\end{definition}
\vfill
\begin{definition}[{Even vertex}]
A vertex is called \defword{even vertex} is its degree is even.
\end{definition}
\vfill
\begin{theorem}
Every graph contains an even number of odd vertices.
\end{theorem}
\end{frame}

\begin{frame}{Regular graph}
\begin{definition}[{Regular graph}]
	If all the vertices of $G$ have the same degree $k$, then the graph $G$ is $k$-regular
\end{definition}
\end{frame}



%%%%%%%%%%%%%%%%%%%%%%%%%%%%%%
%%%%%%%%%%%%%%%%%%%%%%%%%%%%%%
\subsection{Complete, bipartite and other notable graphs}
\begin{frame}
\begin{definition}[{Complete graph}]
A graph is complete if every two of its vertices are adjacent
\end{definition}
\vfill
\begin{definition}[$n$-clique]
	A simple, complete graph on $n$ vertices is called an $n$-\defword{clique} and is often denoted $K_n$
\end{definition}
\vfill
Note that a complete graph of order $p$ is $(p-1)-$regular
\end{frame}

\begin{frame}\frametitle{Bipartite graph}
	\begin{definition}[Bipartite graph]
		A graph $G=(V,E)$ is called \defword{bipartite} if it is possible to partition the vertex set of $G$ into two subsets, say $V_1$ and $V_2$ so that every edge of $G$ joins a vertex of $V_1$ with a vertex of $V_2$ and no vertex joins another vertex of its own set
	\end{definition}
	\vfill
	\begin{definition}[Bipartite graph]
		A graph is \defword{bipartite} if its vertices can be partitioned into two sets $X_1$ and $X_2$, s.t. no two vertices in the same set are adjacent.
		This graph may be written $G=(X_1,X_2,U)$
	\end{definition}
\end{frame}


\begin{frame}\frametitle{Complete Bipartite graph}
	\begin{definition}[Complete bipartite graph]
		A bipartite graph in which every two vertices from the 2 different partitions are adjacent is called a complete bipartite graph
	\end{definition}
	\vfill
	\begin{definition}[Complete bipartite graph]
		If $\forall x_1\in X_1, \forall x_2\in X$, we have $m_G(x_1,x_2)\geq 1$, then $G=(X_1,X_2,U)$ is a \defword{complete bipartite} graph
		\vskip0.25cm
		A simple, complete bipartite graph with $|X_1|=p$ and $|X_2|=q$ is often denoted $K_{p,q}$
	\end{definition}
\end{frame}

\begin{frame}\frametitle{Some specific graphs}
	\begin{definition}[{Tree}]
		Any connected graph that has no cycles is a \defword{tree}
	\end{definition}
	\begin{definition}[{Cycle $C_n$}]
		For $n\geq 3$, the \defword{cycle} $C_n$ is a connected graph of order $n$ that is a cycle on $n$ vertices
	\end{definition}
	\begin{definition}[{Path $P_n$}]
		The \defword{path} $P_n$ is a connected graph that consists of $n\geq 2$ vertices and $n-1$ edges. Two vertices of $P_n$ have degree 1 and the rest are of degree 2
	\end{definition}
	\begin{definition}[{Star $S_n$}]
		The \defword{star} of order $n$ is the complete bipartite graph $K_{1,n-1}$ (1 vertex of degree $n-1$ and $n-1$ vertices of degree 1)
	\end{definition}
\end{frame}





\subsection{Isomorphic graphs}

\begin{frame} \frametitle{Isomorphic graphs} 
\begin{definition}[Isomorphic graphs]
Let $G_1=(V(G_1),E(G_1))$ and $G_2=(V(G_2),E(G_2))$ be two graphs.
$G_1$ and $G_2$ are \defword{isomorphic} if there exists an isomorphism $\phi$ from $G_1$ to $G_2$, that is defined as an injective mapping $\phi:\; V(G_1) \rightarrow V(G_2)$ such that two vertices $u_1$ and $v_1$ are adjacent in $G_1$ if and only if the vertices $\phi(u_1)$ and $\phi(v_1)$ are adjacent in $G_2$
\end{definition}
\vfill
\begin{definition}[Another formulation of isomorphic graphs]
Let $G_1=(V(G_1),E(G_1))$ and $G_2=(V(G_2),E(G_2))$ be two graphs;
$G_1$ and $G_2$ are isomorphic if there exists an isomorphism $\phi$ from $G_1$ to $G_2$, defined as 
an injective mapping $\phi : \; V(G_1) \rightarrow V(G_2)$ such that $\{u_1,v_1\} \in E(G_1)$ if and only if $\{\phi(u_1),\phi(v_1)\}\in E(G_2)$ for all $u_1, v_1 \in V(G_1)$
\end{definition}
\end{frame}
 
 
 
\begin{frame}
If $\phi$ is an isomorphism from $G_1$ to $G_2$, then the inverse mapping $\phi ^{-1}$ from $V(G_2)$ to $V(G_1)$ also satisfies the definition of an isomorphism.
As a consequence, if $G_1$ and $G_2$ are isomorphic graphs, then
\begin{itemize}
\item $G_1$ is isomorphic to $G_2$
\item $G_2$ is isomorphic to $G_1$
\end{itemize}
\vfill
\begin{theorem}
The relation ``is isomorphic to'' is an equivalence relation on the set of all graphs
\end{theorem}
\vfill
\begin{theorem}
If $G_1$ and $G_2$ are isomorphic graphs, then the degrees of vertices of $G_1$ are exactly the degrees of vertices of $G_2$
\end{theorem}
\end{frame}


\subsection{Subgraphs, unions of graphs}

\begin{frame}\frametitle{Subgraph}
\begin{definition}[Subgraph]
Let $G=(V(G),E(G))$ be a graph.
A graph $H=(V(H),E(H))$ is a \defword{subgraph} of $G$ if $V(H)\subseteq V(G)$ and $E(H)\subseteq E(G)$
\end{definition}
\vfill
\begin{definition}[Another definition of subgraph]
If a graph $F$ is isomorphic to a subgraph $H$ of $G$, then $F$ is also called a subgraph of $G$
\end{definition}
\end{frame}


\begin{frame}
Let $G_1=(V_1,E_1)$ and $G_2=(V_2,E_2)$ be two graphs.
\begin{definition}[{Union of $G_1$ and $G_2$}]
$G_1\cup G_2=(V_1\cup V_2,E_1\cup E_2)$
\end{definition}
\begin{definition}[{Intersection of $G_1$ and $G_2$}]
$G_1\cap G_2=(V_1\cap V_2,E_1\cap E_2)$
\end{definition}
\begin{definition}[{Disjoint graphs}]
If $G_1\cap G_2=(\emptyset,\emptyset)= \emptyset$ (empty graph) then $G_1$ and $G_2$ are disjoint.
\end{definition}
\begin{definition}[{Complement of $G_1$}]
The complement $\bar G_1$ of $G_1$ is the graph on $V_1$, with the edge set $E(\bar G_1)=[V_1]^2\backslash E_1$ ($e\in E(\bar G_1)$ iff $e\not \in E_1$)
\end{definition}
\end{frame}


\subsection{Walks, trails, paths}


\begin{frame}\frametitle{Walk}
\begin{definition}[{Walk}]
A \defword{walk}\index{walk, graph} in a graph $G=(V,E)$ is a non-empty alternating sequence $v_0 e_0 v_1 e_1 v_2 \dots e_{k-1} v_k$ of vertices and edges in $G$ such that $e_i=\{v_i, v_{i+1}\}$ for all $i<k$. 
This walk begins with $v_0$ and ends with $v_k$
\end{definition}
\vfill
\begin{definition}[Length of a walk]
The \defword{length} of a walk is equal to the number of edges in the walk
\end{definition}
\vfill
\begin{definition}[{Closed walk}]
If $v_0=v_k$, the walk is \defword{closed}
\end{definition}
\end{frame}


\begin{frame}\frametitle{Trail and path}
\begin{definition}[{Trail}]
If the edges in the walk are all distinct, it defines a \defword{trail} in $G=(V,E)$
\end{definition}
\vfill
\begin{definition}[{Path}]
If the vertices in the walk are all distinct, it defines a \defword{path} in $G$
\end{definition}
\vfill
The sets of vertices and edges determined by a trail is a subgraph
\end{frame}


\begin{frame}\frametitle{Distance between two vertices}
\begin{definition}[{Distance between two vertices}]
The distance $d(u,v)$ in $G=(V,E)$ between two vertices $u$ and $v$ is the length of the shortest path linking $u$ and $v$ in $G$. If no such path exists, we assume $d(u,v)=\infty$
\end{definition}
\end{frame}


\begin{frame}\frametitle{Circuit and cycle}
\begin{definition}[Circuit]
A trail linking $u$ to $v$, containing at least 3 edges and in which $u=v$, is a \defword{circuit}
\end{definition}
\vfill
\begin{definition}[Cycle]
A circuit which does not repeat any vertices (except the first and the last) is a \defword{cycle} (or \defword{simple circuit})
\end{definition}
\vfill
\begin{definition}[Length of a cycle]
The \defword{length of a cycle} is its number of edges
\end{definition}
\end{frame}


\begin{frame}
\begin{definition}[{Eulerian trail}]
	A trail containing all the vertices and edges of a multigraph $M$ is a \defword{Eulerian trail} of $M$
\end{definition}
\vfill
\begin{definition}[{Traversable graph}]
	If a graph $G$ has a Eulerian trail, then $G$ is a \defword{traversable graph}
\end{definition}
\vfill
\begin{definition}[{Eulerian circuit}]
	A circuit containing all the vertices and edges of a multigraph $M$ is a \defword{Eulerian circuit} of $M$
\end{definition}
\vfill
\begin{definition}[{Eulerian graph}]
	A graph (resp. multigraph) containing an Eulerian circuit is a \defword{Eulerian graph} (resp. \defword{Eulerian multigraph})
\end{definition}
\end{frame}

\begin{frame}
\begin{theorem}
	A multigraph $M$ is traversable if and only if $M$ is connected and has exactly two odd vertices. Furthermore, any Eulerian trail of $M$ begins at one of the odd vertices and ends at the other odd vertex
\end{theorem}
\vfill
\begin{theorem}
	A multigraph $M$ is Eulerian $\iff$ $M$ is connected and every vertex of $M$ is even
\end{theorem}
\end{frame}




\begin{frame}\frametitle{Fleury's algorithm to find a Eulerian circuit}
(for a connected graph with no odd vertices)

\begin{itemize}
	\item Pick any vertex as a starting point.
	\item Marking your path as you move from vertex to vertex, travel along any edges you wish, but DO NOT travel along an edge that is a bridge for the graph formed by the EDGES THAT HAVE YET TO BE TRAVELED -- unless you have to.
	\item Continue until you return to your starting point.
\end{itemize}
RESULT: a Eulerian circuit
\end{frame}


\begin{frame}\frametitle{Fleury's algorithm to find a Eulerian trail}
(for a connected graph with exactly 2 odd vertices)

\begin{itemize}
	\item Start at one of the odd vertices.
	\item Marking your path as you move from vertex to vertex, travel along any edges you wish, but DO NOT travel along an edge that is a bridge for the graph formed by the EDGES THAT HAVE YET TO BE TRAVELED -- unless you have to.
	\item Continue until every edge has been traveled.
\end{itemize}
RESULT: a Eulerian trail
\end{frame}


\begin{frame}\frametitle{}
\begin{definition}[{Hamiltonian path}]
	A path containing all vertices of a graph $G$ is a \defword{Hamiltonian path} of $G$
\end{definition}
\vfill
\begin{definition}[{Traceable graph}]
	If a graph $G$ has an Hamiltonian path, then $G$ is a \defword{traceable graph}
\end{definition}
\vfill
\begin{definition}[{Hamiltonian cycle}]
	A cycle containing all vertices of a graph $G$ is a \defword{Hamiltonian cycle} of $G$
\end{definition}
\vfill
\begin{definition}[{Hamiltonian graph}]
	A graph containing a Hamiltonian cycle is a \defword{Hamiltonian graph}
\end{definition}
\end{frame}

\begin{frame}
\begin{theorem}[Dirac's theorem]
	If $G$ is a graph of order $p\geq 3$ such that $deg(v)\geq p/2$ for every vertex $v$ of $G$, then $G$ is Hamiltonian
\end{theorem}
\vfill
\begin{theorem}[Ore's theorem]
	If $G$ is a graph of order $p\geq 3$ such that for all distinct nonadjacent vertices $u$ and $v$ of $G$, $$deg(u)+deg(v)\geq p,$$ 
	then $G$ is Hamiltonian
\end{theorem}
\end{frame}


%%%%%%%%%%%%%%%%%%%%%%
%%%%%%%%%%%%%%%%%%%%%%
\subsection{Connectedness}

\begin{frame}\frametitle{Connected vertices and graph, components}
\begin{definition}[{Connected vertices}]
Two vertices $u$ and $v$ in a graph $G$ are \defword{connected}\index{connected vertices} if $u=v$, or if $u\not =v$ and there exists a path in $G$ that links $u$ and $v$
\end{definition}
\vfill
\begin{definition}[{Connected graph}]
A graph is \defword{connected}\index{connected graph} if every two vertices of $G$ are connected; otherwise, $G$ is \defword{disconnected}\index{disconnected graph}
\end{definition}
\end{frame}

\begin{frame}\frametitle{A necessary condition for connectedness}
	\begin{theorem}
		A connected graph on $p$ vertices has at least $p-1$ edges
	\end{theorem}
	\vfill
	In other words, a connected graph $G$ of order $p$ has $\text{size}(G)\geq p-1$
\end{frame}

\begin{frame}{Connectedness is an equivalence relation}
	Denote $x\equiv y$ the relation  ``$x=y$, or $x\neq y$ and there exists a path in $G$ connecting $x$ and $y$''. $\equiv$ is an equivalence relation since
	\begin{enumerate}
		\item $x\equiv y$ \hfill[reflexivity]
		\item $x\equiv y\implies y\equiv x$ \hfill[symmetry]
		\item $x\equiv y, y\equiv z\implies x\equiv z$ \hfill[transitivity]
	\end{enumerate}
	\vfill
	\begin{definition}[Connected component of a graph]
		The classes of the equivalence relation $\equiv$ partition $X$ into connected sub-graphs of $G$ called \defword{connected components} (or \defword{components} for short) of $G$
	\end{definition}
	\vfill
	A connected subgraph $H$ of a graph $G$ is a component of $G$ if $H$ is not contained in any connected subgraph of $G$ having more vertices or edges than $H$
\end{frame}



\begin{frame}{Vertex deletion \& cut vertices}
\begin{definition}[{Vertex deletion}]
If $v$ is a vertex of $G$, the graph $G-v$ is the graph formed from $G$ by removing $v$ and all edges incident with $v$
\end{definition}
\vfill
\begin{definition}[{Cut-vertices}]
	If $v$ is a cut-vertex of a connected graph $G$, then $G-v$ is disconnected
\end{definition}
\end{frame}


\begin{frame}{Edge deletion \& bridges}
\begin{definition}[{Edge deletion}]
	If $e$ is an edge of $G$, the graph $G-e$ is the graph formed from $G$ by removing $e$ from $G$
\end{definition}
\vfill
\begin{definition}[{Bridge}]
An edge $e$ in a connected graph $G$ is called a bridge if $G-e$ is disconnected
\end{definition}
\vfill
\begin{theorem}
Let $G$ be a connected graph. An edge $e$ of $G$ is a bridge of $G$ if and only if $e$ does not lie on any cycle of $G$
\end{theorem}
\end{frame}


%\begin{frame}\frametitle{Undirected network}
%\begin{definition}{Definition}
%A undirected network is a graph together with a function which maps the edge set into the set of real number.
%\end{definition}\end{frame}


%%%%%%%%%%%%%%%%%%%%%%%
%%%%%%%%%%%%%%%%%%%%%%%
\subsection{Planar graphs}
\begin{frame}\frametitle{Planar graph}
\begin{definition}[Planar graph]
A graph is \defword{planar} if it can be drawn in the plane with no crossing edges. Otherwise it is nonplanar
\end{definition}
\vfill
\begin{definition}[{Plane graph}]
A \defword{plane graph} is a graph that is drawn in the plane with no crossing edges. (This is only possible if the graph is planar)
\end{definition}
\end{frame}



\begin{frame}
Let $G$ be a plane graph
\begin{itemize}
\item the connected parts of the plane are called \defword{regions}
\item vertices and edges that are incident with a region $R$ make up a \defword{boundary} of $R$
\end{itemize}
\vfill
\begin{theorem}[{Euler's formula}]
Let $G$ be a connected plane graph with $p$ vertices, $q$ edges, and $r$ regions, then $$p-q+r=2$$
\end{theorem}
\vfill
\begin{corollary}
Let $G$ be a plane graph with $p$ vertices, $q$ edges, $r$ regions, and $k$ connected components, then $$p-q+r=k+1$$
\end{corollary}
\end{frame}


\begin{frame}
\begin{theorem}
Let $G$ be a connected planar graph with $p$ vertices and $q$ edges, where $p\geq 3$, then $$q\leq 3p-6.$$
(a maximal connected planar graph with $p$ vertices has $q=3p-6$ edges)
\end{theorem}
\vfill
\begin{corollary}
If $G$ is a planar graph, then $\delta(G)\leq 5,$
where $\delta(G)$ is the minimal degree of $G$.
(every planar graph contains a vertex of degree less than 6)
\end{corollary}
\end{frame}


\begin{frame}{Two well-known non-planar graphs}
$K_{3,3}$ is nonplanar
\vfill
$K_5$ is nonplanar
\vfill
As a matter of fact, a graph having a subgraph that is $K_{3,3}$ or $K_5$ is a sufficient condition for non-planarity, as we see later.
\end{frame}


\begin{frame}
\begin{definition}[{Subdivision of $G$}]
Given a graph $G$, a subdivision of $G$ is a graph that can be obtained by inserting any number of vertices of degree 2 along a edges of $G$
\end{definition}
\vfill
\begin{theorem}[{Kuratowski Theorem}]
A graph $G$ i planar if and only if it contains no subgraph isomorphic to $K_5$ or $K_{3,3}$ or any subdivision of $K_5$ or $K_{3,3}$
\end{theorem}
\vfill
\textbf{Note:} If a graph $G$ is nonplanar and $G$ is a subgraph of $G'$, then $G'$ is also nonplanar
\end{frame}


\begin{frame}
\begin{definition}[{Colouring of a graph $G$}]
A \defword{colouring} of a graph $G$ is an assignment of colours to the vertices of $G$ such that adjacent vertices have different colours
\end{definition}
\vfill
\begin{definition}[{$n$-colouring of $G$}]
A \defword{$n$-colouring} is a colouring of $G$ using $n$ colours
\end{definition}
\vfill
\begin{definition}[{$n$-colourable}]
$G$ is \defword{$n$-colourable} if there exists a colouring of $G$ that uses $n$ colours
\end{definition}
\end{frame}


\begin{frame}
\begin{definition}[{Chromatic number}]
The \defword{chromatic number} $\chi (G)=$ of a graph $G$ is the minimal value $n$ for which an $n$-colouring of $G$ exists
\end{definition}
\vfill
\begin{theorem}[{Some properties}]
\begin{itemize}
\item $\chi (G)=1$ if and only if $G$ have no edges
\item If $G=K_{n,m}$, then $\chi(G)=2$
\item If $G=K_n$, then $\chi(G)=n$
\item For any graph $G$, $$\chi (G)\leq 1+ \Delta (G)$$
where $\Delta (G)$ is the maximum degree of $G$
\item If $G$ is a planar graph, then $\chi(G) \leq 4$
\item If $G$ is a planar graph, then $\chi(G) \leq 5$
\end{itemize}
\end{theorem}
\end{frame}


\begin{frame}\frametitle{Welch-Powell algorithm for colouring a graph $G$}
\begin{enumerate}
\item Order the vertices of $G$ in decreasing degree. (Such an ordering may not be unique since some vertices may have the same degree)
\vfill
\item Use one colour to paint the first vertex and to paint, in sequential order, each vertex on the list that is not adjacent to a vertex previously painted with this colour
\vfill
\item Start again at the top of the list and repeat the process painting previously unpainted vertices using a second colour
\vfill
\item Continue repeating with additional colours until all the vertices have been painted
\end{enumerate}\end{frame}

\begin{frame}\frametitle{``Real life'' problem}
What is the minimal number of colors to color all counties in the map?
(two adjacent counties must have different colors)
\begin{center}
\includegraphics[width=.55\textwidth]{FIGS_slides/california}
\end{center}\end{frame}

\begin{frame}\frametitle{``Real life'' problem}
What is the minimal number of colors to color all counties in the map?
(two adjacent counties must have different colors)
\vfill
Mathematical representation:
\begin{itemize}
\item vertices correspond to the counties
\item two vertices are linked iff the two counties are adjacent.
\end{itemize}
\vfill
\begin{block}{Mathematical problem}
What is the chromatic number of the graph associated to the map?
\end{block}
\end{frame}




%%%%%%%%%%%%%%%%%%%
%%%%%%%%%%%%%%%%%%%
%%%%%%%%%%%%%%%%%%%
%%%%%%%%%%%%%%%%%%%
\section{Directed graphs}


\subsection{Directed graph}
\begin{frame} \frametitle{Definitions} 
\begin{definition}[{Digraph}]
A directed graph (or \defword{digraph}) is a pair $G=(V,A)$ of sets such that
\begin{itemize}
\item $V$ is a set of points:  $V=\{v_1,v_2,v_3,..,v_p\}$
\item $A$ is a set of ordered pairs of $V$: $A=\{(v_i,v_j),(v_i,v_k),\dots,(v_n,v_p)\}$, also noted $A=\{v_iv_j,v_iv_k,\dots,v_nv_p\}$.
\end{itemize}
\end{definition}
\begin{definition}[{Vertex}]
The elements of $V$ are the vertices of the digraph $G$.
$V$ is the vertex set of the digraph $G$, also noted $V(G)$.
\end{definition}
\begin{definition}[{Arc}]
The elements of $A$ are the \defword{arcs} (directed edges) of the digraph $G$.
$A$ is the arc set of the digraph $G$, also noted $A(G)$.
\end{definition}
\end{frame}



\begin{frame}\frametitle{Digraph and binary relation}
A digraph $D$ can be defined in term of a vertex set $V$ and an irreflexive relation $R$ over $V$
\vfill
The defining relation $R$ of the digraph $G$ need not be symmetric
\end{frame}


\begin{frame}\frametitle{Directed network}
\begin{definition}[Directed network]
A directed network is a digraph together with a function $f$,
\[
f: A \rightarrow \mathbb{R},
\]
which maps the arc set $A$ into the set of real number. 
The value of the arc $uv \in A$ is $f(uv)$
\end{definition}
\end{frame}



%\begin{frame}\frametitle{Order and Size}
%\begin{definition}{Order of a graph}
%The number of vertices in $G$ is the order of $G$
%$$|V(G)|=\textrm{order of G}$$
%\end{definition}
%\begin{definition}{Size of a graph}
%The number of edges in $G$ is the size of $G$
%$$|E(G)|=\textrm{size of G}$$
%\end{definition}
%\begin{itemize}
%\item A graph having order $p$ and size $q$ is noted a $(p,q)-$graph.
%\item A graph is finite is $|V(G)|<\infty$
%\end{itemize}
%}



\begin{frame}{Loops \& Multiple arcs}
	\begin{definition}[{Loop}]
	A \defword{loop} is an arc with both the same ends; \emph{e.g.} $(u,u)$ is a loop.
\end{definition}
\vfill
	\begin{definition}[{Multiple arcs}]
	\defword{Multiple arcs} (or multi-arcs) are two or more arcs connecting the same two vertices.
	\end{definition}
\end{frame}
\begin{frame}{Multidigraph/Digraph}
	\begin{definition}[{Multidigraph}]
	A \defword{multidigraph} is a digraph which allows repetition of arcs or loops.
	\end{definition}
	\vfill
	\begin{definition}[{Digraph}]
	In a digraph, no more than one arc can join any pair of vertices.
\end{definition}
\end{frame}

\begin{frame}{Examples}
	\def\skip{2.75cm}
	\begin{tikzpicture}[scale=0.7, 
		every node/.style={transform shape},
		auto,
		cloud/.style={minimum width={width("N-1")+2pt},
			draw, ellipse},
		connected/.style={dotted,-}]
		%% Vertices
		\node [cloud] at (0,0) (a) {$a$};
		\node [cloud] at (1*\skip,1*\skip) (b) {$b$};
		\node [cloud] at (1*\skip,-1*\skip) (c) {$c$};
		\node [cloud] at (2*\skip,0) (d) {$d$};
		%% Arcs
		\path [line, thick,bend right] (a) to node [midway, above] (TextNode) {2} (b);
		\path [line, thick] (a) to node [midway, above] (TextNode) {3} (b);
		\path [line, thick,bend left] (a) to node [midway, above] (TextNode) {4} (b);
		\path [line, thick, looseness=5] (b) to node [midway, above] (TextNode) {5} (b);
		\path [line, thick, looseness=15] (b) to node [midway, above] (TextNode) {6} (b);
		\path [line, thick] (c) to node [midway, left] (TextNode) {7} (b);
		\path [line, thick] (c) to node [midway, above] (TextNode) {1} (a);
		\path [line, thick, bend left] (d) to node [midway, above] (TextNode) {8} (c);
		\path [line, thick, bend right] (d) to node [midway, above] (TextNode) {9} (c);
		\path [line, thick, looseness=5] (d) to node [midway, above] (TextNode) {10} (d);
	\end{tikzpicture}
	\begin{tikzpicture}[scale=0.7, 
		every node/.style={transform shape},
		auto,
		cloud/.style={minimum width={width("N-1")+2pt},
			draw, ellipse},
		connected/.style={dotted,-}]
		%% Vertices
		\node [cloud] at (0,0) (x1) {$x_1$};
		\node [cloud] at (1*\skip,1*\skip) (x2) {$x_2$};
		\node [cloud] at (2*\skip,1*\skip) (x3) {$x_3$};
		\node [cloud] at (1.5*\skip,0) (x4) {$x_4$};
		\node [cloud] at (2.5*\skip,0*\skip) (x5) {$x_5$};
		\node [cloud] at (2.5*\skip,1*\skip) (x6) {$x_6$};
		%% Arcs
		\path [line, thick,bend left] (x1) to node [midway, above] (TextNode) {1} (x2);
		\path [line, thick,bend left] (x2) to node [midway, below] (TextNode) {2} (x1);
		\path [line, thick, looseness=5] (x2) to node [midway, above] (TextNode) {3} (x2);
		\path [line, thick] (x2) to node [midway, above] (TextNode) {4} (x3);
		\path [line, thick] (x2) to node [midway, below] (TextNode) {5} (x4);
		\path [line, thick] (x4) to node [midway, below] (TextNode) {6} (x3);
		\path [line, thick, looseness=5, below] (x6) to node [midway, above] (TextNode) {7} (x6);
		\path [line, thick] (x6) to node [midway, right] (TextNode) {8} (x5);
	\end{tikzpicture}
\end{frame}

\begin{frame}
	Let $G=(V,A)$ be a digraph.
	\begin{definition}[Arc endpoints]
	For an arc $u=(x,y)$, vertex $x$ is the \defword{initial endpoint}, and vertex $y$ is the \defword{terminal endpoint}
	\end{definition}
	\begin{definition}[{Predecessor - Successor}]
		If $(u,v)\in A(G)$ is an arc of $G$, then
		\begin{itemize}
			\item  $u$ is a \defword{predecessor} of $v$,
			\item  $v$ is a \defword{successor} of $u$.
		\end{itemize}
	\end{definition}
	\begin{definition}[Neighbours of a vertex]
	Let $x\in V$ be a vertex. The \defword{neighbours} of $x$ is the set $\Gamma(x) = \Gamma^+_G(x)\cup\Gamma^-_G(x)$, where $\Gamma^+_G(x)$ and $\Gamma^-_G(x)$ are, respectively, the set of successors and predecessors of $v$.
	\end{definition}
\end{frame}
	
	
\begin{frame}{Sources and sinks}
	\begin{definition}[{Directed away - Directed towards}]
		If $a=(u,v)\in A(G)$ is an arc of $G$, then
		\begin{itemize}
			\item the arc $a$ is said to be \defword{directed away} from $u$
			\item the arc $a$ is said to be \defword{directed towards} $v$
		\end{itemize}
	\end{definition}
	\vfill
	\begin{definition}[{Source - Sink}]
		\begin{itemize}
			\item Any vertex which has no arcs directed towards it is a \defword{source}
			\item Any vertex which has no arcs directed away from it is a \defword{sink}
		\end{itemize}
	\end{definition}
\end{frame}

\begin{frame}{Adjacent arcs}
	\begin{definition}[Adjacent arcs]
		Two arcs are \defword{adjacent} if they have at least one endpoint in common.
	\end{definition}
\end{frame}

\begin{frame}{Arc incident out of a subset of arcs}
	\begin{definition}[Arc incident out of $A\subset X$]
		If the initial endpoint of an arc $u$ belongs to $A$, and if the terminal endpoint of arc $u$ does not belong to $A$, then $u$ is said to be incident out of $A$, and we write $u\in\omega^+(A)$.
		Similarly, we define an arc incident into $A$, and the set $\omega^-(A)$.
		Finally, the set of arcs incident to $A$ is denoted
		\[
		\omega(A) = \omega^+(A)\cup\omega^-(A).
		\]
	\end{definition}
\end{frame}

\begin{frame}
	\begin{definition}[Symmetric graph]
		If $m_G^+(x,y)=m_G^-(x,y)$ for all $x,y\in X$, the graph $G$ is
		\defword{symmetric}. A 1-graph $G=(X,U)$ is symmetric if, and only if,
		\[
		(x,y)\in U\implies (y,x)\in U
		\]
	\end{definition}
	\vfill
	\begin{definition}[Anti-symmetric graph]
		If for each pair $(x,y)\in X\times X$,
		\[
		m_G^+(x,y)+m_G^-(x,y)\leq 1
		\]
		then the graph $G$ is \defword{anti-symmetric}. A 1-graph $G=(X,U)$ is anti-symmetric if, and only if,
		\[
		(x,y)\in U\implies (y,x)\not\in U
		\]
		An anti-symmetric 1-graph without its direction is a simple graph
	\end{definition}
\end{frame}

\begin{frame}
	\begin{definition}[Subgraph of $G$ generated by $A\subset X$]
		The \defword{subgraph} of $G$ generated by $A$ is the graph with $A$ as its vertex set and with all the arcs in $G$ that have both their endpoints in $A$. If $G=(X,\Gamma)$ is a 1-graph, then the subgraph generated by $A$ is the 1-graph $G_A=(A,\Gamma_A)$ where
		\[
		\Gamma_A(x)=\Gamma(x)\cap A\qquad (x\in A)
		\]
	\end{definition}
	\vfill
	\begin{definition}[Partial graph of $G$ generated by $V\subset U$]
		\label{def:partial_graph}
		The graph $(X,V)$ whose vertex set is $X$ and whose arc set is $V$. 
		In other words, it is graph $G$ without the arcs $U-V$
	\end{definition}
	\vfill
	\begin{definition}[Partial subgraph of $G$]
		A partial subgraph of $G$ is the subgraph of a partial graph of $G$
	\end{definition}
\end{frame}



%%%%%%%%%%%%%%%%%%%%%%%%%%
%%%%%%%%%%%%%%%%%%%%%%%%%%
\subsection{Degrees in digraphs}
\begin{frame}{Degree}
	Let $v$ be a vertex of a digraph $G=(V,A)$.
	\begin{definition}[{Outdegree of a vertex}]
	The number of arcs directed away from a vertex $v$, in a digraph is called the \defword{outdegree} of $v$ and is written $d^+(v)$ or $outdeg(v)$.
	\end{definition}
	\begin{definition}[{Indegree of a vertex}]
	The number of arcs directed towards a vertex $v$, in a digraph is called the \defword{indegree} of $v$ and is written $d^-(v)$ or $indeg(v)$.
	\end{definition}
	\begin{definition}[{Degree}]
	For any vertex $v$ in a digraph, the \defword{degree}\index{degree, digraph} of $v$ is defined as $d(v)=d^+(v)+d^-(v)$.
	\end{definition}
\end{frame}



\begin{frame}
	\begin{theorem}
		For any (di)graph, the sum of the degrees of the vertices equals twice the number of edges (arcs).
	\end{theorem}
	\begin{corollary}
		In any (di)graph, the sum of the degrees of the vertices is a nonnegative even integer.
	\end{corollary}
	\begin{theorem}
		If $G$ is a digraph with a vertex set $V(G)=\{v_1, \dots , v_p\}$ and having $q$ arcs then $$\sum_{i=1}^p d^+(v_i)=\sum_{i=1}^p d^-(v_i)=q.$$
	\end{theorem}
\end{frame}

\begin{frame}
	\begin{definition}[{Regular digraph}]
		A digraph $G$ is $r-$regular if $indeg(v)=outdeg(v)=r$ for each vertex $v$ of $G$.
\end{definition}\end{frame}



%%%%%%%%%%%%%%%%%%%
%%%%%%%%%%%%%%%%%%%
\subsection{Walks, paths, etc.}
\begin{frame}\frametitle{Walks}
	Let $G=(V,A)$ be a digraph.
	\begin{definition}[Directed walk]
	A \defword{directed walk} in a digraph $G$ is a non-empty alternating sequence $v_0 a_0 v_1 a_1 v_2 \dots a_{k-1} v_k$ of vertices and arcs in $G$ such that $a_i=(v_i, v_{i+1})$ for all $i<k$. 
	This walk begins with $v_0$ and ends with $v_k$.
	\end{definition}
	\begin{definition}[{Length of a directed walk}]
	The length of a directed walk is equal to the number of arcs in the directed walk.
	\end{definition}
	\begin{definition}[{Closed walk}]
	If $v_0=v_k$, the walk is closed.
	\end{definition}
\end{frame}


\begin{frame}\frametitle{Trails}
	Let $G=(V,A)$ be a digraph.
	\begin{definition}[{Directed trail}]
	A directed walk in $G$ in which all arcs are distinct is a \defword{directed trail} in $G$.
	\end{definition}
	\begin{definition}[Directed path]
	A directed walk in $G$ in which all vertices are distinct is a \defword{directed path} in $G$.
	\end{definition}
	\begin{definition}[{Directed cycle}]
	A closed walk is a \defword{directed cycle} if it contains at least three vertices and all its vertices are distinct except for $v_0=v_k$.
	\end{definition}
\end{frame}

\begin{frame}{Examples of directed cycles}
	\begin{minipage}{0.5\textwidth}
		\centering
		\def\skip{2.75cm}
		\begin{tikzpicture}[scale=1, 
			every node/.style={transform shape},
			auto,
			cloud/.style={minimum width={width("N-1")+2pt},
				draw, ellipse},
			connected/.style={dotted,-}]
			%% Vertices
			\node [cloud] at (0,0) (a) {$a$};
			\node [cloud] at (1*\skip,0) (b) {$b$};
			\node [cloud] at (0.5*\skip,1*\skip) (c) {$c$};
			\node [cloud] at (1.5*\skip,1*\skip) (d) {$d$};
			%% Arcs
			\path [line, thick] (a) to node [midway, above] (TextNode) {1} (b);
			\path [line, thick, bend left] (a) to node [midway, above] (TextNode) {2} (c);
			\path [line, thick] (b) to node [midway, above] (TextNode) {6} (c);
			\path [line, thick] (b) to node [midway, above] (TextNode) {4} (d);
			\path [line, thick, bend left] (c) to node [midway, above] (TextNode) {3} (a);
			\path [line, thick] (c) to node [midway, above] (TextNode) {5} (d);
		\end{tikzpicture}
	\end{minipage}
	\begin{minipage}{0.48\textwidth}
		Cycles:
		\begin{itemize}
			\item $\bmu^1=(1,6,2)=[abca]$
			\item $\bmu^2=(1,6,3)=[abca]$
			\item $\bmu^3=(2,3)=[aca]$
			\item $\bmu^4=(1,4,5,2)=[abdca]$
			\item $\bmu^5=(6,5,4)=[acdb]$
			\item $\bmu^6=(1,4,5,3)=[abdca]$
		\end{itemize}
	\end{minipage}
\end{frame}

\begin{frame}
	Given a cycle $\mu$, denote $\mu^+$ the set of all arcs in $\mu$ that are in the direction that the cycle is traversed and $\mu^-$ the set of all the other arcs in $\mu$
	\vfill
	Number the arcs in $G$ as $1,2,\ldots,m$, then the cycle $\mu$ is the vector
	\[
	\bm{\mu}=(\mu_1,\ldots,\mu_m)
	\]
	where
	\[
	\mu_i = \begin{cases}
		0 & \textrm{if }i\not\in\mu^+\cup\mu^- \\
		+1 & \textrm{if }i\in\mu^+ \\
		-1 & \textrm{if }i\in\mu^-
	\end{cases}
	\]
\end{frame}


\begin{frame}{Cocycles}
	Let $A\subset X$ be nonempty and denote $\omega^+(A)$ the set of arcs that have only their initial endpoint in $A$ and $\omega^-(A)$ the set of arcs that have only their terminal endpoint in $A$. Let
	\[
	\omega(A)=\omega^+(A)\cup\omega^-(A)
	\]
	\vfill
	A \defword{cocycle} is a nonempty set of arcs of the form $\omega(A)$, partitioned into two sets $\omega^+(A)$ and $\omega^-(A)$
	\vfill
	An \defword{elementary cocycle} is the set of arcs joining two connected subgraphs $A_1$ and $A_2$ s.t. 
	\begin{itemize}
		\item $A_1,A_2\neq\emptyset$
		\item $A_1\cap A_2=\emptyset$
		\item $A_1\cup A_2=C$, with $C$ a connected component of the graph
	\end{itemize}
\end{frame}


\begin{frame}{A colouring lemma}
	\begin{lemma}[Arc colouring Lemma]\label{lemma:arc_colouring_lemma}
		Consider $G$ with arcs $1,\ldots,m$. Colour arc 1 black and arbitrarily colour the remaining arcs red, black or green. Then exactly one of the following holds true:
		\begin{enumerate}
			\item there is an elementary cycle containing arc 1 and only red and black
			arcs with the property that all black arcs in the cycle have the same
			direction
			\item there is an elementary cocycle containing arc 1 and only green and black
			arcs, with the property that all black arcs in the cocycle have the same
			direction
		\end{enumerate}
	\end{lemma}
\end{frame}

\begin{frame}{Independent cycles and cycle bases}
	Consider cycles $\bmu^1,\bmu^2,\ldots,\bmu^k$. The cycles are \defword{independent} if
	\begin{multline*}
		c_1\bmu^1+c_2\bmu^2+\cdots+c_k\bmu^k=\b0 \\
		\iff
		c_1=c_2=\cdots=c_k=0
	\end{multline*}
	A \defword{cycle basis} is an independent set $\{\bmu^1,\bmu^2,\ldots,\bmu^k\}$ of cycles such that any cycle $\bmu$ can be written as
	\[
	\bmu= c_1\bmu^1+c_2\bmu^2+\cdots+c_k\bmu^k
	\]
	for $c_1,\ldots,c_k\in\IR$
	\vfill
	The constant $k$ is the \defword{cyclomatic number} of $G$, denoted $\nu(G)$
\end{frame}


\begin{frame}{Example}
	\begin{minipage}{0.5\textwidth}
		\centering
		\def\skip{2.75cm}
		\begin{tikzpicture}[scale=1, 
			every node/.style={transform shape},
			auto,
			cloud/.style={minimum width={width("N-1")+2pt},
				draw, ellipse},
			connected/.style={dotted,-}]
			%% Vertices
			\node [cloud] at (0,0) (a) {$a$};
			\node [cloud] at (1*\skip,0) (b) {$b$};
			\node [cloud] at (0.5*\skip,1*\skip) (c) {$c$};
			\node [cloud] at (1.5*\skip,1*\skip) (d) {$d$};
			%% Arcs
			\path [line, thick] (a) to node [midway, above] (TextNode) {1} (b);
			\path [line, thick, bend left] (a) to node [midway, above] (TextNode) {2} (c);
			\path [line, thick] (b) to node [midway, above] (TextNode) {6} (c);
			\path [line, thick] (b) to node [midway, above] (TextNode) {4} (d);
			\path [line, thick, bend left] (c) to node [midway, above] (TextNode) {3} (a);
			\path [line, thick] (c) to node [midway, above] (TextNode) {5} (d);
		\end{tikzpicture}
	\end{minipage}
	\begin{minipage}{0.49\textwidth}
	Elementary cycles:
	\begin{itemize}
		\item $\bmu^1=(1,6,2)=[abca]$
		\item $\bmu^2=(1,6,3)=[abca]$
		\item $\bmu^3=(2,3)=[aca]$
		\item $\bmu^4=(1,4,5,2)=[abdca]$
		\item $\bmu^5=(6,5,4)=[acdb]$
		\item $\bmu^6=(1,4,5,3)=[abdca]$
	\end{itemize}
	We have $\bmu^1-\bmu^2+\bmu^3=\b0$
	\end{minipage} 
\end{frame}

\begin{frame}{An important result}
	\begin{importanttheorem}\label{th:cyclomatic_nb}
		Let $G$ be a graph with $n$ vertices, $m$ arcs and $p$ connected components. Then the cyclomatic number of $G$ is
		\[
		\nu(G)=m-n+p
		\]
	\end{importanttheorem}
\end{frame}


%%%%%%%%%%%%%%%%%%%%%%%%%%
%%%%%%%%%%%%%%%%%%%%%%%%%%
\subsection{Connectivity in digraphs}
\begin{frame}\frametitle{Definitions}
	%Let $D=(V,A)$ be a digraph.

	\begin{definition}[{Underlying graph}]
	Given a digraph, the undirected graph with each arc replaced by an edge is called the \defword{underlying graph}.
	\end{definition}


	\begin{definition}[{Weakly connected digraph}]
	If the underlying graph is a connected graph, then the digraph is \defword{weakly connected}.
	\end{definition}

	\begin{definition}[{Strongly connected digraph}]
	A digraph $G$ is \defword{strongly connected} if for every two distinct vertices $u$ and $v$ of $G$, there exists a directed path from $u$ to $v$.
	\end{definition}


	\begin{definition}[{Disconnected digraph}]
	A digraph is said to be \defword{disconnected}\index{disconnected digraph} if it is not weakly connected.
	\end{definition}
\end{frame}

\begin{frame}{Strong connectedness is an equivalence relation}
Denote $x\equiv y$ the relation  ``$x=y$, or $x\neq y$ and there exists a directed path in $G$ from $x$ to $y$''. $\equiv$ is an equivalence relation since
\begin{enumerate}
	\item $x\equiv y$ \hfill[reflexivity]
	\item $x\equiv y\implies y\equiv x$ \hfill[symmetry]
	\item $x\equiv y, y\equiv z\implies x\equiv z$ \hfill[transitivity]
\end{enumerate}
\begin{definition}[Connected component of a graph]
	\label{def:strong_components}
	Sets of the form 
	\[
	A(x_0)=\{x:x\in X, x\equiv x_0\}	
	\]
	are equivalence classes. 
	They partition $X$ into strongly connected sub-digraphs of $G$ called \defword{strongly connected components} (or \defword{strong components}) of $G$.
\end{definition}
A strong component in $G$ is a maximal strongly connected subdigraph of $G$.
\end{frame}


\begin{frame}
	Let $G=(V,A)$ be a digraph.

	\begin{theorem}{Properties}
	\begin{itemize}
	\item If a digraph is strongly connected, it has only one strongly connected component.
	\item The strongly connected components partition the vertices in the digraph, with every vertex in exactly one strongly connected component.
	\end{itemize}
	\end{theorem}
\end{frame}



\begin{frame}\frametitle{Algorithm for determining strongly connected components in $G=(V,A)$}
	\begin{itemize}
	\item Determine the strongly connected component $C(v)$ containing the vertex $v$; if $V-C(v)$ is non-empty, re-do the same operation on the sub-digraph $G'=(V-C(v),A')$.
	\item To determine $C(v)$, the strongly connected component containing $v$: let $v$ be a vertex of a digraph , which is not already in any strongly connected component. 
	%Set $C(v)=\{v\}$ the strongly connected component containing $v$.
	%
	%
	%
	%
	%%Repeat until $V-C(v)=\emptyset$:
	%Do for any $v\in V-C(v)$
	\begin{enumerate}
	\item Mark the vertex $v$ with $\pm$
	\item Mark with $+$ all successors (not already marked with $+$) of a vertex marked with $+$
	\item Mark with $-$ all predecessors (not already marked with $-$) of a vertex marked with $-$
	\item Repeat until no more possible marking with $+$ or $-$
	\end{enumerate}
	All vertices marked with $\pm$ belong to the same strongly connected component $C(v)$ containing the vertex $v$.
	\end{itemize}
\end{frame}

\begin{frame}\frametitle{Condensation of a digraph}
	\begin{definition}[{Condensation of a digraph}]
		The condensation $G^*$ of a digraph $G$ is a digraph having as vertices the strongly connected components (SCC) of $G$ and such that there exists an arc in $G^*$ from a SCC $C_i$ to another SCC $C_j$ if there is an arc in $G$ from some vertex of $S_i$ to a vertex of $S_j$.
	\end{definition}
\end{frame}


\begin{frame} 
	\begin{definition}[Articulation set]
		For a connected graph, a set $A$ of vertices is called an \defword{articulation set} (or a \defword{cutset}) if the subgraph of $G$ generated by $X-A$ is not connected
	\end{definition}
	\vfill
	\begin{definition}[Stable set]
		A set $S$ of vertices is called a stable set if no arc joins two distinct vertices in $S$
	\end{definition}
	\vfill
\end{frame}




%%%%%%%%%%%%%%%%%
%%%%%%%%%%%%%%%%%
\subsection{Orientable graphs}
\begin{frame}\frametitle{Orientation}
\begin{definition}[Orienting a graph]
Given a connected \emph{graph}, we describe the act of assigning a direction to each edge (edge $\rightarrow$ arc) as \defword{orienting the graph}.
\end{definition}
\begin{definition}[{Strong orientation}]
If the resulting digraph is strongly connected the orientation is called a \defword{strong orientation}.
\end{definition}
\end{frame}
 
 
\begin{frame}\frametitle{Orientable graph}
\begin{definition}[Orientable graph]
A connected graph $G$ is called \defword{orientable} if it is possible to assign a direction to each edge of $G$ to produce a strongly connected digraph $D$.
(If there exists a strong orientation of a connected graph, then the graph is orientable.)
\end{definition}

\begin{theorem}
A connected graph $G$ is orientable (has a strong orientation) if and only if $G$ contains no bridges; that is every edge is contained in a cycle.
\end{theorem}
\end{frame}
 






%%%%%%%%%%%%%%%%%%
%%%%%%%%%%%%%%%%%%
%%%%%%%%%%%%%%%%%%
%%%%%%%%%%%%%%%%%%
\section{Trees}

\begin{frame}\frametitle{Trees}
	\begin{definition}[Forest, trees and branches]
	\begin{itemize}
	\item A connected graph with no cycle is a \defword{tree}.
	\item A tree is a connected acyclic graph, its edges are called \defword{branches}.
	\item A graph (connected or not) without any cycle is a \defword{forest}. Each component is a tree. (A forest is a graph whose connected components are trees)
	\end{itemize}
	\end{definition}

	\begin{theorem}[{Properties}]
	\begin{itemize}
	\item Every edge of a tree is a bridge (the deletion of any edge of a tree diconnects it)
	\item Given two vertices $u $ and $v$ of a tree, there is an unique path linking $u$ to $v$.
	\item A tree with $p$ vertices and $q$ edges satisfies $q=p-1$. Thus, a tree is minimally connected. 
	\end{itemize}
	\end{theorem}
\end{frame}


\begin{frame}\frametitle{Spanning tree}
	\begin{definition}[Spanning tree]
	A \defword{spanning tree} of a connected graph $G$ is a subgraph of $G$ that contains all the vertices of $G$ and is a tree.
	\end{definition}
	A graph may have many spanning trees.
\end{frame}


\begin{frame}\frametitle{Minimal spanning tree}
	\begin{definition}[{Value of a spanning tree}]
	The \defword{value of a spanning tree} $T$ of order $p$ is
	$$\sum_{i=1}^{p-1}f(e_i)$$
	where $f$ is the function that maps the edge set into the set of real number.
	\end{definition}
	\begin{definition}[{Minimal spanning tree}]
	Let $G$ be an undirected network, and let $T$ be a \defword{minimal spanning tree} of $G$. Then $T$ is a spanning tree whose the value is minimum.
	\end{definition}
\end{frame}

\begin{frame}\frametitle{Algorithm to find a minimal spanning tree}
Let $G=(V(G),E(G))$ be an undirected network, and let $T$ be a minimal spanning tree.
 
\begin{enumerate}
\item sort the edges of $G$ in increasing order by value
\item $T=(V(G),\emptyset)$
\item for each edge $e$ in sorted order
        if the endpoints of $e$ are disconnected in $T$
        add $e$ to $T$
\end{enumerate}
\end{frame}

\begin{frame}\frametitle{Minimal connector problem}
\begin{itemize}
\item Model: a graph $G$ such that edges represent all possible connections, and each edge has a positive value which represents its cost; 
an undirected network $G$
\item Solution: a  minimal spanning tree $T$ of $G$
\begin{itemize}
\item a spanning tree of $G$ is a subgraph of $G$ that contains all the vertices of $G$ and is a tree.
\item the cost of the spanning tree is the sum of values of the edges of $T$
\item a spanning tree such that no other spanning tree has a smaller cost is a minimmal spanning tree.
\end{itemize}
\end{itemize}
\end{frame}



\begin{frame}
\begin{theorem}[Characterisation of trees]\label{th:characterisation_tree}
$H=(X,U)$ a graph of order $|X|=n>2$. The following are equivalent and all characterise a tree :
\begin{enumerate}
\item $H$ connected and has no cycles
\item $H$ has $n-1$ arcs and no cycles
\item $H$ connected and has exactly $n-1$ arcs
\item $H$ has no cycles, and if an arc is added to $H$, exactly one cycle is created
\item $H$ connected, and if any arc is removed, the remaining graph is not
connected
\item Every pair of vertices of $H$ is connected by one and only one chain
\end{enumerate}
\end{theorem}
\end{frame}


\begin{frame}{Proof of Theorem~\ref{th:characterisation_tree}}
We make abundant use of Theorem~\ref{th:cyclomatic_nb}
\vfill
[$\mathbf{1\implies 2}$] 
Let $p$ be the number of connected components, $m$ the number of arcs and $\nu(H)$ the cyclomatic number. Since $H$ connected, $p=1$. Since $H$ has no cycles, $\nu(H)=m-n+p=0$

$\implies$ $m=n-p=n-1$
\vfill

[$\mathbf{2\implies 3}$] 
Assume $H$ has no cycles ($\nu(H)=0$) and has $n-1$ arcs ($m=n-1$). Then, since
\[
\nu(H)=m-n+p
\]
$p=\nu(H)-m+n=0-(n-1)-n=1$, i.e., $H$ is connected
\end{frame}

\begin{frame}{Proof of Theorem~\ref{th:characterisation_tree} (cont.)}
[$\mathbf{3\implies 4}$]
Assume $H$ connected ($p=1$) and contains exactly $n-1$ arcs ($m=n-1$). Then
\[
\nu(H)=m-n+p=(n-1)-n+1=0
\]
and $H$ has no cycles
\vskip0.5cm
Now add an arc, i.e., suppose $m=n$. Then $\nu(H)=m-n+p=n-n+1=1$ and there is one cycle in the new graph
\end{frame}


\begin{frame}{Proof of Theorem~\ref{th:characterisation_tree} (cont.)}
[$\mathbf{4\implies 5}$]
Assume $H$ has no cycles ($\nu(H)=0$) and that addition of an arc to $H$ creates exactly one cycle
\vskip0.5cm
Suppose $H$ not connected. Then there are two vertices, say $a$ and $b$, that are not connected and adding the arc $(a,b)$ does not create a cycle, a contradiction with ``addition of an arc to $H$ creates exactly one cycle'' 

$\implies$ $p=1$. Since $\nu(H)=0$, this implies that $m=n-1$
\vskip0.5cm
Now suppose we remove an arc. We obtain graph $H'$ with
\[
m'=n'-2\quad\text{and}\quad \nu(H')=0
\]
So
\[
p'=\nu(H')-m'+n'=2
\]
$\implies$ $H'$ not connected
\end{frame}

\begin{frame}{Proof of Theorem~\ref{th:characterisation_tree} (cont.)}
[$\mathbf{5\implies 6}$]
Assume $H$ connected and if any arc is removed, the remaining graph is not connected
\vskip0.5cm
Any vertices $a,b\in X$ are connected by a chain (since $H$ connected). That chain is unique: suppose there is a second chain connecting $a$ to $b$; then removing an arc from that chain does not disconnect the graph, since there is still the original chain connecting $a$ and $b$
\vfill
[$\mathbf{6\implies 1}$]
Assume every pair of vertices of $H$ is connected by one and only one chain

Now assume $H$ has a cycle. Then at least one pair of vertices would be connected by two distinct chains, a contradiction
$\qed$
\end{frame}


\begin{frame}
\begin{definition}[Pendant vertex]
A vertex is \defword{pendant} if it is adjacent to exactly one other vertex
\end{definition}
\vfill
\begin{theorem}
A tree of order $n\geq 2$ has at least two pendant vertices
\end{theorem}
\vfill
\textbf{Proof:} 
Suppose $H$ tree of order $n\geq 2$ with 0 or 1 pendant vertices
\vskip0.5cm
Consider a traveller traversing the graph edges starting from a pendant vertex (if there is one) or anywhere if there is no pendant vertex
\vskip0.5cm
If he does not allow himself to use same edge twice, he cannot go to the same vertex twice ($H$ has no cycle)
\vskip0.5cm
If he arrives at vertex $x$, he can depart $x$ using a new edge ($x$ is not a pendant vertex as there are 0 or 1 pendant vertex and if there is 1, that's where he started)
\vskip0.5cm
So the trip continues without end, which is impossible as $H$ finite $\qed$
\end{frame}

\begin{frame}
\begin{theorem}\label{th:partial_graph_tree_G_connected}
A graph $G=(X,U)$ has a partial graph that is a tree $\iff$ $G$ connected
\end{theorem}
\vfill
Recall that a partial graph is a graph generated by a subset of the arcs (Definition~\ref{def:partial_graph} slide~\pageref{def:partial_graph})
\vfill
\textbf{Proof:}
Suppose $G$ not connected. Then no partial graph of $G$ is connected $\implies$ there is no partial graph of $G$ that is a tree [we want to show $P\iff Q$, we start with $\wedge Q\implies \wedge P$ (which $\iff$ $P\implies Q$)]
\vskip1cm
Suppose $G$ connected. Look for an arc whose removal does not disconnect $G$
\begin{itemize}
\item if there is none, $G$ is a tree by Theorem~\ref{th:characterisation_tree}(5)
\item if there is, remove it and look for another one, etc. When no more arcs can be removed, the remaining graph is a tree with vertex set $X$ $\qed$
\end{itemize}
\end{frame}

\begin{frame}{Spanning tree}
The procedure in the proof of Theorem~\ref{th:partial_graph_tree_G_connected} gives a \defword{spanning tree}
\vfill
Can also build a spanning tree as follows:
\begin{itemize}
\item Consider any arc $u_0$
\item Find arc $u_1$ that does not form a cycle with $u_0$
\item Find arc $u_2$ that does not form a cycle with $\{u_0,u_1\}$
\item Continue
\item When you cannot continue anymore, you have a spanning tree
\end{itemize}
\end{frame}




\begin{frame}
\begin{theorem}
$G$ connected graph with $\geq 1$ arc. TFAE
\begin{enumerate}
\item $G$ strongly connected
\item Every arc lies on a circuit
\item $G$ contains no cocircuits
\end{enumerate}
\end{theorem}
\vfill
\textbf{Proof:}
[$\mathbf{1\implies 2}$]
$(x,y)$ an arc of $G$; there is a path from $y$ to $x$ ($G$ strongly connected), so arc $(x,y)$ is contained in a circuit of $G$

[$\mathbf{2\implies 3}$]
Suppose $G$ has a cocircuit containing arc $(x,y)$; then $G$ cannot have
a circuit containing this arc by the Arc Colouring Lemma (Lemma~\ref{lemma:arc_colouring_lemma} slide~\pageref{lemma:arc_colouring_lemma}) with
all arcs coloured black. This contradicts (2)

[$\mathbf{3\implies 1}$]
Assume $G$ connected graph without cocircuits, but $G$ not strongly connected.
Since $G$ not strongly connected, it has $>1$ strongly
connected component. 
Since $G$ connected, there exist two distinct strongly connected components that are joined by an arc $(a,b)$. Arc $(a,b)$ is not contained in any circuit because otherwise $a$ and $b$ would be in the same strongly connected component. 
By Lemma~\ref{lemma:arc_colouring_lemma}, arc $(a,b)$ is contained in some co-
circuit. This contradicts (3)\qed
\end{frame}


\begin{frame}
\begin{theorem}
$G$ graph with $\geq 1$ arc. TFAE
\begin{enumerate}
\item $G$ is a graph without circuits
\item Each arc is contained in a cocircuit
\end{enumerate}
\end{theorem}
\vfill
\begin{theorem}
If $G$ is a strongly connected graph of order $n$, then $G$ has a cycle basis of $\nu(G)$ circuits
\end{theorem}
\end{frame}

\begin{frame}
\begin{definition}[Node, anti-node, branch]
$G=(X,U)$ strongly connected without loops and $>1$ vertex. For each $x\in X$, there is a path from it and a path to it so $x$ has at least 2 incident arcs. Specifically,
\begin{itemize}
\item $x\in X$ with $>2$ incident arcs is a \defword{node}
\item $x\in X$ with $2$ incident arcs is an \defword{anti-node}
\end{itemize}
A path whose only nodes are its endpoints is a \defword{branch}
\end{definition}
\vfill
\begin{definition}[Minimally connected graph]
$G$ is \defword{minimally connected} if it is strongly connected and removal of any arc destroys strong-connectedness
\end{definition}
A minimally connected graph is 1-graph without loops
\vfill
\begin{definition}[Contraction]
$G=(X,U)$. The \defword{contraction} of the set $A\subset X$ of vertices consists in replacing $A$ by a single vertex $a$ and replacing each arc into (resp. out of) $A$ by an arc with same index into (resp. out of) $a$
\end{definition}
\end{frame}


\begin{frame}
\begin{theorem}
$G$ minimally connected, $A\subset X$ generating a strongly connected subgraph of $G$. Then the contraction of $A$ gives a minimally connected graph
\end{theorem}
\vfill
\textbf{Proof:}
First, show that contraction of $A$ yields a 1-graph. 
If this were not the case, there would exist $x\not\in A$ and $a,a'\in A$ s.t. $(x,a),(x,a')\in U$ (or, with $(a,x),(a',x)\in U$ but this would not
change the proof). 
If one of these arcs is removed, the graph remains strongly
connected. Thus, $G$ is not minimally connected, a contradiction

\vskip1cm
Now show that the contraction of $A$ yields a graph $G'$ that is minimally connected. Clearly, $G'$ strongly connected. If an arc $u$ is removed, the remaining graph is not strongly connected, since the graph
$(X,U-\{u\})$ not strongly connected \qed
\end{frame}

\begin{frame} 
\begin{theorem}
$G$ a minimally connected graph, $G'$ be the minimally connected graph obtained by the contraction of an elementary circuit of $G$.	Then
\[
\nu(G)=\nu(G')+1
\]
\end{theorem}
\vfill
\begin{theorem}
$G$ minimally connected of order $n\geq 2$ $\implies$ $G$ has $\geq 2$ anti-nodes
\end{theorem}
\vfill
\begin{theorem}
$G=(X,U)$. Then the graph $C'$ obtained by contracting each strongly connected component of $G$ contains no circuits
\end{theorem}
\end{frame}


\begin{frame}{Arborescences}
\begin{definition}[Root]
Vertex $a\in X$ in $G=(X,U)$ is a \defword{root} if all vertices of $G$ can be reached by paths \emph{starting} from $a$
\end{definition}
Not all graphs have roots
\vfill
\begin{definition}[Quasi-strong connectedness]
$G$ is \defword{quasi-strongly connected} if $\forall x,y\in X$, exists $z\in X$ (denoted $z(x,y)$ to emphasize dependence on $x,y$) from which there is a path to $x$ and a path to $y$
\end{definition}
Strongly connected $\implies$ quasi-strongly connected (take $z(x,y)=x$); converse not true

Quasi-strongly connected $\implies$ connected
\end{frame}

\begin{frame}{Arborescence}
\begin{definition}[Arborescence]
An \defword{arborescence} is a tree that has a root
\end{definition}
\vfill
\begin{lemma}
$G=(X,U)$ has a root $\iff$ $G$ quasi-strongly connected
\end{lemma}
\end{frame}

\begin{frame}
\begin{theorem}
$H$ graph of order $n>1$. TFAE (and all characterise an arborescence)
\begin{enumerate}
\item $H$ quasi-strongly connected without cycles
\item $H$ quasi-strongly connected with $n-1$ arcs
\item $H$ tree having a root $a$
\item $\exists a\in X$ s.t. all other vertices are connected with $a$ by 1 and only 1 path from $a$
\item $H$ quasi-strongly connected and loses quasi-strong connectedness if any arc is removed
\item $H$ quasi-strongly connected and $\exists a\in X$ s.t.
\begin{align*}
d_H^-(a)&=0 \\
d_H^-(x)&=1\qquad \forall x\neq a
\end{align*}
\item $H$ has no cycles and $\exists a\in X$ s.t.
\begin{align*}
d_H^-(a)&=0 \\
d_H^-(x)&=1\qquad \forall x\neq a
\end{align*}
\end{enumerate}
\end{theorem}
\end{frame}


\begin{frame}
\begin{theorem}
$G$ has a partial graph that is an arborescence $\iff$ $G$ quasi-strongly connected
\end{theorem}
\vfill
\begin{theorem}
$G=(X,E)$ simple connected graph and $x_1\in X$. It is possible to direct all edges of $E$ so that the resulting graph $G_0=(X,U)$ has a spanning tree $H$ s.t.
\begin{enumerate}
	\item $H$ is an arborescence with root $x_1$
	\item The cycles associated with $H$ are circuits
	\item The only elementary circuits of $G_0$ are the cycles associated with $H$
\end{enumerate}
\end{theorem}
\end{frame}


\begin{frame}{Counting trees}
\begin{proposition}
$X$ a set with $n$ distinct objects, $n_1,\ldots,n_p$ nonnegative integers s.t. $n_1+\cdots+n_p=n$. The number of ways to place the $n$ objects into $p$ boxes $X_1,\ldots,X_p$ containing $n_1,\ldots,n_p$ objects respectively is 
\[
	\begin{pmatrix}
		n\\ n_1,\ldots,n_p
	\end{pmatrix}
	=
	\frac{n!}{n_1!\cdots n_p!}
\]
\end{proposition}
\vfill
\begin{proposition}[Multinomial formula]
Let $a_1,\ldots,a_p\in\IR$ be $p$ real numbers, then
\[
	(a_1+\cdots+a_p)^n= 
	\sum_{n_1,\ldots,n_p\geq 0}
	\begin{pmatrix}
		n\\ n_1,\ldots,n_p
	\end{pmatrix}
	(a_1)^{n_1}\cdots(a_p)^{n_p}	
\]
\end{proposition}
\end{frame}


\begin{frame}
\begin{theorem}
Denote $T(n;d_1,\ldots,d_n)$ the number of distinct trees $H$ with vertices $x_1,\ldots,x_n$ and with degrees $d_H(x_1)=d_1,\ldots,d_H(x_n)=d_n$. Then
\[
	T(n;d_1,\ldots,d_n) = 
	\begin{pmatrix}
		n-2 \\
		d_1-1,\ldots,d_n-1
	\end{pmatrix}
\]
\end{theorem}
\vfill
\begin{theorem}
The number of different trees with vertices $x_1,\ldots,x_n$ is 
$n^{n-2}$
\end{theorem}
\vfill
There is a whole industry of similar results (as well as for arborescences), but we will stop here. The main point is that we are talking about a large number of possibilities..
\end{frame}


%%%%%%%%%%%%%%%%%%%%%%%%%%
%%%%%%%%%%%%%%%%%%%%%%%%%%
%%%%%%%%%%%%%%%%%%%%%%%%%%
%%%%%%%%%%%%%%%%%%%%%%%%%%
\section{Matrices associated to a graph/digraph}

\begin{frame}{Matrices associated to a graph/digraph}
	There are multiple matrices associated to a graph/digraph.

	The branch of graph theory that studies the properties of matrices derived from graphs and uses of these matrices in determining graph properties is \emph{spectral graph theory}.
	
	Graphs greatly simplify some problems in linear algebra and vice versa.
\end{frame}

\subsection{Adjacency matrices}


\begin{frame}\frametitle{Adjacency matrix (undirected case)}
	Let $G=(V,E)$ be a graph of order $p$ and size $q$, with vertices $v_1, v_2, \dots , v_p$ and edges $e_1, e_2, \dots , e_q$.
	\begin{definition}[{Adjacency matrix}]
		The adjacency matrix is $$M_A=M_A(G)=[m_{ij}]$$ is a $p\times p$ matrix in which
		$$m_{ij}=\left \{ 
		\begin{array}{cc}
			1 & \textrm{if } v_i \textrm{ and } v_j \textrm{ are adjacent}\\
			0 & \textrm{otherwise}
		\end{array}
		\right .
		$$
	\end{definition}
\end{frame}

\begin{frame}
	\begin{theorem}[{Adjacency matrix and degree}]
		The sum of the entries in row $i$ of the adjacency matrix is the degree of $v_i$ in the graph.
	\end{theorem}
\end{frame}

\begin{frame}
	\begin{definition}[Multiplicity of a pair]
		The \defword{multiplicity} of a pair $x,y$ is the number $m_G^+(x,y)$ of arcs with initial endpoint $x$ and terminal endpoint $y$. Let
		\begin{align*}
			m_G^-(x,y) &= m_G^+(y,x) \\
			m_G(x,y) &= m_G^+(x,y)+m_G^-(x,y)
		\end{align*}
		If $x\neq y$, then $m_G(x,y)$ is number of arcs with both $x$ and $y$ as endpoints. If $x=y$, then $m_G(x,y)$ equals twice the number of loops attached to vertex $x$. If $A,B\subset X$, $A\neq B$, let
		\begin{align*}
			m_G^+(A,B) &= \{u:u\in U, u=(x,y),x\in A,y\in B\} \\
			m_G(A,B) &= m_G^+(A,B)+m_G^+(A,B)
		\end{align*}
	\end{definition}
\end{frame}

\begin{frame}{Adjacency matrix of a multigraph}
\begin{definition}[Matrix associated with $G$]
	If $G$ has vertices $x_1,x_2,\ldots,x_n$, then the \defword{matrix associated} with $G$ is 
	\[
	a_{ij}=m_G^+(x_i,x_j)
	\]
\end{definition}
\vfill
\begin{definition}[Adjacency matrix]
	The matrix $a_{ij}+a_{ji}$ is the \defword{adjacency matrix} associated with $G$
\end{definition}
\end{frame}

\begin{frame}\frametitle{Adjacency matrix}
	Let $D=(V,A)$ be a digraph of order $p$ with vertices denoted by $v_1, v_2, \dots , v_p$.
	\begin{definition}[Adjacency matrix]
	The adjacency matrix $M=M(D)=[m_{ij}]$ is a $p\times p$ matrix in which
	$$m_{ij}=\left \{ 
	\begin{array}{cl}
	1 & \textrm{if arc } v_iv_j \in A\\
	0 & \textrm{otherwise}
	\end{array}
	\right .
	$$
	\end{definition}
	\begin{theorem}[{Properties}]
	\begin{itemize}
	\item $M$ is not necessarily symmetric.
	\item The sum of any column of $M$ is equal to the number of arcs directed towards $v_j$
	\item The sum of the entries in row $i$ is equal to the number of arcs directed away from vertex $v_i$.
	\item The $(i,j)-$entry of $M^n$ is equal to the number of walks of length $n$ from vertex $v_i$ to $v_j$.
	\end{itemize}
	\end{theorem}
\end{frame}
	
\begin{frame}\frametitle{Incidence matrix}
	Let $D=(V,A)$ be a digraph of order $p$, and size $q$, with vertices denoted by $v_1, v_2, \dots , v_p$, and arcs denoted $a_1, a_2, \dots , a_q$.
	\begin{definition}{Definition}
	The incidence matrix $B=B(D)=[b_{ij}]$ is a $p\times q$ matrix in which
	$$b_{ij}=\left \{ 
	\begin{array}{cl}
	1 & \textrm{if arc } a_j  \textrm{ is directed away from a vertex } v_i\\
	-1 & \textrm{if arc } a_j  \textrm{ is directed towards a vertex } v_i\\
	0 & \textrm{otherwise}
	\end{array}
	\right .
	$$
	\end{definition}
\end{frame}
	
	
	
	
\begin{frame}{Adjacency matrix}
	We have already seen adjacency matrices, let us recall the definition here
	\begin{definition}[Adjacency matrix]
		$G$ a 1-graph, then the \defword{adjacency matrix} $A=[a_{ij}]$ is defined as follows
		\[
		a_{ij} = \begin{cases}
			1 & \text{if arc } (i,j)\in U\\
			0 & \text{otherwise}
		\end{cases}
		\]
	\end{definition}
	\vfill
	We often write $A(G)$ and, reciprocally, if $A$ is an adjacency matrix, $G(A)$ the corresponding graph
	\vfill
	$G$ undirected $\implies$ $A(G)$ symmetric
	\vfill
	$A(G)$ has nonzero diagonal entries if $G$ is not simple
\end{frame}

\begin{frame}{Adjacency matrix (multigraph case)}
	\begin{definition}[Adjacency matrix of a multigraph]
		$G$ an $\ell$-graph, then the adjacency matrix $M_A=[m_{ij}]$ is defined as follows
		\[
		m_{ij} = \begin{cases}
			k & \text{if arc there are $k$ arcs } (i,j)\in U\\
			0 & \text{otherwise}
		\end{cases}
		\]
		with $k\leq \ell$
	\end{definition}
	\vfill
	$G$ undirected $\implies$ $M_A(G)$ symmetric
	\vfill
	$M_A(G)$ has nonzero diagonal entries if $G$ is not simple.
\end{frame}

\begin{frame}{Weighted adjacency matrices}
	Sometimes, adjacency matrices (typically for 1-graphs) have real entries, usually positive
	\vfill
	This means that the arcs/edges have been given a weight
\end{frame}


\begin{frame}
	\begin{theorem}[{Number of walks of length $n$}]
		Let $A$ be the adjacency matrix of a graph $G=(V(G),E(G))$, where $V(G)=\{v_1,v_2, \dots , v_p\}$. Then the $(i,j)-$entry of $A^n$, $n\geq 1$, is the number of different walks linking $v_i$ to $v_j$ of length $n$ in $G$.
	\end{theorem}
	(two walks of the same length are equal if their edges occur in exactly the same order)
	
	Example: let $A$ be the adjacency matrix of a graph $G=(V(G),E(G))$. 
	\begin{itemize}
		\item the $(i,i)-$entry of $A^2$ is equal to the degree of $v_i$.
		\item the $(i,i)-$entry of $A^3$ is equal to twice the number of $C_3$ containing $v_i$.
	\end{itemize}
\end{frame}

\subsection{Other matrices associated to a graph/digraph}

\begin{frame}\frametitle{Incidence matrix}
	Let $G=(V,E)$ be a graph of order $p$, and size $q$, with vertices denoted by $v_1, v_2, \dots , v_p$, and edges denoted $e_1, e_2, \dots , e_q$.
	\begin{definition}[{Incidence matrix}]
		The incidence matrix is $$B=B(G)=[b_{ij}]$$ is that $p\times q$ matrix in which
		$$b_{ij}=\left \{ 
		\begin{array}{cc}
			1 & \textrm{if } v_i \textrm{ is incident with } e_j\\
			0 & \textrm{otherwise}
		\end{array}
		\right .
		$$
	\end{definition}
	\begin{theorem}[{Incidence matrix and degrees}]
		The sum of the entries in row $i$ of the incidence matrix is the degree of $v_i$ in the graph.
	\end{theorem}
\end{frame}

\begin{frame}{Incidence matrix}
	\begin{definition}[Incidence matrix -- Undirected case]
		$G=(X,E)$ an undirected graph of order $n$ with $p$ edges. The \defword{incidence} matrix of $G$ is an $n\times p$ matrix with vertices as rows and edges as columns and where $B=[b_{ij}]$ satisfies
		\[
		b_{ij} = \begin{cases}
			1 & \text{if edge }j\text{ is incident to vertex }i \\
			0 & \text{otherwise}
		\end{cases}
		\]
	\end{definition}
	\vfill
	\begin{definition}[Incidence matrix -- Directed case]
		$G=(X,U)$ a directed graph of order $n$ with $p$ arcs. The \defword{incidence} matrix of $G$ is an $n\times p$ matrix with vertices as rows and edges as columns and where $B=[b_{ij}]$ satisfies
		\[
		b_{ij} = \begin{cases}
			1 & \text{if arc }j\text{ ``enters'' vertex }i \\
			-1 & \text{if arc }j\text{ ``leaves'' vertex }i \\
			0 & \text{otherwise}
		\end{cases}
		\]
	\end{definition}
\end{frame}




\begin{frame}{Spectrum of a graph}
	We will come back to this later, but for now..
	\vfill
	\begin{definition}[Spectrum of a graph]
		The \defword{spectrum} of a graph $G$ is the spectrum (set of eigenvalues) of its associated adjacency matrix $A(G)$
	\end{definition}
	\vfill
	This is regardless of the type of adjacency matrix or graph
\end{frame}





\begin{frame}{Degree matrix}
	\begin{definition}[Degree matrix]
		The \defword{degree} matrix $D=[d_{ij}]$ for $G$ is a $n \times n$ diagonal matrix defined as
		\[
			d_{ij} = \begin{cases}
				d_G(v_i) & \text{if } i = j \\
				0 & \text{otherwise}
			\end{cases}
		\]
	\end{definition}
\vfill
In an undirected graph, this means that each loop increases the degree of a vertex by two
\vfill
In a directed graph, the term ``degree'' may refer either to indegree (the number of incoming edges at each vertex) or outdegree (the number of outgoing edges at each vertex)
\end{frame}

\begin{frame}{Laplacian matrix}
\begin{definition}
	$G=(X,U)$ a simple graph with $n$ vertices. The \defword{Laplacian} matrix is 
	\[
		L=D(G)-A(G)
	\]
	where $D(G)$ is the degree matrix and $A(G)$ is the adjacency matrix
\end{definition}
$G$ simple graph $\implies$ $A(G)$ only contains 1 or 0 and its diagonal elements are all 0
\vfill
For directed graphs, either the indegree or outdegree is used, depending on the application
\vfill
Elements of $L$ are given by
\[
\ell_{ij}=\begin{cases}
	d_G(v_{i}) & \text{if } i=j\\
	-1 & \text{if } i\neq j \text{ and } v_{i} 
	\text{ is adjacent to }v_{j}\\
	0 & \text{otherwise}
\end{cases}
\]
\end{frame}


\begin{frame}\frametitle{Distance matrix}
	Let $G$ be a graph of order $p$ with vertices denoted by $v_1, v_2, \dots , v_p$.
	\begin{definition}[Distance matrix]
	The distance matrix $DIST=DIST(D)=[d_{ij}]$ is a $p\times p$ matrix in which
	$$d_{ij}=
	dist(v_i,v_j).
	$$
	Note $d_{ii}=0$ for $i=1,\cdots, p$.
	\end{definition}

	\begin{definition}{Properties}
		\begin{itemize}
		\item $M$ is not necessarily symmetric.
		\item The sum of any column of $M$ is equal to the number of arcs directed towards $v_j$
		\item The sum of the entries in row $i$ is equal to the number of arcs directed away from vertex $v_i$.
		\item The $(i,j)-$entry of $M^n$ is equal to the number of walks of length $n$ from vertex $v_i$ to $v_j$.
		\end{itemize}
	\end{definition}
\end{frame}



\subsection{Linking graphs and linear algebra}

\begin{frame}{Counting paths}
	To count paths between vertices $x$ and $y$ in a graph, we use the adjacency matrix
	\vfill
	\begin{theorem}
		$G$ a graph and $A(G)$ its adjacency matrix. Denote $P=[p_{ij}]$ the matrix $P=A^k$. Then $p_{ij}$ is the number of distinct paths of length $k$ from $i$ to $j$ in $G$
	\end{theorem}
	\vfill
	This provides an interesting connection with linear algebra
	\begin{definition}[Irreducible matrix]
		A matrix $A\in\M_n$ is \defword{reducible} if $\exists P\in\M_n$, permutation matrix, s.t. $P^TAP$ can be written in block triangular form. If no such $P$ exists, $A$ is \defword{irreducible}
	\end{definition}
	\vfill
	\begin{theorem}
		$A$ irreducible $\iff$ $G(A)$ strongly connected
	\end{theorem}
\end{frame}

\begin{frame}
	\begin{theorem}
		Let $A$ be the adjacency matrix of a graph $G$ on $p$ vertices. 
		A graph $G$ on $p$ vertices is connected if and only if $$I+A+A^2+\dots+A^{p-1}=C$$
		has no zero entries.
	\end{theorem}
	\begin{theorem}
		Let $M$ be the adjacency matrix of a digraph $D$ on $p$ vertices. 
		A digraph $D$ on $p$ vertices is strongly connected if and only if $$I+M+M^2+\dots+M^{p-1}=C$$
		has no zero entries.
	\end{theorem}
\end{frame}


\begin{frame}{Perron-Frobenius theorem}
\vfill
$A=[a_{ij}]\in\M_n(\IR)$ \defword{nonnegative} if $a_{ij}\geq 0$ $\forall i,j=1,\ldots,n$; $\bv\in\IR^n$ nonnegative if $v_i\geq 0$ $\forall i=1,\ldots,n$.  \defword{Spectral radius} of $A$
\[
\rho(A) = \max_{\lambda\in\mathsf{Sp}(A)}\{|\lambda|\}
\]
$\mathsf{Sp}(A)$ the \defword{spectrum} of $A$
\vfill
\begin{theorem}[PF -- Nonnegative case]
$0\leq A\in\M_n(\IR)$. Then $\exists\bv\geq\b0$ s.t. 
\[
	A\bv=\rho(A)\bv
\]
\end{theorem}
\vfill

\begin{theorem}[PF -- Irreducible case]
	\label{th:PF}
	Let $0\leq A\in \M_n(\IR)$ irreducible. Then $\exists\bv>\b0$ s.t.
	\[
		A\bv=\rho(A)\bv
	\]
	$\rho(A)>0$ and with algebraic multiplicity 1.
	No nonnegative eigenvector is associated to any other eigenvalue of $A$
\end{theorem}
\end{frame}



\begin{frame}{Primitive matrices}
\begin{definition}
$0\leq A\in\M_n(\IR)$ \defword{primitive} (with \defword{primitivity index} $k\in \IN_{+}^{*}$) if $\exists k\in \IN_{+}^{*}$ s.t.
\[
A^k>0,
\]
with $k$ the smallest integer for which this is true.
$A$ \defword{imprimitive} if it is not primitive
\end{definition}
\vfill
$A$ primitive $\implies$ $A$ irreducible; converse false
\vfill
\begin{theorem}
	$A\in\M_n(\IR)$ irreducible and $\exists i=1,\ldots,n$ s.t. $a_{ii}>0$ $\implies$ $A$ primitive
\end{theorem}
\vfill
Here $d$ is the index of imprimitivity (i.e., the number of eigenvalues that have the same modulus as $\lambda_p=\rho(A)$). If $d=1$, then $A$ is primitive. We have that $d=\mathsf{gcd}$ of all the lengths of closed walks in $G(A)$
\end{frame}

\begin{frame}
\begin{minipage}{0.49\textwidth}
	\begin{tikzpicture}[auto, %node distance = 2cm, auto,
		cloud/.style={minimum width={width("N-1")+2pt},
		draw, ellipse}]
		\node [cloud] (S1) at (0,0) {$1$};
		\node [cloud] (S2) at (2,0) {$2$};
		\node [cloud] (S3) at (1,-2) {$3$};
		%% Recoveries
		\path [line, thick] (S1) to [bend left = 20] (S2);
		\path [line, thick] (S2) to [bend left = 20] (S3);
		\path [line, thick] (S3) to [bend left = 20] (S1);
	\end{tikzpicture}
\end{minipage}
\begin{minipage}{0.49\textwidth}
Adjacency matrix
\[
	A=\begin{pmatrix}
	0&1& 0\\
	0&0&1\\
	1&0&0
	\end{pmatrix}
\]
\end{minipage}
\vskip0.2cm
Closed walks in $G(A)$ (lengths): $1\to 1$ (3), $2\to 2$ (3), $2\to 2$ (3)
$\implies$ $\gcd = 3$ $\implies$ $d=3$ (here, all eigenvalues have modulus $1$)
\vfill
\begin{minipage}{0.49\textwidth}
\begin{tikzpicture}[auto, %node distance = 2cm, auto,
cloud/.style={minimum width={width("N-1")+2pt},
draw, ellipse}]
\node [cloud] (S1) at (0,0) {$1$};
\node [cloud] (S2) at (2,0) {$2$};
\node [cloud] (S3) at (1,-2) {$3$};
%% Loops
\path [line, thick, distance=30] (S1) to (S1);
%% Recoveries
\path [line, thick] (S1) to [bend left = 20] (S2);
\path [line, thick] (S2) to [bend left = 20] (S3);
\path [line, thick] (S3) to [bend left = 20] (S1);
\end{tikzpicture}
\end{minipage}
\begin{minipage}{0.49\textwidth}
\[
A= \begin{pmatrix}
1&1&0\\
0&0&1\\
1&0&0
\end{pmatrix}
\]
\end{minipage}
\vskip0.2cm
Closed walk $1\to 1$ has length 1 $\implies$
gcd of lengths of closed walks is 1 $\implies$ $A$ primitive
\end{frame}

\begin{frame}
\begin{theorem}
$0\leq A\in \M_n$. $A$ primitive $\implies$ $A^k>0$ for some $0<k\leq (n-1)n^n$
\end{theorem}
\vfill
\begin{theorem}
$A\geq 0$ primtive. Suppose the shortest simple directed cycle in $G(A)$ has length $s$, then primitivity index is $\leq n+s(n-1)$
\end{theorem}
\vfill
\begin{theorem}
$0\leq A\in \M_n$ primitive $\iff$ $A^{n^2-2n+2}>0$
\end{theorem}
\vfill
\begin{theorem}
	$0\leq A\in \M_n$ irreducible. $A$ has $d$ positive entries on the diagonal $\implies$ primitivity index $\leq 2n-d-1$
\end{theorem}
\end{frame}


\begin{frame}
\begin{theorem}\label{th:PF_allCases}
	$0\leq A\in\M_n$, $\lambda_P=\rho(A)$ the Perron root of $A$, $\bv_P$ and $\bw_P$ the corresponding right and left Perron vectors of $A$, respectively, $d$ the index of imprimitivity of $A$ (with $d=1$ when $A$ is primitive) and $\lambda_j\in\sigma(A)$ the spectrum of $A$, with $j=2,\ldots,n$ unless otherwise specified (assuming $\lambda_1=\lambda_P$) 
	\vskip0.5cm
	{\centering
		\begin{tikzpicture}[scale=0.75, every node/.style={transform shape},
			auto,
		cloud/.style={minimum width={width("N-1")+2pt},
			draw, rectangle}]
		\node [cloud] at (10,0) (nonneg) {\sc Nonnegative};
		\node [cloud] (reduc) at (2,-3) [text width=3cm] {{\sc Reducible}\\ \noindent\begin{itemize}
			\item $\lambda_P\geq 0$
			\item $\bw_P\geq 0$
			\item $\bv_P\geq 0$
			\item $\lambda_P \geq | \lambda_j |$
			\end{itemize}
		};
		\node [cloud] (irred) at (12,-2) {\sc Irreducible};	
		\node [cloud] (notprim) at (8,-4) [text width=3cm] {{\sc Imprimitive}\\ \begin{itemize}
			\item $\lambda_P> 0$
			\item $\bw_P> 0$
			\item $\bv_P> 0$
			\item $\lambda_P=|\lambda_j|$, $j=2,\ldots,d$
			\item $\lambda_P > |\lambda_j|$, $j>d$
			\end{itemize}
		};	
		\node [cloud] (prim) [text width=3cm] at (16,-4) {{\sc Primitive}\\
			\begin{itemize}
			\item $\lambda_P> 0$
			\item $\bw_P> 0$
			\item $\bv_P> 0$
			\item $\lambda_P>|\lambda_j|$, $j\neq P$ 
			\end{itemize}
		};	
		%% Links
		\path [line, thick] (nonneg) to (reduc);
		\path [line, thick] (nonneg) to (irred);
		\path [line, thick] (irred) to (notprim);
		\path [line, thick] (irred) to (prim);
		\end{tikzpicture}
	}
	\end{theorem}	
\end{frame}

%%%%%%%%%%%%%%%%%%%%%%%%
%%%%%%%%%%%%%%%%%%%%%%%%
%%%%%%%%%%%%%%%%%%%%%%%%
%%%%%%%%%%%%%%%%%%%%%%%%
% \section{Index of terms used}
% \begin{frame}{Index}
% 	\printindex
% \end{frame}

\end{document}
