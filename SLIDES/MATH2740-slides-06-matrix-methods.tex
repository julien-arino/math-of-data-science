\documentclass[aspectratio=169]{beamer}

\usetheme{default}
% Slide setup, colour independent

\usepackage{amsmath,amssymb,amsthm}
\usepackage[utf8]{inputenc}
\usepackage{colortbl}
\usepackage{bm}
\usepackage{xcolor}
\usepackage{dsfont}
\usepackage{setspace}
% To use \ding{234} and the like
\usepackage{pifont}
% To cross reference between slide files
\usepackage{zref-xr,zref-user}
% Use something like
% \zexternaldocument{fileI}
% in the tex files. And cite using \zref instead of \ref

% Cross-reference system - see CROSS-REFERENCE-SETUP.md for manual setup instructions
\usepackage{booktabs}
\usepackage{marvosym}
\usepackage{cancel}
%\usepackage{transparent}
% Make doi clickable in the bibliography?
\usepackage{doi}

\usepackage[T1]{fontenc}

\usepackage{longtable}

% For heavier titles
\usepackage{helvet} % Enables Helvetica font family


% Fields and the like
\def\IC{\mathbb{C}}
\def\IE{\mathbb{E}}
\def\IF{\mathbb{F}}
\def\II{\mathbb{I}}
\def\IJ{\mathbb{J}}
\def\IK{\mathbb{K}}
\def\IM{\mathbb{M}}
\def\IN{\mathbb{N}}
\def\IP{\mathbb{P}}
\def\IR{\mathbb{R}}
\newcommand{\IRplus}{\mathbb{R}_{\ge 0}}
\def\IZ{\mathbb{Z}}
\def\11{\mathds{1}}


% Bold lowercase
\def\ba{\bm{a}}
\def\bb{\bm{b}}
\def\bc{\bm{c}}
\def\bd{\bm{d}}
\def\be{\bm{e}}
\def\bf{\bm{f}}
\def\bg{\bm{g}}
\def\bh{\bm{h}}
\def\bi{\bm{i}}
\def\bj{\bm{j}}
\def\bk{\bm{k}}
\def\bn{\bm{n}}
\def\bp{\bm{p}}
\def\br{\bm{r}}
\def\bs{\bm{s}}
\def\bu{\bm{u}}
\def\bv{\bm{v}}
\def\bw{\bm{w}}
\def\bx{\bm{x}}
\def\by{\bm{y}}
\def\bz{\bm{z}}
\newcommand{\vect}[1]{\bm{#1}}

% Bold capitals
\def\bB{\bm{B}}
\def\bD{\bm{D}}
\def\bE{\bm{E}}
\def\bF{\bm{F}}
\def\bG{\bm{G}}
\def\bI{\bm{I}}
\def\bL{\bm{L}}
\def\bN{\bm{N}}
\def\bP{\bm{P}}
\def\bR{\bm{R}}
\def\bS{\bm{S}}
\def\bT{\bm{T}}
\def\bX{\bm{X}}

% Bold numbers
\def\b0{\bm{0}}

% Bold greek
\bmdefine{\bmu}{\bm{\mu}}
\def\bphi{\bm{\phi}}
\def\bvarphi{\bm{\varphi}}
\def\bPi{\bm{\Pi}}
\def\bGamma{\bm{\Gamma}}

% Bold red sentence
\def\boldred#1{{\color{red}\textbf{#1}}}
\def\defword#1{{\color{orange}\textbf{#1}}}

% Caligraphic letters
\def\A{\mathcal{A}}
\def\B{\mathcal{B}}
\def\C{\mathcal{C}}
\def\D{\mathcal{D}}
\def\E{\mathcal{E}}
\def\F{\mathcal{F}}
\def\G{\mathcal{G}}
\def\H{\mathcal{H}}
\def\I{\mathcal{I}}
\def\L{\mathcal{L}}
\def\M{\mathcal{M}}
\def\N{\mathcal{N}}
\def\P{\mathcal{P}}
\def\R{\mathcal{R}}
\def\S{\mathcal{S}}
\def\T{\mathcal{T}}
\def\U{\mathcal{U}}
\def\V{\mathcal{V}}

% Adding space for prime (') where needed
\def\pprime{\,'}
% Adding space for star (\star) where needed
\def\pstar{{\,\star}}

% tt font for code
\def\code#1{{\tt #1}}

% i.e., e.g.
\def\eg{\emph{e.g.}}
\def\ie{\emph{i.e.}}


% Operators and special symbols
\def\nbOne{{\mathchoice {\rm 1\mskip-4mu l} {\rm 1\mskip-4mu l}
{\rm 1\mskip-4.5mu l} {\rm 1\mskip-5mu l}}}
\def\cov{\ensuremath{\mathsf{cov}}}
\def\Var{\ensuremath{\mathsf{Var}\ }}
\def\Im{\textrm{Im}\;}
\def\Re{\textrm{Re}\;}
\def\det{\ensuremath{\mathsf{det}}}
\def\diag{\ensuremath{\mathsf{diag}}}
\def\nullspace{\ensuremath{\mathsf{null}}}
\def\nullity{\ensuremath{\mathsf{nullity}}}
\def\rank{\ensuremath{\mathsf{rank}}}
\def\range{\ensuremath{\mathsf{range}}}
\def\sgn{\ensuremath{\mathsf{sgn}}}
\def\Span{\ensuremath{\mathsf{span}}}
\def\tr{\ensuremath{\mathsf{tr}}}
\def\imply{$\Rightarrow$}
\def\restrictTo#1#2{\left.#1\right|_{#2}}
\newcommand{\parallelsum}{\mathbin{\!/\mkern-5mu/\!}}
\def\dsum{\mathop{\displaystyle \sum }}%
\def\dind#1#2{_{\substack{#1\\ #2}}}

\newcommand{\Qmatrix}[1]{%
  \begin{pmatrix}#1\end{pmatrix}%
}

\DeclareMathOperator{\GL}{GL}
\DeclareMathOperator{\Rel}{Re}
\def\Nt#1{\left|\!\left|\!\left|#1\right|\!\right|\!\right|}
\newcommand{\tripbar}{|\! |\! |}



% The beamer bullet (in base colour)
\def\bbullet{\leavevmode\usebeamertemplate{itemize item}\ }

% Theorems and the like
\newtheorem{proposition}[theorem]{Proposition}
\newtheorem{property}[theorem]{Property}
\newtheorem{importantproperty}[theorem]{Property}
\newtheorem{importanttheorem}[theorem]{Theorem}
%\newtheorem{lemma}[theorem]{Lemma}
%\newtheorem{corollary}[theorem]{Corollary}
\newtheorem{remark}[theorem]{Remark}
\setbeamertemplate{theorems}[numbered]
%\setbeamertemplate{theorems}[ams style]

%
%\usecolortheme{orchid}
%\usecolortheme{orchid}

\def\red{\color[rgb]{1,0,0}}
\def\blue{\color[rgb]{0,0,1}}
\def\green{\color[rgb]{0,1,0}}

% Fix skipping lines after items in the bibliography
\setbeamertemplate{bibliography entry title}{}
\setbeamertemplate{bibliography entry location}{}
\setbeamertemplate{bibliography entry note}{}

% Get rid of navigation stuff
\setbeamertemplate{navigation symbols}{}

% Set footline/header line
\setbeamertemplate{footline}
{%
\quad p. \insertpagenumber \quad--\quad \insertsection\vskip2pt
}
% \setbeamertemplate{headline}
% {%
% \quad\insertsection\hfill p. \insertpagenumber\quad\mbox{}\vskip2pt
% }


\makeatletter
\newlength\beamerleftmargin
\setlength\beamerleftmargin{\Gm@lmargin}
\makeatother

% Colours for special pages
\def\extraContent{yellow!20}


%%%%%%%%%%%%%%%%%
\usepackage{tikz}
\usetikzlibrary{shapes,arrows}
\usetikzlibrary{positioning}
\usetikzlibrary{shapes.symbols,shapes.callouts,patterns}
\usetikzlibrary{calc,fit}
\usetikzlibrary{backgrounds}
\usetikzlibrary{decorations.pathmorphing,fit,petri}
\usetikzlibrary{automata}
\usetikzlibrary{fadings}
\usetikzlibrary{patterns,hobby}
\usetikzlibrary{backgrounds,fit,petri}
\usetikzlibrary{tikzmark}

\usepackage{pgfplots}
\pgfplotsset{compat=1.6}
\pgfplotsset{ticks=none}

\usetikzlibrary{decorations.markings}
\usetikzlibrary{arrows.meta}
\tikzset{>=stealth}

% For tikz
\tikzstyle{cloud} = [draw, ellipse,fill=red!20, node distance=0.87cm,
minimum height=2em]
\tikzstyle{line} = [draw, -latex']


%%% For max frame images
\newenvironment{changemargin}[2]{%
\begin{list}{}{%
\setlength{\topsep}{0pt}%
\setlength{\leftmargin}{#1}%
\setlength{\rightmargin}{#2}%
\setlength{\listparindent}{\parindent}%
\setlength{\itemindent}{\parindent}%
\setlength{\parsep}{\parskip}%
}%
\item[]}{\end{list}}


% Make one image take up the entire slide content area in beamer,.:
% centered/centred full-screen image, with title:
% This uses the whole screen except for the 1cm border around it
% all. 128x96mm
\newcommand{\titledFrameImage}[2]{
\begin{frame}{#1}
%\begin{changemargin}{-1cm}{-1cm}
\begin{center}
\includegraphics[width=108mm,height=\textheight,keepaspectratio]{#2}
\end{center}
%\end{changemargin}
\end{frame}
}

% Make one image take up the entire slide content area in beamer.:
% centered/centred full-screen image, no title:
% This uses the whole screen except for the 1cm border around it
% all. 128x96mm
\newcommand{\plainFrameImage}[1]{
\begin{frame}[plain]
%\begin{changemargin}{-1cm}{-1cm}
\begin{center}
\includegraphics[width=108mm,height=76mm,keepaspectratio]{#1}
\end{center}
%\end{changemargin}
\end{frame}
}

% Make one image take up the entire slide area, including borders, in beamer.:
% centered/centred full-screen image, no title:
% This uses the entire whole screen
\newcommand{\maxFrameImage}[1]{
\begin{frame}[plain]
\begin{changemargin}{-1cm}{-1cm}
\begin{center}
\includegraphics[width=\paperwidth,height=\paperheight,keepaspectratio]
{#1}
\end{center}
\end{changemargin}
\end{frame}
}

% This uses the entire whole screen (to include in frame)
\newcommand{\maxFrameImageNoFrame}[1]{
\begin{changemargin}{-1cm}{-1cm}
\begin{center}
\includegraphics[width=\paperwidth,height=0.99\paperheight,keepaspectratio]
{#1}
\end{center}
\end{changemargin}
}

% Make one image take up the entire slide area, including borders, in beamer.:
% centered/centred full-screen image, no title:
% This uses the entire whole screen
\newcommand{\maxFrameImageColor}[2]{
\begin{frame}[plain]
\setbeamercolor{normal text}{bg=#2!20}
\begin{changemargin}{-1cm}{-1cm}
\begin{center}
\includegraphics[width=\paperwidth,height=\paperheight,keepaspectratio]
{#1}
\end{center}
\end{changemargin}
\end{frame}
}


\usepackage{tikz}
\usetikzlibrary{patterns,hobby,matrix}
\usepackage{pgfplots}
\pgfplotsset{compat=1.6}
\pgfplotsset{ticks=none}

\usetikzlibrary{backgrounds}
\usetikzlibrary{decorations.markings}
\usetikzlibrary{arrows.meta}
\tikzset{>=stealth}

\tikzset{
  clockwise arrows/.style={
    postaction={
      decorate,
      decoration={
        markings,
        mark=between positions 0.1 and 0.9 step 40pt with {\arrow{>}},
   }}}}


% New integrated section command: creates section and section slide
\newcommand{\Ssection}[2]{
\section{#1}
\begin{frame}[noframenumbering,plain]
  \begin{tikzpicture}[remember picture,overlay]
    \node[above right,inner sep=0pt,opacity=0.2] at (current page.south west)
    {
        \includegraphics[height=\paperheight,width=\paperwidth]{#2}
    };
  \end{tikzpicture}
  \setbeamercolor{section in toc}{fg=section_page_list_colour}
  \setbeamerfont{section in toc}{size=\Large,series=\bfseries}
  \setbeamertemplate{section in toc shaded}[default][60]
  \tableofcontents[
    currentsection,
    sectionstyle=show/shaded,
    subsectionstyle=show/hide/hide,
    subsubsectionstyle=hide/hide/hide]
\end{frame}
\addtocounter{page}{-1}
}

% New integrated section command with subsections: creates section and section slide showing subsections
\newcommand{\SsectionWithSubs}[2]{
\section{#1}
\begin{frame}[noframenumbering,plain]
  \begin{tikzpicture}[remember picture,overlay]
    \node[above right,inner sep=0pt,opacity=0.2] at (current page.south west)
    {
        \includegraphics[height=\paperheight,width=\paperwidth]{#2}
    };
  \end{tikzpicture}
  \setbeamercolor{section in toc}{fg=section_page_list_colour}
  \setbeamerfont{section in toc}{size=\Large,series=\bfseries}
  \setbeamertemplate{section in toc shaded}[default][60]
  \tableofcontents[
    currentsection,
    sectionstyle=show/hide,
    subsectionstyle=show/show/hide,
    subsubsectionstyle=hide/hide/hide]
\end{frame}
\addtocounter{page}{-1}
}

% New integrated subsection command: creates subsection and subsection slide
\newcommand{\Ssubsection}[2]{
\subsection{#1}
\begin{frame}[noframenumbering,plain]
  \begin{tikzpicture}[remember picture,overlay]
    \node[above right,inner sep=0pt,opacity=0.2] at (current page.south west)
    {
        \includegraphics[height=\paperheight,width=\paperwidth]{#2}
    };
  \end{tikzpicture}
  \setbeamercolor{section in toc}{fg=subsection_page_list_colour}
  \setbeamerfont{section in toc}{size=\Large,series=\bfseries}
  \setbeamertemplate{section in toc shaded}[default][60]
  \setbeamerfont{subsection in toc}{series=\bfseries}
  \setbeamertemplate{subsection in toc shaded}[default][50]
  \tableofcontents[
    currentsection,
    sectionstyle=show/hide,
    subsectionstyle=show/shaded/hide,
    subsubsectionstyle=hide/hide/hide]
\end{frame}
\addtocounter{page}{-1}
}

% New integrated subsubsection command: creates subsubsection and subsubsection slide
\newcommand{\Ssubsubsection}[2]{
\subsubsection{#1}
\begin{frame}[noframenumbering,plain]
  \begin{tikzpicture}[remember picture,overlay]
    \node[above right,inner sep=0pt,opacity=0.2] at (current page.south west)
    {
        \includegraphics[height=\paperheight,width=\paperwidth]{#2}
    };
  \end{tikzpicture}
  \setbeamercolor{section in toc}{fg=subsub_header_section}
  \setbeamerfont{section in toc}{size=\Large,series=\bfseries}
  \setbeamertemplate{section in toc shaded}[default][60]
  \setbeamerfont{subsection in toc}{series=\bfseries}
  \setbeamertemplate{subsection in toc shaded}[default][50]
  \setbeamertemplate{subsubsection in toc shaded}[default][50]
  \tableofcontents[
    currentsection,
    sectionstyle=show/hide,
    subsectionstyle=show/hide/hide,
    subsubsectionstyle=show/shaded/hide]
\end{frame}
\addtocounter{page}{-1}
}

% Legacy commands (kept for backward compatibility)
% Beginning of a section
\newcommand{\newSectionSlide}[1]{
\begin{frame}[noframenumbering,plain]
  \begin{tikzpicture}[remember picture,overlay]
    \node[above right,inner sep=0pt,opacity=0.2] at (current page.south west)
    {
        \includegraphics[height=\paperheight,width=\paperwidth]{#1}
    };
  \end{tikzpicture}
  \setbeamercolor{section in toc}{fg=section_page_list_colour}
  \setbeamerfont{section in toc}{size=\Large,series=\bfseries}
  \setbeamertemplate{section in toc shaded}[default][60]
  \tableofcontents[
    currentsection,
    sectionstyle=show/shaded,
    subsectionstyle=show/hide/hide,
    subsubsectionstyle=hide/hide/hide]
\end{frame}
\addtocounter{page}{-1}
}

% Beginning of a section in which we also show subsections
\newcommand{\newSectionWithSubsSlide}[1]{
	\begin{frame}[noframenumbering,plain]
		\begin{tikzpicture}[remember picture,overlay]
			\node[above right,inner sep=0pt,opacity=0.2] at (current page.south west)
			{
				\includegraphics[height=\paperheight,width=\paperwidth]{#1}
			};
		\end{tikzpicture}
		\setbeamercolor{section in toc}{fg=section_page_list_colour}
		\setbeamerfont{section in toc}{size=\Large,series=\bfseries}
		\setbeamertemplate{section in toc shaded}[default][60]
		\tableofcontents[
		currentsection,
		sectionstyle=show/hide,
		subsectionstyle=show/show/hide,
		subsubsectionstyle=hide/hide/hide]
	\end{frame}
	\addtocounter{page}{-1}
}

% Beginning of a subsection
\newcommand{\newSubSectionSlide}[1]{
\begin{frame}[noframenumbering,plain]
  \begin{tikzpicture}[remember picture,overlay]
    \node[above right,inner sep=0pt,opacity=0.2] at (current page.south west)
    {
        \includegraphics[height=\paperheight,width=\paperwidth]{#1}
    };
  \end{tikzpicture}
  \setbeamercolor{section in toc}{fg=subsection_page_list_colour}
  \setbeamerfont{section in toc}{size=\Large,series=\bfseries}
  \setbeamertemplate{section in toc shaded}[default][60]
  \setbeamerfont{subsection in toc}{series=\bfseries}
  \setbeamertemplate{subsection in toc shaded}[default][50]
  \tableofcontents[
    currentsection,
    sectionstyle=show/hide,
    subsectionstyle=show/shaded/hide,
    subsubsectionstyle=hide/hide/hide]
\end{frame}
\addtocounter{page}{-1}
}

% Beginning of a subsubsection
\newcommand{\newSubSubSectionSlide}[1]{
\begin{frame}[noframenumbering,plain]
  \begin{tikzpicture}[remember picture,overlay]
    \node[above right,inner sep=0pt,opacity=0.2] at (current page.south west)
    {
        \includegraphics[height=\paperheight,width=\paperwidth]{#1}
    };
  \end{tikzpicture}
  \setbeamercolor{section in toc}{fg=subsub_header_section}
  \setbeamerfont{section in toc}{size=\Large,series=\bfseries}
  \setbeamertemplate{section in toc shaded}[default][60]
  \setbeamerfont{subsection in toc}{series=\bfseries}
  \setbeamertemplate{subsection in toc shaded}[default][50]
  \setbeamertemplate{subsubsection in toc shaded}[default][50]
  \tableofcontents[
    currentsection,
    sectionstyle=show/hide,
    subsectionstyle=show/hide/hide,
    subsubsectionstyle=show/shaded/hide]
\end{frame}
\addtocounter{page}{-1}
}


   %%%%%%%%%%%
% To have links to parts in the outline
\makeatletter
\AtBeginPart{%
  \addtocontents{toc}{\protect\beamer@partintoc{\the\c@part}{\beamer@partnameshort}{\the\c@page}}%
}
%% number, shortname, page.
\providecommand\beamer@partintoc[3]{%
  \ifnum\c@tocdepth=-1\relax
    % requesting onlyparts.
    \makebox[6em]{Part #1:} \textcolor{green!30!blue}{\hyperlink{#2}{#2}}
    \par
  \fi
}
\define@key{beamertoc}{onlyparts}[]{%
  \c@tocdepth=-1\relax
}
\makeatother%

\newcommand{\nameofthepart}{}
\newcommand{\nupart}[1]%
    {   \part{#1}%
        \renewcommand{\nameofthepart}{#1}%
        {
          \setbeamercolor{background canvas}{bg=orange!50}
          \begin{frame}{#1}%\partpage 
          \hypertarget{\nameofthepart}{}\tableofcontents%
          \end{frame}
        }
    }

% This command creates a title page using TikZ only
\newcommand{\tikztitlepage}[1]{%
\begin{frame}[plain,noframenumbering]
  \begin{tikzpicture}[remember picture,overlay]
    % Background image
    \node[above right,inner sep=0pt,opacity=0.1] 
      at (current page.south west) 
      {\includegraphics[width=\paperwidth,height=\paperheight]{#1}};

    % University logo
    \node[anchor=north east, inner sep=5pt, opacity=0.9] 
      at (current page.north east)
      {\includegraphics[width=0.2\textwidth]{FIGS-slides-admin/UM-logo-horizontal-CMYK.png}};
    
    % Title
    \node[anchor=center, align=center, 
          font=\fontsize{13}{15}\bfseries\color{UMbrown}, 
          text width=0.9\textwidth] 
          at ([yshift=2cm]current page.center)
          {\inserttitle};

      % Authors
      \node[anchor=center, align=center,
        font=\fontsize{10}{12}\bfseries\color{UMbrown},
        text width=0.7\textwidth]
        at ([yshift=0.8cm]current page.center)
        {\insertauthor};

      % Affiliation
      \node[anchor=north, align=center,
        font=\fontsize{9}{11}\color{UMbrown},
        text width=0.7\textwidth]
        at ([yshift=-0.2cm]current page.center)
        {\insertaffiliation};      
    % Date
    \node[anchor=north, align=center, 
          font=\fontsize{12}{16}\bfseries\color{UMbrown},
          text width=0.7\textwidth] 
          at ([yshift=0.2cm]current page.center)
          {\insertdate};

    % Land acknowledgement
    \node[anchor=south, align=justify, 
          font=\footnotesize, text=black, 
          text width=1.1\textwidth] 
          at ([yshift=0.5cm]current page.south)
          {The University of Manitoba campuses are located on original lands of Anishinaabeg, Ininew, Anisininew, Dakota and Dene peoples, and on the National Homeland of the Red River Métis.\\
          We respect the Treaties that were made on these territories, we acknowledge the harms and mistakes of the past, and we dedicate ourselves to move forward in partnership with Indigenous communities in a spirit of Reconciliation and collaboration.};
  \end{tikzpicture}
  \addtocounter{page}{-1}
\end{frame}
}
% The title page with figure
% \newcommand{\titlepagewithfigure}[1]{%
%   \begin{frame}[noframenumbering,plain]
%     \begin{tikzpicture}[remember picture,overlay]
%       \node[above right,inner sep=0pt,opacity=0.1] at (current page.south west)
%       {
%           \includegraphics[height=\paperheight,width=\paperwidth]{#1}
%       };
%       \node[anchor=north east,
%       inner sep=5pt,
%       opacity=0.9] at (current page.north east)
%       {
%           \includegraphics[width=0.2\textwidth]{FIGS-slides-admin/UM-logo-horizontal-CMYK.png}
%       };
%       \node[anchor=south, 
%       align=justify, 
%       text=black, 
%       text width=1.1\textwidth,
%       font=\footnotesize]  (land_acknowledgement)
%       at (current page.south) 
%       {The University of Manitoba campuses are located on original lands of Anishinaabeg, Ininew, Anisininew, Dakota and Dene peoples, and on the National Homeland of the Red River Métis.
%       We respect the Treaties that were made on these territories, we acknowledge the harms and mistakes of the past, and we dedicate ourselves to move forward in partnership with Indigenous communities in a spirit of Reconciliation and collaboration.};  
%       % \node[align=center, anchor=south,
%       % above=0.5cm of land_acknowledgement,
%       % text=black,
%       % font=\bfseries] {\insertdate};
%   \end{tikzpicture}
%   \setbeamercolor{title}{fg=title_page_title_colour}
%   \setbeamerfont{title}{size=\Large,series=\bfseries}
%   \setbeamercolor{author}{fg=title_page_author_colour}
%   \setbeamerfont{author}{size=\large,series=\bfseries}
%   \setbeamercolor{institute}{fg=title_page_institute_colour}
%   \setbeamerfont{institute}{size=\large,series=\bfseries}
%   \setbeamercolor{date}{fg=title_page_date_colour}
%   \setbeamerfont{date}{series=\bfseries}
% 	\titlepage
% \end{frame}
% \addtocounter{page}{-1}
% }

\newcommand{\titlepagewithfigure}[1]{%
  \begin{frame}[noframenumbering,plain]
    \begin{tikzpicture}[remember picture,overlay]
      \node[above right,inner sep=0pt,opacity=0.1] at (current page.south west)
      {
          \includegraphics[height=\paperheight,width=\paperwidth]{#1}
      };
      \node[anchor=north east,
      inner sep=5pt,
      opacity=0.9] at (current page.north east)
      {
          \includegraphics[width=0.2\textwidth]{FIGS-slides-admin/UM-logo-horizontal-CMYK.png}
      };
      \node[anchor=south, 
      align=justify, 
      text=black, 
      text width=1.1\textwidth,
      font=\footnotesize]  (land_acknowledgement)
      at (current page.south) 
      {The University of Manitoba campuses are located on original lands of Anishinaabeg, Ininew, Anisininew, Dakota and Dene peoples, and on the National Homeland of the Red River Métis.
      We respect the Treaties that were made on these territories, we acknowledge the harms and mistakes of the past, and we dedicate ourselves to move forward in partnership with Indigenous communities in a spirit of Reconciliation and collaboration.};  
      % \node[align=center, anchor=south,
      % above=0.5cm of land_acknowledgement,
      % text=black,
      % font=\bfseries] {\insertdate};
  \end{tikzpicture}
  \setbeamercolor{title}{fg=title_page_title_colour}
  \setbeamerfont{title}{size=\Large,series=\bfseries,family=\usefont{T1}{phv}{b}{n}}
  \setbeamercolor{author}{fg=title_page_author_colour}
  \setbeamerfont{author}{size=\large,series=\bfseries,family=\usefont{T1}{phv}{b}{n}}
  \setbeamercolor{institute}{fg=title_page_institute_colour}
  \setbeamerfont{institute}{size=\large,series=\bfseries,family=\usefont{T1}{phv}{b}{n}}
  \setbeamercolor{date}{fg=title_page_date_colour}
  \setbeamerfont{date}{series=\bfseries,family=\usefont{T1}{phv}{b}{n}}
	\titlepage
\end{frame}
\addtocounter{page}{-1}
}
% The outline page, with figure
% \newcommand{\outlinepage}[1]{%
% \begin{frame}[noframenumbering,plain]
%   \begin{tikzpicture}[remember picture,overlay]
%     \node[above right,inner sep=0pt,opacity=0.2] at (current page.south west)
%     {
%         \includegraphics[height=\paperheight,width=\paperwidth]{#1}
%     };
%   \end{tikzpicture}
%   \setbeamercolor{section in toc}{fg=outline_page_list_colour}
%   \setbeamerfont{section in toc}{size=\Large,series=\bfseries,family=\sffamily}
%   \frametitle{\textcolor{outline_page_title_colour}{\LARGE\bfseries Outline}}
%   \tableofcontents[hideallsubsections]
% \end{frame}
% \addtocounter{page}{-1}
% }
% The outline page, with figure
\newcommand{\outlinepage}[1]{%
\begin{frame}[noframenumbering,plain]
  \begin{tikzpicture}[remember picture,overlay]
    \node[above right,inner sep=0pt,opacity=0.2] at (current page.south west)
    {
        \includegraphics[height=\paperheight,width=\paperwidth]{#1}
    };
  \end{tikzpicture}
  \setbeamercolor{section in toc}{fg=outline_page_list_colour}
  % Use Helvetica Bold only for the outline slide TOC
  \setbeamerfont{section in toc}{size=\Large,family=\usefont{T1}{phv}{b}{n}}
  % Use Helvetica Bold for the outline title
  \frametitle{\textcolor{outline_page_title_colour}{\usefont{T1}{phv}{b}{n}\LARGE Outline}}
  \tableofcontents[hideallsubsections]
\end{frame}
\addtocounter{page}{-1}
}


%\let\oldsection\section
%\renewcommand{\section}[2]{\oldsection[#1]\newSectionSlide[#2]}


%%%%%%%%%%%%%%%%%%%%%
% CUSTOM SLIDE BACKGROUNDS
%%%%%%%%%%%%%%%%%%%%%
% Define custom background templates for different colors
\defbeamertemplate*{background canvas}{blue}{%
  \color{blue!15}\vrule width\paperwidth height\paperheight%
}
\defbeamertemplate*{background canvas}{green}{%
  \color{green!15}\vrule width\paperwidth height\paperheight%
}
\defbeamertemplate*{background canvas}{red}{%
  \color{red!15}\vrule width\paperwidth height\paperheight%
}
\defbeamertemplate*{background canvas}{yellow}{%
  \color{yellow!20}\vrule width\paperwidth height\paperheight%
}
\defbeamertemplate*{background canvas}{purple}{%
  \color{purple!15}\vrule width\paperwidth height\paperheight%
}
\defbeamertemplate*{background canvas}{orange}{%
  \color{orange!20}\vrule width\paperwidth height\paperheight%
}

% Define keys for the different background options
\makeatletter
\define@key{beamerframe}{blue}[true]{\setbeamertemplate{background canvas}[blue]}
\define@key{beamerframe}{green}[true]{\setbeamertemplate{background canvas}[green]}
\define@key{beamerframe}{red}[true]{\setbeamertemplate{background canvas}[red]}
\define@key{beamerframe}{yellow}[true]{\setbeamertemplate{background canvas}[yellow]}
\define@key{beamerframe}{purple}[true]{\setbeamertemplate{background canvas}[purple]}
\define@key{beamerframe}{orange}[true]{\setbeamertemplate{background canvas}[orange]}
\makeatother

% Reset to normal background for all frames by default
\BeforeBeginEnvironment{frame}{\setbeamertemplate{background canvas}[default]}

\newcommand{\punnett}[2]{
    \begin{center}
    \renewcommand{\arraystretch}{1.5} % Add space to rows
    \begin{tabular}{|c c | c|c|}
        \multicolumn{2}{c}{} & \multicolumn{2}{c}{\textbf{Father}} \\
        \multicolumn{2}{c}{} & #1 \\ \hline
        #2
    \end{tabular}
    \end{center}
}

% Colour definitions for Punnett squares
\usepackage{multirow}
\usepackage{colortbl}
\colorlet{punnettorange}{orange!30}
\colorlet{punnettblack}{black!50}
\colorlet{punnetttortie}{orange!60!black}

\usetikzlibrary{positioning, decorations.pathreplacing, arrows.meta}

% --- TIKZ STYLES FOR CAT DIAGRAMS ---
\tikzset{
    % This is the main placeholder style for the cat images
    catnode/.style={
        draw, 
        rectangle, 
        rounded corners, 
        minimum height=1.5cm, 
        minimum width=2.5cm, 
        align=center, 
        font=\small\bfseries
    },
    % Allele styles (the X and Y at the top/side)
    allele/.style={font=\Large\bfseries},
    female/.style={allele, color=purple!80!black},
    male/.style={allele, color=blue!80!black},
    % --- Placeholder styles for different cats ---
    % YOU CAN REPLACE THE CONTENTS OF THESE NODES WITH YOUR IMAGES
    orange_cat/.style={
        catnode, 
        fill=orange!30, 
        text=black
    },
    black_cat/.style={
        catnode, 
        fill=black!70, 
        text=white
    },
    tortie_cat/.style={
        catnode, 
        fill=orange!50!black, % A mix for tortoiseshell
        text=white
    }
}

\usetikzlibrary{
    arrows.meta, % For nicer arrow heads (e.g., -Latex)
    positioning, % For relative node placement (e.g., above=of)
    automata     % For state diagrams, loops
}
% --- END TIKZ STYLES ---


\usecolortheme{orchid}
%% Listings
\usepackage{listings}
\definecolor{mygreen}{rgb}{0,0.6,0}
\definecolor{mygray}{rgb}{0.5,0.5,0.5}
\definecolor{mymauve}{rgb}{0.58,0,0.82}
\definecolor{mygold}{rgb}{1,0.843,0}
\definecolor{myblue}{rgb}{0.537,0.812,0.941}

\definecolor{mygold2}{RGB}{120,105,22}
\definecolor{mygrey2}{RGB}{50,50,50}

\definecolor{lgreen}{rgb}{0.6,0.9,.6}
\definecolor{lred}{rgb}{1,0.5,.5}

\lstloadlanguages{R}
\lstset{ %
  language=R,
  backgroundcolor=\color{black!05},   % choose the background color
  basicstyle=\footnotesize\ttfamily,        % size of fonts used for the code
  breaklines=true,                 % automatic line breaking only at whitespace
  captionpos=b,                    % sets the caption-position to bottom
  commentstyle=\color{mygreen},    % comment style
  escapeinside={\%*}{*)},          % if you want to add LaTeX within your code
  keywordstyle=\color{red},       % keyword style
  stringstyle=\color{mygold},     % string literal style
  keepspaces=true,
  columns=fullflexible,
  tabsize=4,
}
% Could also do (in lstset)
% basicstyle==\fontfamily{pcr}\footnotesize
\lstdefinelanguage{Renhanced}%
  {keywords={abbreviate,abline,abs,acos,acosh,action,add1,add,%
      aggregate,alias,Alias,alist,all,anova,any,aov,aperm,append,apply,%
      approx,approxfun,apropos,Arg,args,array,arrows,as,asin,asinh,%
      atan,atan2,atanh,attach,attr,attributes,autoload,autoloader,ave,%
      axis,backsolve,barplot,basename,besselI,besselJ,besselK,besselY,%
      beta,binomial,body,box,boxplot,break,browser,bug,builtins,bxp,by,%
      c,C,call,Call,case,cat,category,cbind,ceiling,character,char,%
      charmatch,check,chol,chol2inv,choose,chull,class,close,cm,codes,%
      coef,coefficients,co,col,colnames,colors,colours,commandArgs,%
      comment,complete,complex,conflicts,Conj,contents,contour,%
      contrasts,contr,control,helmert,contrib,convolve,cooks,coords,%
      distance,coplot,cor,cos,cosh,count,fields,cov,covratio,wt,CRAN,%
      create,crossprod,cummax,cummin,cumprod,cumsum,curve,cut,cycle,D,%
      data,dataentry,date,dbeta,dbinom,dcauchy,dchisq,de,debug,%
      debugger,Defunct,default,delay,delete,deltat,demo,de,density,%
      deparse,dependencies,Deprecated,deriv,description,detach,%
      dev2bitmap,dev,cur,deviance,off,prev,,dexp,df,dfbetas,dffits,%
      dgamma,dgeom,dget,dhyper,diag,diff,digamma,dim,dimnames,dir,%
      dirname,dlnorm,dlogis,dnbinom,dnchisq,dnorm,do,dotplot,double,%
      download,dpois,dput,drop,drop1,dsignrank,dt,dummy,dump,dunif,%
      duplicated,dweibull,dwilcox,dyn,edit,eff,effects,eigen,else,%
      emacs,end,environment,env,erase,eval,equal,evalq,example,exists,%
      exit,exp,expand,expression,External,extract,extractAIC,factor,%
      fail,family,fft,file,filled,find,fitted,fivenum,fix,floor,for,%
      For,formals,format,formatC,formula,Fortran,forwardsolve,frame,%
      frequency,ftable,ftable2table,function,gamma,Gamma,gammaCody,%
      gaussian,gc,gcinfo,gctorture,get,getenv,geterrmessage,getOption,%
      getwd,gl,glm,globalenv,gnome,GNOME,graphics,gray,grep,grey,grid,%
      gsub,hasTsp,hat,heat,help,hist,home,hsv,httpclient,I,identify,if,%
      ifelse,Im,image,\%in\%,index,influence,measures,inherits,install,%
      installed,integer,interaction,interactive,Internal,intersect,%
      inverse,invisible,IQR,is,jitter,kappa,kronecker,labels,lapply,%
      layout,lbeta,lchoose,lcm,legend,length,levels,lgamma,library,%
      licence,license,lines,list,lm,load,local,locator,log,log10,log1p,%
      log2,logical,loglin,lower,lowess,ls,lsfit,lsf,ls,machine,Machine,%
      mad,mahalanobis,make,link,margin,match,Math,matlines,mat,matplot,%
      matpoints,matrix,max,mean,median,memory,menu,merge,methods,min,%
      missing,Mod,mode,model,response,mosaicplot,mtext,mvfft,na,nan,%
      names,omit,nargs,nchar,ncol,NCOL,new,next,NextMethod,nextn,%
      nlevels,nlm,noquote,NotYetImplemented,NotYetUsed,nrow,NROW,null,%
      numeric,\%o\%,objects,offset,old,on,Ops,optim,optimise,optimize,%
      options,or,order,ordered,outer,package,packages,page,pairlist,%
      pairs,palette,panel,par,parent,parse,paste,path,pbeta,pbinom,%
      pcauchy,pchisq,pentagamma,persp,pexp,pf,pgamma,pgeom,phyper,pico,%
      pictex,piechart,Platform,plnorm,plogis,plot,pmatch,pmax,pmin,%
      pnbinom,pnchisq,pnorm,points,poisson,poly,polygon,polyroot,pos,%
      postscript,power,ppoints,ppois,predict,preplot,pretty,Primitive,%
      print,prmatrix,proc,prod,profile,proj,prompt,prop,provide,%
      psignrank,ps,pt,ptukey,punif,pweibull,pwilcox,q,qbeta,qbinom,%
      qcauchy,qchisq,qexp,qf,qgamma,qgeom,qhyper,qlnorm,qlogis,qnbinom,%
      qnchisq,qnorm,qpois,qqline,qqnorm,qqplot,qr,Q,qty,qy,qsignrank,%
      qt,qtukey,quantile,quasi,quit,qunif,quote,qweibull,qwilcox,%
      rainbow,range,rank,rbeta,rbind,rbinom,rcauchy,rchisq,Re,read,csv,%
      csv2,fwf,readline,socket,real,Recall,rect,reformulate,regexpr,%
      relevel,remove,rep,repeat,replace,replications,report,require,%
      resid,residuals,restart,return,rev,rexp,rf,rgamma,rgb,rgeom,R,%
      rhyper,rle,rlnorm,rlogis,rm,rnbinom,RNGkind,rnorm,round,row,%
      rownames,rowsum,rpois,rsignrank,rstandard,rstudent,rt,rug,runif,%
      rweibull,rwilcox,sample,sapply,save,scale,scan,scan,screen,sd,se,%
      search,searchpaths,segments,seq,sequence,setdiff,setequal,set,%
      setwd,show,sign,signif,sin,single,sinh,sink,solve,sort,source,%
      spline,splinefun,split,sqrt,stars,start,stat,stem,step,stop,%
      storage,strstrheight,stripplot,strsplit,structure,strwidth,sub,%
      subset,substitute,substr,substring,sum,summary,sunflowerplot,svd,%
      sweep,switch,symbol,symbols,symnum,sys,status,system,t,table,%
      tabulate,tan,tanh,tapply,tempfile,terms,terrain,tetragamma,text,%
      time,title,topo,trace,traceback,transform,tri,trigamma,trunc,try,%
      ts,tsp,typeof,unclass,undebug,undoc,union,unique,uniroot,unix,%
      unlink,unlist,unname,untrace,update,upper,url,UseMethod,var,%
      variable,vector,Version,vi,warning,warnings,weighted,weights,%
      which,while,window,write,\%x\%,x11,X11,xedit,xemacs,xinch,xor,%
      xpdrows,xy,xyinch,yinch,zapsmall,zip},%
   otherkeywords={!,!=,~,$,*,\%,\&,\%/\%,\%*\%,\%\%,<-,<<-,_,/},%
   alsoother={._$},%
   sensitive,%
   morecomment=[l]\#,%
   morestring=[d]",%
   morestring=[d]'% 2001 Robert Denham
  }%

%%%%%%% 
%% Definitions in yellow boxes
\usepackage{etoolbox}
\setbeamercolor{block title}{use=structure,fg=structure.fg,bg=structure.fg!40!bg}
\setbeamercolor{block body}{parent=normal text,use=block title,bg=block title.bg!20!bg}

\BeforeBeginEnvironment{definition}{%
	\setbeamercolor{block title}{fg=black,bg=yellow!20!white}
	\setbeamercolor{block body}{fg=black, bg=yellow!05!white}
}
\AfterEndEnvironment{definition}{
	\setbeamercolor{block title}{use=structure,fg=structure.fg,bg=structure.fg!20!bg}
	\setbeamercolor{block body}{parent=normal text,use=block title,bg=block title.bg!50!bg, fg=black}
}
\BeforeBeginEnvironment{importanttheorem}{%
	\setbeamercolor{block title}{fg=black,bg=red!20!white}
	\setbeamercolor{block body}{fg=black, bg=red!05!white}
}
\AfterEndEnvironment{importanttheorem}{
	\setbeamercolor{block title}{use=structure,fg=structure.fg,bg=structure.fg!20!bg}
	\setbeamercolor{block body}{parent=normal text,use=block title,bg=block title.bg!50!bg, fg=black}
}
\BeforeBeginEnvironment{importantproperty}{%
	\setbeamercolor{block title}{fg=black,bg=red!50!white}
	\setbeamercolor{block body}{fg=black, bg=red!30!white}
}
\AfterEndEnvironment{importantproperty}{
	\setbeamercolor{block title}{use=structure,fg=structure.fg,bg=structure.fg!20!bg}
	\setbeamercolor{block body}{parent=normal text,use=block title,bg=block title.bg!50!bg, fg=black}
}

% Colour for the outline page
\definecolor{outline_colour}{RGB}{230,165,83}
%% Colours for sections, subsections aand subsubsections
\definecolor{section_colour}{RGB}{27,46,28}
\definecolor{subsection_colour}{RGB}{52,128,56}
\definecolor{subsubsection_colour}{RGB}{150,224,154}
\definecolor{subsub_header_section}{RGB}{196,44,27}
%\definecolor{mygold}{rgb}{1,0.843,0}
% Beginning of a section
% \AtBeginSection[]{
% 	{
% 	  \setbeamercolor{section in toc}{fg=mygold}
% 		\setbeamercolor{background canvas}{bg=section_colour}
% 		\begin{frame}[noframenumbering,plain]
% 			\framesubtitle{\nameofthepart Chapter \insertromanpartnumber \ -- \iteminsert{\insertpart}}
% 			\tableofcontents[
% 				currentsection,
% 				sectionstyle=show/shaded,
% 				subsectionstyle=show/hide/hide,
% 				subsubsectionstyle=hide/hide/hide]
% 		\end{frame}
% 	\addtocounter{page}{-1}
% 	%\addtocounter{framenumber}{-1} 
% 	}
% }


% % Beginning of a section
% \AtBeginSubsection[]{
% 	{
% 	  \setbeamercolor{section in toc}{fg=mygold}
% 		\setbeamercolor{background canvas}{bg=subsection_colour}
% 		\begin{frame}[noframenumbering,plain]
% 				\framesubtitle{\nameofthepart Chapter \insertromanpartnumber \ -- \iteminsert{\insertpart}}
% 				\tableofcontents[
% 					currentsection,
% 					sectionstyle=show/hide,
% 					currentsubsection,
% 					subsectionstyle=show/shaded/hide,
% 					subsubsectionstyle=show/hide/hide]
% 			\end{frame}
% 		\addtocounter{page}{-1}
% 	}
% }

% \newcommand{\newSubSectionSlide}[1]{
% \begin{frame}[noframenumbering,plain]
%   \begin{tikzpicture}[remember picture,overlay]
%     \node[above right,inner sep=0pt,opacity=0.2] at (current page.south west)
%     {
%         \includegraphics[height=\paperheight,width=\paperwidth]{#1}
%     };
%   \end{tikzpicture}
%   \setbeamercolor{section in toc}{fg=subsub_header_section}
%   \setbeamerfont{section in toc}{size=\Large,series=\bfseries}
%   \setbeamertemplate{section in toc shaded}[default][60]
%   \setbeamertemplate{subsection in toc shaded}[default][60]
%   %\setbeamercolor{background canvas}{bg=section_colour}
%   \tableofcontents[
%     currentsection,
%     sectionstyle=show/hide,
%     currentsubsection,
%     subsectionstyle=show/shaded/hide,
%     subsubsectionstyle=show/hide/hide]
% \end{frame}
% \addtocounter{page}{-1}
% }


% % Beginning of a section
% \AtBeginSubsubsection[]{
% 	{
% 	  \setbeamercolor{section in toc}{fg=subsub_header_section}
% 	  \setbeamercolor{subsubsection in toc}{fg=mygold2}
% 	  \setbeamercolor{subsubsection in toc shaded}{fg=mygrey2}
% 		\setbeamercolor{background canvas}{bg=subsubsection_colour}
% 		\begin{frame}[noframenumbering,plain]
% 				\framesubtitle{\nameofthepart Chapter \insertromanpartnumber \ -- \iteminsert{\insertpart}}
% 				\tableofcontents[
% 					currentsection,
% 					sectionstyle=show/hide,
% 					currentsubsection,
% 					subsectionstyle=show/hide/shaded
% 					currentsubsubsection]%,
% 					%subsubsectionstyle=hide/hide/shaded]
% 					%currentsubsubsection]
% 			\end{frame}
% 		\addtocounter{page}{-1}
% 	}
% }


\title{Matrix methods}
\date{}

\usepackage{Sweave}
\begin{document}
\input{MATH2740-slides-06-matrix-methods-concordance}
\DeclareFontShape{OT1}{cmss}{b}{n}{<->ssub * cmss/bx/n}{} 
\begin{frame}
	\titlepage
\end{frame}


%%%%%%%%%%%%%%%%%%%%%
%%%%%%%%%%%%%%%%%%%%%
%%%%%%%%%%%%%%%%%%%%%
%%%%%%%%%%%%%%%%%%%%%
\section{Orthogonality and projections}

\begin{frame}
\begin{definition}[Orthogonal set of vectors]
The set of vectors $\{\bv_1,\ldots,\bv_k\}\in\IR^n$ is an \textbf{orthogonal set} if
\[
\forall i,j=1,\ldots,k,\quad i\neq j \implies \bv_i\bullet\bv_j=0
\]
\end{definition}

\begin{theorem}\label{th:ortho_implies_LI}
$\{\bv_1,\ldots,\bv_k\}\in\IR^n$ with $\forall i$, $\bv_i\neq\b0$, orthogonal set $\implies$ $\{\bv_1,\ldots,\bv_k\}\in\IR^n$ linearly independent
\end{theorem}

\begin{definition}[Orthogonal basis]
Let $S$ be a basis of the subspace $W\subset\IR^n$ composed of an orthogonal set of vectors. We say $S$ is an \textbf{orthogonal basis} of $W$
\end{definition}
\end{frame}

\begin{frame}{Proof of Theorem~\ref{th:ortho_implies_LI}}
Assume $\{\bv_1,\ldots,\bv_k\}$ orthogonal set with $\bv_i\neq\b0$ for all $i=1,\ldots,k$. Recall $\{\bv_1,\ldots,\bv_k\}$ is LI if 
\[
c_1\bv_1+\cdots+c_k\bv_k=\b0\iff c_1=\cdots=c_k=0
\]
So assume $c_1,\ldots,c_k\in\IR$ are s.t. $c_1\bv_1+\cdots+c_k\bv_k=\b0$.
Recall that $\forall\bx\in\IR^k$, $\b0_k\bullet\bx=0$. So for some $\bv_i\in\{\bv_1,\ldots,\bv_k\}$
\begin{align}
0 &= \b0\bullet\bv_i \nonumber \\
&= (c_1\bv_1+\cdots+c_k\bv_k)\bullet\bv_i \nonumber \\
&= c_1\bv_1\bullet\bv_i+\cdots+c_k\bv_k\bullet\bv_i \label{eq:proof_th_ortho_implies_LI}
\end{align}
As $\{\bv_1,\ldots,\bv_k\}$ orthogonal, $\bv_j\bullet\bv_i=0$ when $i\neq j$, \eqref{eq:proof_th_ortho_implies_LI} reduces to
\[
c_i\bv_i\bullet\bv_i = 0 \iff c_i\|\bv_i\|^2 = 0
\]
As $\bv_i\neq 0$ for all $i$, $\|\bv_i\|\neq 0$ and so $c_i=0$. This is true for all $i$, hence the result \hfill\qed
\end{frame}

\begin{frame}{Example -- Vectors of the standard basis of $\IR^3$}
For $\IR^3$, we denote
\[
\bi =\begin{pmatrix}
1\\0\\0
\end{pmatrix},\quad
\bj =\begin{pmatrix}
0\\1\\0
\end{pmatrix}
\textrm{ and }
\bk =\begin{pmatrix}
0\\0\\1
\end{pmatrix}
\]
($\IR^k$ for $k>3$, we denote them $\be_i$)
\vfill
Clearly, $\{\bi,\bj\}$, $\{\bi,\bk\}$, $\{\bj,\bk\}$ and $\{\bi,\bj,\bk\}$ orthogonal sets. The standard basis vectors are also $\neq\b0$, so the sets are LI. And
\[
\{\bi,\bj,\bk\}
\]
is an orthogonal basis of $\IR^3$ since it spans $\IR^3$ and is LI
\vfill
\[
c_1\bi+c_2\bj+c_3\bk
=
c_1\begin{pmatrix}
1\\0\\0
\end{pmatrix}
+c_2\begin{pmatrix}
0\\1\\0
\end{pmatrix}
+c_3\begin{pmatrix}
0\\0\\1
\end{pmatrix}
=
\begin{pmatrix}
c_1\\c_2\\c_3
\end{pmatrix}
\]
\end{frame}

\begin{frame}{Orthonormal version of things}
\begin{definition}[Orthonormal set]
The set of vectors $\{\bv_1,\ldots,\bv_k\}\in\IR^n$ is an \textbf{orthonormal set} if it is an orthogonal set and furthermore
\[
\forall i=1,\ldots,k,\quad \|\bv_i\|=1
\]
\end{definition}
\begin{definition}[Orthonormal basis]
A basis of the subspace $W\subset\IR^n$ is an \textbf{orthonormal basis} if the vectors composing it are an orthonormal set
\end{definition}
\vfill
$\{\bv_1,\ldots,\bv_k\}\in\IR^n$ is orthonormal if
\[
\bv_i\bullet\bv_j =
\begin{cases}
1 &\textrm{if }i=j \\
0 &\textrm{otherwise}
\end{cases}
\]
\end{frame}


\begin{frame}{Projections}
	\begin{definition}[Orthogonal projection onto a subspace]
	$W\subset\IR^n$ a subspace and $\{\bu_1,\ldots,\bu_k\}$ an orthogonal basis of $W$. $\forall\bv\in\IR^n$, the \textbf{orthogonal projection} of $\bv$ \textbf{onto} $W$ is
	\[
	\mathsf{proj}_W(\bv) =
	\frac{\bu_1\bullet\bv}{\|\bu_1\|^2}\bu_1
	+\cdots+
	\frac{\bu_k\bullet\bv}{\|\bu_k\|^2}\bu_k
	\]  
	\end{definition}
	\vfill
	\begin{definition}[Component orthogonal to a subspace]
	$W\subset\IR^n$ a subspace and $\{\bu_1,\ldots,\bu_k\}$ an orthogonal basis of $W$. $\forall\bv\in\IR^n$, the \textbf{component} of  $\bv$ \textbf{orthogonal to} W is
	\[
	\mathsf{perp}_W(\bv)=\bv-\mathsf{proj}_W(\bv)
	\]
	\end{definition}	
	\end{frame}
	
	
	\begin{frame}
	What this aims to do is to construct an orthogonal basis for a subspace $W\subset\IR^n$
	\vfill
	To do this, we use the \emph{Gram-Schmidt orthogonalisation process}, which turn s a basis of $W$ into an orthogonal basis of $W$
	\end{frame}
	
	\begin{frame}{Gram-Schmidt process}
	\begin{theorem}
	$W\subset\IR^n$ a subset and $\{\bx_1,\ldots,\bx_k\}$ a basis of $W$. Let
	\begin{align*}
	\bv_1 &= \bx_1 \\
	\bv_2 &= \bx_2 -\frac{\bv_1\bullet\bx_2}{\|\bv_1\|^2}\bv_1 \\
	\bv_3 &= \bx_3 -\frac{\bv_1\bullet\bx_3}{\|\bv_1\|^2}\bv_1 -\frac{\bv_2\bullet\bx_3}{\|\bv_2\|^2}\bv_2 \\
	&\;\;\vdots & \\
	\bv_k &= \bx_k -\frac{\bv_1\bullet\bx_k}{\|\bv_1\|^2}\bv_1 -\cdots-\frac{\bv_{k-1}\bullet\bx_k}{\|\bv_{k-1}\|^2}\bv_{k-1}
	\end{align*}
	and
	\[
	W_1=\mathsf{span}(\bx_1),W_2 = \mathsf{span}(\bx_1,\bx_2),\ldots,
	W_k = \mathsf{span}(\bx_1,\ldots,\bx_k)
	\]
	Then $\forall i=1,\ldots,k$, $\{\bv_1,\ldots,\bv_i\}$ orthogonal basis for $W_i$
	\end{theorem}
	\end{frame}
	
	


%%%%%%%%%%%%%%%%%%%%%
%%%%%%%%%%%%%%%%%%%%%
%%%%%%%%%%%%%%%%%%%%%
%%%%%%%%%%%%%%%%%%%%%
\section{Least squares problems}


\begin{frame}{The least squares problem (simplest version)}
	\begin{definition}
		Given a collection of points $(x_1,y_1),\ldots,(x_n,y_n)$, find the coefficients $a,b$ of the line $y=a+bx$ such that
		$$
		\|\mathbf{e}\|=\sqrt{\varepsilon_1^2+\cdots+\varepsilon_n^2}
		=\sqrt{(y_1-\tilde y_1)^2+\cdots+(y_n-\tilde y_n)^2}
		$$
		is minimal, where $\tilde y_i=a+bx_i$ for $i=1,\ldots,n$
	\end{definition}
	\vfill
	We just saw how to solve this by brute force using a genetic algorith to minimise $\|e\|$, let us now see how to solve this problem ``properly''
\end{frame}


\begin{frame}
	For a data point $i=1,\ldots,n$
	\[
	\varepsilon_i = y_i-\tilde y_i = y_i - (a+bx_i)
	\]
	So if we write this for all data points,
	\begin{align*}
	\varepsilon_1 &= y_1 - (a+bx_1) \\
	&\;\;\vdots \\
	\varepsilon_n &= y_n - (a+bx_n) \\
	\end{align*}
	In matrix form
	\[
	\be = \bb-A\bx
	\]
	with
	\[
	\be = \begin{pmatrix}
	\varepsilon_1\\ \vdots\\ \varepsilon_n
	\end{pmatrix},
	A=\begin{pmatrix}
	1 & x_1 \\ \vdots & \vdots \\ 1 & x_n
	\end{pmatrix},
	\bx = \begin{pmatrix}
	a\\b
	\end{pmatrix}\textrm{ and }
	\bb = \begin{pmatrix}
	y_1\\ \vdots\\ y_n
	\end{pmatrix}
	\]
\end{frame}

\begin{frame}{The least squares problem (reformulated)}
\begin{definition}[Least squares solutions]
Consider a collection of points $(x_1,y_1),\ldots,(x_n,y_n)$, a matrix $A\in\M_{mn}$, $\bb\in\IR^m$. A \textbf{least squares solution} of $A\bx=\bb$ is a vector $\tilde \bx\in\IR^n$ s.t.
\[
\forall \bx\in\IR^n,\quad \|\bb-A\tilde\bx\|\leq \|\bb-A\bx\|
\]
\end{definition}
\end{frame}


\begin{frame}{Needed to solve the problem}
\begin{definition}[Best approximation]
Let $V$ be a vector space, $W\subset V$ and $\mathbf{v}\in V$. The \textbf{best approximation} to $\mathbf{v}$ in $W$ is $\tilde{\mathbf{v}}\in W$ s.t.
\[
\forall\mathbf{w}\in W, \mathbf{w}\neq\tilde{\mathbf{v}}, \quad
\|\mathbf{v}-\tilde{\mathbf{v}}\| < \|\mathbf{v}-\mathbf{w}\|
\]
\end{definition}
\vfill
\begin{theorem}[Best approximation theorem]
Let $V$ be a vector space with an inner product, $W\subset V$ and $\mathbf{v}\in V$. Then $\mathsf{proj}_W(\mathbf{v})$ is the best approximation to $\mathbf{v}$ in W
\end{theorem}
\end{frame}


\begin{frame}{Let us find the least squares solution}
$\forall \bx\IR^n$, $A\bx$ is a vector in the \textbf{column space} of $A$ (the space spanned by the vectors making up the columns of $A$)
\vfill
Since $\bx\in\IR^n$, $A\bx\in\mathsf{col}(A)$
\vfill
$\implies$ least squares solution of $A\bx=\bb$ is a vector $\tilde\by\in\mathsf{col}(A)$ s.t.
\[
\forall\by\in\mathsf{col}(A),\quad\|\bb-\tilde\by\|\leq\|\bb-\by\|
\]
\vfill
This looks very much like Best approximation and Best approximation theorem
\end{frame}

\begin{frame}{Putting things together}
We just stated: The least squares solution of $A\bx=\bb$ is a vector $\tilde\by\in\mathsf{col}(A)$ s.t.
\[
\forall\by\in\mathsf{col}(A),\quad\|\bb-\tilde\by\|\leq\|\bb-\by\|
\]
\vfill
We know (reformulating a tad):
\begin{theorem}[Best approximation theorem]
Let $V$ be a vector space with an inner product, $W\subset V$ and $\mathbf{v}\in V$. Then $\mathsf{proj}_W(\mathbf{v})\in W$ is the best approximation to $\mathbf{v}$ in W, i.e.,
\[
\forall\mathbf{w}\in W, \mathbf{w}\neq\mathsf{proj}_W(\mathbf{v}), \quad
\|\mathbf{v}-\mathsf{proj}_W(\mathbf{v})\| < \|\mathbf{v}-\mathbf{w}\|
\]
\end{theorem}
\vfill
$\implies$ $W=\mathsf{col}(A)$, $\bv=\bb$ and $\tilde\by=\mathsf{proj}_{\mathsf{col}(A)}(\mathbf{b})$
\end{frame}

\begin{frame}
So if $\tilde\bx$ is a least squares solution of $A\bx=\bb$, then
\[
\tilde\by = A\tilde\bx = \mathsf{proj}_{\mathsf{col}(A)}(\mathbf{b})
\]
\vfill
We have
\[
\bb-A\tilde\bx = \bb-\mathsf{proj}_{\mathsf{col}(A)}(\mathbf{b}) 
= \mathsf{perp}_{\mathsf{col}(A)}(\mathbf{b})
\]
and it is easy to show that
\[
\mathsf{perp}_{\mathsf{col}(A)}(\mathbf{b}) \perp \mathsf{col}(A)
\]
\vfill
So for all columns $\ba_i$ of $A$
\[
\ba_i\boldsymbol{\cdot}(\bb-A\tilde\bx) = 0
\]
which we can also write as $\ba_i^T(\bb-A\tilde\bx) = 0$
\end{frame}

\begin{frame}
For all columns $\ba_i$ of $A$,
\[\ba_i^T(\bb-A\tilde\bx) = 0
\]
\vfill
This is equivalent to saying that
\[
A^T(\bb-A\tilde\bx) = \b0
\]
\vfill
We have
\begin{align*}
A^T(\bb-A\tilde\bx) = \b0 &\iff A^T\bb - A^TA\tilde\bx = \b0 \\
&\iff A^T\bb = A^TA\tilde\bx \\
&\iff A^TA\tilde\bx = A^T\bb
\end{align*}
The latter system constitutes the \textbf{normal equations} for $\tilde\bx$
\end{frame}


\begin{frame}{Least squares theorem}
\begin{importanttheorem}[Least squares theorem]\label{th:least_squares}
$A\in\M_{mn}$, $\bb\in\IR^m$. Then
\begin{enumerate}
\item $A\bx=\bb$ always has at least one least squares solution $\tilde\bx$
\item $\tilde\bx$ least squares solution to $A\bx=\bb$ $\iff$ $\tilde\bx$ is a solution to the normal equations $A^TA\tilde\bx = A^T\bb$
\item $A$ has linearly independent columns $\iff$ $A^TA$ invertible.  
\newline In this case, the least squares solution is unique and 
\[
\tilde\bx = \left(A^TA\right)^{-1}A^T\bb
\]
\end{enumerate}
\end{importanttheorem}
\vfill
We have seen 1 and 2, we will not show 3 (it is not hard)
\end{frame}


\section{Fitting something more complicated}

\begin{frame}{Suppose we want to fit something a bit more complicated..}
For instance, instead of the affine function
\[
y = a+bx
\]
suppose we want to do the quadratic
\[
y = a_0+a_1x+a_2x^2
\]
or even
\[
y = k_0 e^{k_1x}
\]
\vfill
How do we proceed?
\end{frame}


\begin{frame}{Fitting the quadratic}
We have the data points $(x_1,y_1),(x_2,y_2),\ldots,(x_n,y_n)$ and want to fit
\[
y = a_0+a_1x+a_2x^2
\]
At $(x_1,y_1)$,
\[
\tilde y_1 = a_0+a_1x_1+a_2x_1^2
\]
$\vdots$\\
At $(x_n,y_n)$,
\[
\tilde y_n = a_0+a_1x_n+a_2x_n^2
\]
\end{frame}

\begin{frame}
In terms of the error
\begin{align*}
\varepsilon_1 &= y_1-\tilde y_1 = y_1-(a_0+a_1x_1+a_2x_1^2) \\
&\;\;\vdots\\
\varepsilon_n &= y_n-\tilde y_n = y_n-(a_0+a_1x_n+a_2x_n^2)
\end{align*}
i.e.,
\[
\be = \bb-A\bx 
\]
where
\[
\be = \begin{pmatrix}
\varepsilon_1\\ \vdots\\ \varepsilon_n
\end{pmatrix},
A=\begin{pmatrix}
1 & x_1 & x_1^2\\ \vdots & \vdots & \vdots \\ 1 & x_n & x_n^2
\end{pmatrix},
\bx = \begin{pmatrix}
a_0\\a_1\\a_2
\end{pmatrix}\textrm{ and }
\bb = \begin{pmatrix}
y_1\\ \vdots\\ y_n
\end{pmatrix}
\]
\vfill
Theorem~\ref{th:least_squares} applies, with here $A\in\M_{n3}$ and $\bb\in\IR^n$
\end{frame}


\begin{frame}{Fitting the exponential}
Things are a bit more complicated here
\vfill
If we proceed as before, we get the system
\begin{align*}
y_1 &= k_0 e^{k_1x_1} \\
&\;\;\vdots \\
y_n &= k_0 e^{k_1x_n}
\end{align*}
$e^{k_1x_i}$ is a nonlinear term, it cannot be put in a matrix
\vfill
\emph{However}: take the $\ln$ of both sides of the equation
\[
\ln(y_i) = \ln(k_0e^{k_1x_i}) = \ln(k_0)+\ln(e^{k_1x_i}) = \ln(k_0)+k_1x_i
\]
If $y_i,k_0>0$, then their $\ln$ are defined and we're in business..
\end{frame}

\begin{frame}
\[
\ln(y_i) = \ln (k_0)+k_1x_i
\]
So the system is
\begin{align*}
\by = A\bx+\bb
\end{align*}
with
\[
A = \begin{pmatrix}
x_1\\ \vdots \\ x_n
\end{pmatrix},
\bx = \begin{pmatrix}
k_1
\end{pmatrix},
\bb = \begin{pmatrix}
\ln (k_0)
\end{pmatrix}
\textrm{ and }
\by = \begin{pmatrix}
\ln (y_1)\\ \vdots\\ \ln (y_n)
\end{pmatrix}
\]
\end{frame}



%%%%%%%%%%%%%%%%%%%%%
%%%%%%%%%%%%%%%%%%%%%
%%%%%%%%%%%%%%%%%%%%%
%%%%%%%%%%%%%%%%%%%%%
\section{Orthogonal matrices}

\begin{frame}
\begin{theorem}
Let $Q\in\M_{mn}$. The columns of $Q$ form an orthonormal set if and only if
\[
Q^TQ=\II_n
\]
\end{theorem}
\begin{definition}[Orthogonal matrix]
$Q\in\M_n$ is an \textbf{orthogonal matrix} if its columns form an orthonormal set 
\end{definition}
So $Q\in\M_n$ orthogonal if $Q^TQ=\II$, i.e., $Q^T=Q^{-1}$
\begin{theorem}[NSC for orthogonality]
$Q\in\M_n$ orthogonal $\iff$ $Q^{-1} = Q^T$
\end{theorem}
\end{frame}


\begin{frame}
\begin{theorem}[Orthogonal matrices ``encode" isometries]
\label{th:TFAE_orthogonal_matrices}
Let $Q\in\M_n$. TFAE
\begin{enumerate}
\item $Q$ orthogonal
\item $\forall\bx\in\IR^n$, $\|Q\bx\|=\|\bx\|$
\item $\forall\bx,\by\in\IR^n$, $Q\bx\bullet Q\by=\bx\bullet\by$
\end{enumerate}
\end{theorem}
\vfill
\begin{theorem}\label{th:properties_orthogonal_matrices}
Let $Q\in\M_n$ be orthogonal. Then
\begin{enumerate}
\item The rows of $Q$ form an orthonormal set
\item $Q^{-1}$ orthogonal
\item $\det Q=\pm 1$
\item $\forall\lambda\in\sigma(Q)$, $|\lambda|=1$
\item If $Q_2\in\M_n$ also orthogonal, then $QQ_2$ orthogonal
\end{enumerate}
\end{theorem}
\end{frame}



\begin{frame}{Proof of 4 in Theorem~\ref{th:properties_orthogonal_matrices}}
All statements in Theorem~\ref{th:properties_orthogonal_matrices} are easy, but let's focus on 4 
\vfill
Let $\lambda$ be an eigenvalue of $Q\in\M_n$ orthogonal, i.e., $\exists\IR^n\ni\bx\neq\b0$ s.t.
\[
Q\bx = \lambda\bx
\]
Take the norm on both sides
\[
\|Q\bx\| = \|\lambda\bx\|
\]
From 2 in Theorem~\ref{th:TFAE_orthogonal_matrices}, $\|Q\bx\|=\|\bx\|$ and from the properties of norms, $\|\lambda\bx\|=|\lambda|\;\|\bx\|$, so we have
\[
\|Q\bx\| = \|\lambda\bx\| \iff \|\bx\| = |\lambda|\;\|\bx\| \iff 1=|\lambda|
\]
(we can divide by $\|\bx\|$ since $\bx\neq \b0$ as an eigenvector)\hfill\qed
\end{frame}



%%%%%%%%%%%%%%%%%%%%%
%%%%%%%%%%%%%%%%%%%%%
%%%%%%%%%%%%%%%%%%%%%
%%%%%%%%%%%%%%%%%%%%%
\section{The QR factorisation}

\begin{frame}{Matrix factorisations}
	Matrix factorisations are popular because they allow to perform some computations more easily
	\vfill
	There are several different types of factorisations. Here, we study just the QR factorisation, which is useful for many least squares problems
\end{frame}	
	

\begin{frame}{The QR factorisation}
\begin{theorem}\label{th:QR_factorisation}
Let $A\in\M_{mn}$ with LI columns. Then $A$ can be factored as
\[
A=QR
\]
where $Q\in\M_{mn}$ has orthonormal columns and $R\in\M_n$ is nonsingular upper triangular
\end{theorem}
\end{frame}


\begin{frame}{Back to least squares}
So what was the point of all that..?
\vfill
\begin{theorem}[Least squares with QR factorisation]
\label{th:LSQ_with_QR}
$A\in\M_{mn}$ with LI columns, $\bb\in\IR^m$. If $A=QR$ is a QR factorisation of $A$, then the unique least squares solution $\tilde\bx$ of $A\bx=\bb$ is
\[
\tilde\bx = R^{-1}Q^T\bb
\]
\end{theorem}
\end{frame}


\begin{frame}{Proof of Theorem~\ref{th:LSQ_with_QR}}
$A$ has LI columns so 
\begin{itemize}
\item least squares $A\bx=\bb$ has unique solution $\tilde\bx=(A^TA)^{-1}A^T\bb$
\item by Theorem~\ref{th:QR_factorisation}, $A$ can be written as $A=QR$ with $Q\in\M_{mn}$ with orthonormal columns and $R\in\M_n$ nonsingular and upper triangular
\end{itemize}
So
\begin{align*}
A^TA\tilde\bx= A^T\bb &\implies (QR)^TQR\tilde\bx = (QR)^T\bb \\
&\implies R^TQ^TQR\tilde\bx = R^TQ^T\bb \\
&\implies R^T\II_nR\tilde\bx = R^TQ^T\bb \\
&\implies R^TR\tilde\bx = R^TQ^T\bb \\
&\implies (R^T)^{-1}R\tilde\bx = (R^T)^{-1}R^TQ^T\bb \\
&\implies R\tilde\bx = Q^T\bb \\
&\implies \tilde\bx = R^{-1}Q^T\bb\hfill\qed
\end{align*}
\end{frame}



%%%%%%%%%%%%%%%%%%%%%
%%%%%%%%%%%%%%%%%%%%%
%%%%%%%%%%%%%%%%%%%%%
%%%%%%%%%%%%%%%%%%%%%
\section{The singular values decomposition (SVD)}

\begin{frame}{Matrix factorisations (continued)}
The singular value decomposition (known mostly by its acronym, SVD) is yet another type of factorisation/decomposition..
\end{frame}


\begin{frame}{Singular values}
\begin{definition}[Singular value]
Let $A\in\M_{mn}(\IR)$. The \textbf{singular values} of $A$ are the real numbers 
\[
\sigma_1\geq \sigma_2\geq\cdots\sigma_n\geq 0
\]
that are the square roots of the eigenvalues of $A^TA$
\end{definition}
\end{frame}


\begin{frame}{Singular values are real and nonnegative?}
Recall that $\forall A\in\M_{mn}$, $A^TA$ is symmetric
\vfill
\textbf{Claim 1.} Real symmetric matrices have real eigenvalues
\vfill
\textbf{Proof.} $A\in\M_n(\IR)$ symmetric and $(\lambda,\bv)$ eigenpair of $A$, i.e, $A\bv=\lambda\bv$. Taking the complex conjugate, $\overline{A\bv}=\overline{\lambda\bv}$
\vfill
Since $A\in\M_n(\IR)$, $\overline{A}=A$\qquad ($z=\bar z\iff z\in\IR$)
\vfill
So
\[
A\bar\bv=\overline{A}\bar\bv=\overline{A\bv}=\overline{\lambda\bv}=\overline{\lambda}\bar\bv
\]
i.e., if $(\lambda,\bv)$ eigenpair, $(\bar\lambda,\bar\bv)$ also eigenpair
\end{frame}

\begin{frame}
Still assuming $A\in\M_n(\IR)$ symmetric and $(\lambda,\bv)$ eigenpair of $A$ and using what we just proved (that $(\bar\lambda,\bar\bv)$ also eigenpair), take transposes
\begin{align*}
A\bar\bv = \bar\lambda\bar\bv &\iff (A\bar\bv)^T = (\bar\lambda\bar\bv)^T \\
&\iff \bar\bv^TA^T=\bar\lambda\bar\bv^T \\
&\iff \bar\bv^T A = \bar\lambda\bar\bv^T \qquad{\textrm{[$A$ symmetric]}}
\end{align*}
\vfill
Let us now compute $\lambda (\bar\bv\bullet\bv)$. We have
\begin{align*}
\lambda (\bar\bv\bullet\bv) &= \lambda\bar\bv^T\bv = \bar\bv^T(\lambda\bv) \\
&= \bar\bv^T(A\bv) = (\bar\bv^TA)\bv \\
&= (\bar\lambda\bar\bv^T)\bv = \bar\lambda(\bar\bv\bullet\bv) \\
&\iff (\lambda-\bar\lambda)(\bar\bv\bullet\bv) = 0
\end{align*}
\end{frame}

\begin{frame}
We have shown
\[
(\lambda-\bar\lambda)(\bar\bv\bullet\bv) = 0
\]
Let 
\[
\bv = \begin{pmatrix}
a_1+ib_1 \\
\vdots \\
a_n+ib_n
\end{pmatrix}
\]
Then
\[
\bar\bv = \begin{pmatrix}
a_1-ib_1 \\
\vdots \\
a_n-ib_n
\end{pmatrix}
\]
So
\[
\bar\bv\bullet\bv = (a_1^2+b_1^2)+\cdots+(a_n^2+b_n^2)
\]
But $\bv$ eigenvector is $\neq\b0$, so $\bar\bv\bullet\bv\neq 0$, so
\[
(\lambda-\bar\lambda)(\bar\bv\bullet\bv) = 0
\iff \lambda-\bar\lambda=0
\iff \lambda=\bar\lambda\iff \lambda\in\IR\qed
\]
\end{frame}


\begin{frame}
\textbf{Claim 2.} For $A\in\M_{mn}(\IR)$, the eigenvalues of $A^TA$ are real and nonnegative

\vfill
\textbf{Proof.}
We know that for $A\in\M_{mn}$, $A^TA$ symmetric and from previous claim, if $A\in\M_{mn}(\IR)$, then $A^TA$ is symmetric and real and with real eigenvalues
\vfill
Let $(\lambda,\bv)$ be an eigenpair of $A^TA$, with $\bv$ chosen so that $\|\bv\|=1$
\vfill 
Norms are functions $V\to\IR_+$, so $\|A\bv\|$ and $\|A\bv\|^2$ are $\geq 0$ and thus
\begin{align*}
0\leq \|A\bv\|^2 &= (A\bv)\bullet(A\bv) = (A\bv)^T(A\bv) \\
&= \bv^TA^TA\bv = \bv^T(A^TA\bv) = \bv^T(\lambda\bv) \\
&= \lambda(\bv^T\bv) = \lambda(\bv\bullet\bv) = \lambda\|\bv\|^2 \\
&= \lambda\hfill\qed
\end{align*}
\end{frame}

\begin{frame}
\textbf{Claim 3.} For $A\in\M_{mn}(\IR)$, the nonzero eigenvalues of $A^TA$ and $AA^T$ are the same
\vfill
\textbf{Proof.}
Let $(\lambda,\bv)$ be an eigenpair of $A^TA$ with $\lambda\neq 0$. Then $\bv\neq\b0$ and
\[
	A^TA\bv=\lambda\bv\neq\b0
\]
Left multiply by $A$
\[
	AA^TA\bv = \lambda A\bv
\]
Let $\bw=A\bv$, we thus have $AA^T\bw=\lambda\bw$; in other words, $A\bv$ is an eigenvector of $AA^T$ corresponding to the (nonzero) eigenvalue $\lambda$
\vfill
The reverse works the same way.. \qed
\end{frame}


\begin{frame}{The singular value decomposition (SVD)}
\begin{importanttheorem}[SVD]\label{th:SVD}
$A\in\M_{mn}$ with singular values $\sigma_1\geq\cdots\geq\sigma_r>0$ and $\sigma_{r+1}=\cdots=\sigma_n=0$
\vskip0.5cm
Then there exists $U\in\M_m$ orthogonal, $V\in\M_n$ orthogonal and a block matrix $\Sigma\in\M_{mn}$ taking the form
\[
\Sigma=
\begin{pmatrix}
D & 0_{r,n-r} \\
0_{m-r,r} & 0_{m-r,n-r}
\end{pmatrix}
\]
where 
\[
D = \mathsf{diag}(\sigma_1,\ldots,\sigma_r)\in\M_r
\] 
such that
\[
A=U\Sigma V^T
\]
\end{importanttheorem}
\end{frame}


\begin{frame}
\begin{definition}
We call a factorisation as in Theorem~\ref{th:SVD} the \textbf{singular value decomposition} of $A$. The columns of $U$ and $V$ are, respectively, the \textbf{left} and \textbf{right singular vectors} of $A$
\end{definition}
\vfill
$U$ and $V^T$ are \emph{rotation} or \emph{reflection} matrices, $\Sigma$ is a \emph{scaling} matrix
\vfill
$U\in\M_m$ orthogonal matrix with columns the eigenvectors of $AA^T$
\vfill
$V\in\M_n$ orthogonal matrix with columns the eigenvectors of $A^TA$
\end{frame}


\begin{frame}{Outer product form of the SVD}
\begin{theorem}[Outer product form of the SVD]\label{th:SVD_outer_product_form}
$A\in\M_{mn}$ with singular values $\sigma_1\geq\cdots\geq\sigma_r>0$ and $\sigma_{r+1}=\cdots=\sigma_n=0$, $\bu_1,\ldots,\bu_r$ and $\bv_1,\ldots,\bv_r$, respectively, left and right singular vectors of $A$ corresponding to these singular values
\vskip0.5cm
Then 
\[
A=\sigma_1\bu_1\bv_1^T+\cdots+\sigma_r\bu_r\bv_r^T
\]
\end{theorem}
\end{frame}


\begin{frame}{Computing the SVD (case of $\neq$ eigenvalues)}
To compute the SVD, we use the following result
\vfill
\begin{theorem}\label{th:eigenvectors_of_symmetric_are_orthogonal}
Let $A\in\M_n$ symmetric, $(\lambda_1,\bu_1)$ and $(\lambda_2,\bu_2)$ be eigenpairs, $\lambda_1\neq\lambda_2$. Then $\bu_1\bullet\bu_2=0$
\end{theorem}
\end{frame}

\begin{frame}{Proof of Theorem~\ref{th:eigenvectors_of_symmetric_are_orthogonal}}
$A\in\M_n$ symmetric, $(\lambda_1,\bu_1)$ and $(\lambda_2,\bu_2)$ eigenpairs with $\lambda_1\neq\lambda_2$
\begin{align*}
\lambda_1(\bv_1\bullet\bv_2) 
&= (\lambda_1\bv_1)\bullet\bv_2 \\
&= A\bv_1\bullet\bv_2 \\
&= (A\bv_1)^T\bv_2 \\
&= \bv_1^TA^T\bv_2 \\
&= \bv_1^T(A\bv_2)  \qquad\textrm{[$A$ symmetric so $A^T=A$]} \\
&= \bv_1^T(\lambda_2\bv_2) \\
&= \lambda_2(\bv_1^T\bv_2) \\
&= \lambda_2(\bv_1\bullet\bv_2)
\end{align*}
\vfill
So $(\lambda_1-\lambda_2)(\bv_1\bullet\bv_2)=0$. But $\lambda_1\neq\lambda_2$, so $\bv_1\bullet\bv_2=0$\hfill\qed
\end{frame}


\begin{frame}{Computing the SVD (case of $\neq$ eigenvalues)}
If all eigenvalues of $A^TA$ (or $AA^T$) are distinct, we can use Theorem~\ref{th:eigenvectors_of_symmetric_are_orthogonal}
\vfill
\begin{enumerate}
\item Compute $A^TA\in\M_n$
\item Compute eigenvalues $\lambda_1,\ldots,\lambda_n$ of $A^TA$; order them as $\lambda_1>\cdots>\lambda_n\geq 0$ ($>$ not $\geq$ since $\neq$)
\item Compute singular values $\sigma_1=\sqrt{\lambda_1},\ldots,\sigma_n=\sqrt{\lambda_n}$
\item Diagonal matrix $D$ in $\Sigma$ is either in $\M_n$ (if $\sigma_n>0$) or in $\M_{n-1}$ (if $\sigma_n=0$)
\end{enumerate}
\end{frame}


\begin{frame}
\begin{enumerate}
\setcounter{enumi}{4}
\item Since eigenvalues are distinct, Theorem~\ref{th:eigenvectors_of_symmetric_are_orthogonal} $\implies$ eigenvectors are orthogonal set. Compute these eigenvectors in the same order as the eigenvalues
\item Normalise them and use them to make the matrix $V$, i.e., $V=[\bv_1\cdots\bv_n]$
\item To find the $\bu_i$, compute, for $i=1,\ldots,r$,
\[
\bu_i = \frac{1}{\sigma_i}A\bv_i
\]
and ensure that $\|\bu_i\|=1$
\end{enumerate}
\end{frame}


\begin{frame}{Computing the SVD (case where some eigenvalues are $=$)}
\begin{enumerate}
\item Compute $A^TA\in\M_n$
\item Compute eigenvalues $\lambda_1,\ldots,\lambda_n$ of $A^TA$; order them as $\lambda_1\geq\cdots\geq\lambda_n\geq 0$
\item Compute singular values $\sigma_1=\sqrt{\lambda_1},\ldots,\sigma_n=\sqrt{\lambda_n}$, with $r\leq n$ the index of the last positive singular value
\item For eigenvalues that are distinct, proceed as before
\item For eigenvalues with multiplicity $>1$, we need to ensure that the resulting eigenvectors are LI \emph{and} orthogonal
\end{enumerate}
\end{frame}

\begin{frame}{Dealing with eigenvalues with multiplicity $>1$}
When an eigenvalue has (algebraic) multiplicity $>1$, e.g., characteristic polynomial contains a factor like $(\lambda-2)^2$, things can become a little bit more complicated
\vfill
The proper way to deal with this involves the so-called Jordan Normal Form (another matrix decomposition)
\vfill
In short: not all square matrices are diagonalisable, but all square matrices admit a JNF
\end{frame}


\begin{frame}
Sometimes, we can find several LI eigenvectors associated to the same eigenvalue. Check this. If not, need to use the following
\vfill
\begin{definition}[Generalised eigenvectors]
$\bx\neq\b0$ \textbf{generalized eigenvector} of rank $m$ of $A\in\M_n$ corresponding to eigenvalue $\lambda$ if
\[
(A-\lambda\II)^{m}\bx = \b0
\]
but
\[
(A-\lambda\II)^{m-1}\bx\neq \b0
\]
\end{definition}
\end{frame}


\begin{frame}{Procedure for generalised eigenvectors}
$A\in\M_n$ and assume $\lambda$ eigenvalue with algebraic multiplicity $k$
\vfill
Find $\bv_1$, ``classic" eigenvector, i.e., $\bv_1\neq\b0$ s.t. $(A-\lambda\II)\bv_1=\b0$
\vfill
Find generalised eigenvector $\bv_2$ of rank 2 by solving for $\bv_2\neq\b0$,
\[
(A-\lambda\II)\bv_2 = \bv_1
\]
$\ldots$
\vfill
Find generalised eigenvector $\bv_k$ of rank $k$ by solving for $\bv_k\neq\b0$,
\[
(A-\lambda\II)\bv_k = \bv_{k-1}
\]
\vfill
Then $\{\bv_1,\ldots,\bv_k\}$ LI
\end{frame}


\begin{frame}{Back to the normal procedure}
With the LI eigenvectors $\{\bv_1,\ldots,\bv_k\}$ corresponding to $\lambda$
\vfill
Apply Gram-Schmidt to get orthogonal set
\vfill
For all eigenvalues with multiplicity $>1$, check that you either have LI eigenvectors or do what we just did
\vfill
When you are done, be back on your merry way to step 6 in the case where eigenvalues are all $\neq$
\vfill
I am caricaturing a little here: there can be cases that do not work exactly like this, but this is general enough..
\end{frame}

\begin{frame}{Applications of the SVD}
Many applications of the SVD, both theoretical and practical..
\vfill
\begin{enumerate}
\item Obtaining a unique solutions to least squares when $A^TA$ singular
\item Image compression
\end{enumerate}
\end{frame}


\begin{frame}{Least squares revisited}
\begin{theorem}
Let $A\in\M_{mn}$, $\bx\in\IR^n$ and $\bb\in\IR^m$. The least squares problem $A\bx=\bb$ has a unique least squares solution $\tilde\bx$ of \emph{minimal length} (closest to the origin) given by
\[
\tilde\bx = A^+\bb
\]
where $A^+$ is the \emph{pseudoinverse} of $A$
\end{theorem}
\end{frame}

\begin{frame}
\begin{definition}[Pseudoinverse]
$A=U\Sigma V^T$ an SVD for $A\in\M_{mn}$, where 
\[
\Sigma = \begin{pmatrix}
D & 0 \\ 0 & 0
\end{pmatrix},
\textrm{ with }
D=\mathsf{diag}(\sigma_1,\ldots,\sigma_r)
\]
($D$ contains the nonzero singular values of $A$ ordered as usual)
\vskip0.5cm
The \textbf{pseudoinverse} (or \textbf{Moore-Penrose inverse}) of $A$ is $A^+\in\M_{nm}$ given by
\[
A^+ = V\Sigma^+ U^T
\]
with
\[
\Sigma^+ =
\begin{pmatrix}
D^{-1} & 0 \\ 0 & 0
\end{pmatrix}\in\M_{nm}
\]
\end{definition}
\end{frame}


\begin{frame}{Compressing images}
Consider an image (for simplicity, assume in shades of grey). This can be stored in a matrix $A\in\M_{mn}$
\vfill
Take the SVD of $A$. Then the small singular values carry information about the regions with little variation and can perhaps be omitted, whereas the large singular values carry information about more ``dynamic'' regions of the image
\vfill
Suppose $A$  has $r$ nonzero singular values. For $k\leq r$, let
\[
A_k = \sigma_1\bu_1\bv_1^T+\cdots+\sigma_k\bu_k\bv_k^T
\]
(so for $k=r$ we get the usual outer product form)
\end{frame}


%%%%%%%%%%%%%%%%%%%%%
%%%%%%%%%%%%%%%%%%%%%
%%%%%%%%%%%%%%%%%%%%%
%%%%%%%%%%%%%%%%%%%%%
\section{Principal component analysis (PCA)}


\begin{frame}{Dimensionality reduction}
One of the reasons the SVD is used is for dimensionality reduction. However, SVD has many many other uses
\vfill
Now we look at another dimensionality reduction technique, PCA
\vfill
PCA is often used as a blackbox technique, here we take a look at the math behind it
\end{frame}


\begin{frame}{What is PCA?}
Linear algebraic technique 
\vfill
Helps reduce a complex dataset to a lower dimensional one
\vfill
Non-parametric method: does not assume anything about data distribution (distribution from the statistical point of view)
\end{frame}


\begin{frame}{Brief ``review'' of some probability concepts}
Proper definition of \emph{probability} requires to use \emph{measure theory}.. will not get into details here
\vfill
A \textbf{random variable} $X$ is a \emph{measurable} function $X:\Omega\to E$, where $\Omega$ is a set of outcomes (\emph{sample space}) and $E$ is a measurable space
\vfill
$\IP(X\in S\subseteq E) = \IP(\omega\in\Omega|X(\omega)\in S)$
\vfill
\textbf{Distribution function} of a r.v., $F(x)=\IP(X\leq x)$, describes the distribution of a r.v.
\vfill
R.v. can be discrete or continuous or .. other things. 
\end{frame}

\begin{frame}
\begin{definition}[Variance]
Let $X$ be a random variable. The \textbf{variance} of $X$ is given by
\[
\Var X = E\left[\left(X-E(X)\right)^2\right]
\]
where $E$ is the expected value
\end{definition}
\vfill
\begin{definition}[Covariance]
Let $X,Y$ be jointly distributed random variables. The \textbf{covariance} of $X$ and $Y$ is given by
\[
\cov (X,Y) = E\left[\left(X-E(X)\right)\left(Y-E(Y)\right)\right]
\]
\end{definition}
\vfill
Note that $\cov(X,X)=E\left[\left(X-E(X)\right)^2\right] = \Var X$
\end{frame}

\begin{frame}{In practice: ``true law'' versus ``observation''}
In statistics: we reason on the \emph{true law} of distributions, but we usually have only access to a sample
\vfill
We then use \textbf{estimators} to .. estimate the value of a parameter, e.g., the mean, variance and covariance
\vfill
\end{frame}
    
\begin{frame}
\begin{definition}[Unbiased estimators of the mean and variance]
Let $x_1,\ldots,x_n$ be data points (the \emph{sample}) and 
\[
\bar x = \frac 1n \sum_{i=1}^n x_i
\]
be the \textbf{mean} of the data. An unbiased estimator of the variance of the sample is
\[
\sigma^2 = \frac{1}{n-1}\sum_{i=1}^n (x_i-\bar x)^2
\]
\end{definition}
\end{frame}

\begin{frame}
\begin{definition}[Unbiased estimator of the covariance]
Let $(x_1,y_1),\ldots,(x_n,y_n)$ be data points,
\[
\bar x = \frac 1n \sum_{i=1}^n x_i
\textrm{ and }
\bar y = \frac 1n \sum_{i=1}^n y_i
\]
be the means of the data. An estimator of the covariance of the sample is
\[
\cov(x,y) = \frac{1}{n}\sum_{i=1}^n (x_i-\bar x)(y_i-\bar y)
\]
\end{definition}
\end{frame}

\begin{frame}{What does covariance do?}
Variance explains how data disperses around the mean, in a 1-D case
\vfill
Covariance measures the relationship between two dimensions. E.g., height and weight
\vfill
More than the exact value, the sign is important:
\begin{itemize}
    \item $\cov(X,Y)>0$: both dimensions change in the same ``direction''; e.g., larger height usually means higher weight
    \item $\cov(X,Y)<0$: both dimensions change in reverse directions; e.g., time spent on social media and performance in this class
    \item $\cov(X,Y)=0$: the dimensions are independent from one another; e.g., sex/gender and ``intelligence''
\end{itemize}
\end{frame}

\begin{frame}{The covariance matrix}
Typically, we consider more than 2 variables.. 
\begin{definition}
Suppose $p$ random variables $X_1,\ldots,X_p$. Then the covariance matrix is the symmetric matrix
\[
\begin{pmatrix}
\cov(X_1,X_1) & \cov(X_1,X_2) & \cdots & \cov(X_1,X_p) \\
\cov(X_2,X_1) & \cov(X_2,X_2) & \cdots & \cov(X_2,X_p) \\
\vdots & \vdots & & \vdots \\
\cov(X_p,X_1) & \cov(X_p,X_2) & \cdots & \cov(X_p,X_p) 
\end{pmatrix}
\]
i.e., using the properties of covariance,
\[
\begin{pmatrix}
\Var X_1 & \cov(X_1,X_2) & \cdots & \cov(X_1,X_p) \\
\cov(X_1,X_2) & \Var X_2 & \cdots & \cov(X_2,X_p) \\
\vdots & \vdots & & \vdots \\
\cov(X_1,X_p) & \cov(X_2,X_p) & \cdots & \Var X_p 
\end{pmatrix}
\]
\end{definition}
\end{frame}


\begin{frame}{Example of a PCA problem}
We collect a bunch of information about a bunch of people.. for instance this data from Loughborough University
\vfill
\begin{quote}
This dataset contains the height, weight and 4 fingerprint measurements (length, width, area and circumference), collected from 200 participants.
\end{quote}
\vfill
What best describes a participant?
\end{frame}

\begin{frame}{The variables}
Each participant is associated to 11 variables
\vfill
\begin{itemize}
\item "Participant Number"
\item "Gender"
\item "Age"
\item "Dominant Hand"
\item "Height (cm) (average of 3 measurements)"
\item "Weight (kg) (average of 3 measurements)"
\item "Fingertip Temperature (°C)"
\item "Fingerprint Height (mm)"
\item "Fingerprint Width (mm)"
\item "Fingerprint Area (mm2)"
\item "Fingerprint Circumference (mm)"
\end{itemize}
\end{frame}

\begin{frame}{Nature of variables}
Variables have different natures
\vfill
\begin{itemize}
\item "Participant Number": $\in\IN$ (not interesting)
\item "Gender": categorical
\item "Age": $\in\IN$ 
\item "Dominant Hand": categorical
\item "Height (cm) (average of 3 measurements)": $\in\IR$
\item "Weight (kg) (average of 3 measurements)": $\in\IR$
\item "Fingertip Temperature (°C)": $\in\IR$
\item "Fingerprint Height (mm)": $\in\IR$
\item "Fingerprint Width (mm)": $\in\IR$
\item "Fingerprint Area (mm2)": $\in\IR$
\item "Fingerprint Circumference (mm)": $\in\IR$
\end{itemize}
\end{frame}

\begin{frame}{Setting things up}
Each participant is a row in the matrix (an \emph{observation})
\vfill
Each variable is a column
\vfill
So we have an $200\times 10$ matrix (we discard the ``Participant number'' column)
\vfill
We want to find what carries the most information
\vfill
For this, we are going to project the information in a new basis in which the first ``dimension'' will carry most variance, the second dimension will carry a little less, etc.
\vfill
In order to do so, we need to learn how to change bases
\end{frame}

\begin{frame}
	In the following slide, 
	\[
	[\bx]_\B
	\]
	denotes the coordinates of $\bx$ in the basis $\B$
	\vfill
	The aim of a change of basis is to express vectors in another coordinate system (another basis)
	\vfill
	We do so by finding a matrix allowing to move from one basis to another
\end{frame}

\begin{frame}{Change of basis}
\begin{definition}[Change of basis matrix]
$\B=\{\bu_1,\ldots,\bu_n\}$ and $\C=\{\bv_1,\ldots,\bv_n\}$ bases of vector space $V$
\vskip0.2cm
The \textbf{change of basis matrix} $P_{\C\leftarrow\B}\in\M_n$,
\[
P_{\C\leftarrow\B}
=\left[
[\bu_1]_\C \cdots [\bu_n]_\C
\right]
\]
has columns the coordinate vectors $[\bu_1]_\C,\ldots,[\bu_n]_\C$ of the vectors in $\B$ with respect to $\C$
\end{definition}
\vfill
\begin{theorem}
$\B=\{\bu_1,\ldots,\bu_n\}$ and $\C=\{\bv_1,\ldots,\bv_n\}$ bases of vector space $V$ and $P_{\C\leftarrow\B}$ a change of basis matrix from $\B$ to $\C$
\begin{enumerate}
\item $\forall\bx\in V$, $P_{\C\leftarrow\B}[\bx]_\B = [\bx]_\C$
\item $P_{\C\leftarrow\B}$ s.t. $\forall\bx\in V$, $P_{\C\leftarrow\B}[\bx]_\B = [\bx]_\C$ is \textbf{unique}
\item $P_{\C\leftarrow\B}$ invertible and $P_{\C\leftarrow\B}^{-1}=P_{\B\leftarrow\C}$
\end{enumerate}
\end{theorem}
\end{frame}


\begin{frame}{Row-reduction method for changing bases}
\begin{theorem}
$\B=\{\bu_1,\ldots,\bu_n\}$ and $\C=\{\bv_1,\ldots,\bv_n\}$ bases of vector space $V$. Let $\E$ be any basis for $V$,
\[
B = [[\bu_1]_\E,\ldots,[\bu_n]_\E] 
\textrm{ and }
C = [[\bv_1]_\E,\ldots,[\bv_n]_\E] 
\]
and let $[C|B]$ be the augmented matrix constructed using $C$ and $B$. Then
\[
RREF\left([C|B]\right)
=[\II|P_{\C\leftarrow\B}]
\]
\end{theorem}
\vfill
If working in $\IR^n$, this is quite useful with $\E$ the standard basis of $\IR^n$ (it does not matter if $\B=\E$)
\end{frame}

\begin{frame}
So the question now becomes
\begin{quote}
How do we find what new basis to look at our data in?
\end{quote}
\vfill
(Changing the basis does not change the data, just the view you have of it)
\vfill
(Think of what happens when you do a headstand.. your up becomes down, your right and left switch, but the world does not change, just your view of it)
\vfill
(Changes of bases are \emph{fundamental} operations in Science)
\end{frame}



\begin{frame}{Setting things up}
I will use notation (mostly) as in Joliffe's \emph{Principal Component Analysis} (PDF of older version available for free from UofM Libraries)
\vfill
$\bx=(x_1,\ldots,x_p)$ vector of $p$ random variables
\end{frame}


\begin{frame} 
We seek a linear function $\bm{\alpha}_1^T\bx$ with maximum variance, where $\bm{\alpha}_1=(\alpha_{11},\ldots,\alpha_{1p})$, i.e.,
\[
\bm{\alpha}_1^T\bx = \sum_{j=1}^p\alpha_{1j}x_j
\]
\vfill
Then we seek a linear function $\bm{\alpha}_2^T\bx$ with maximum variance, uncorrelated to $\bm{\alpha}_1^T\bx$
\vfill
And we continue...
\vfill
At $k$th stage, we find a linear function $\bm{\alpha}_k^T\bx$ with maximum variance, uncorrelated to $\bm{\alpha}_1^T\bx,\ldots,\bm{\alpha}_{k-1}^T\bx$
\vfill
$\bm{\alpha}_i^T\bx$ is the $i$th \textbf{principal component} (PC)
\end{frame}

\begin{frame}{Case of known covariance matrix}
Suppose we know $\Sigma$, covariance matrix of $\bx$ (i.e., typically: we know $\bx$)
\vfill
Then the $k$th PC is 
\[
z_k=\bm{\alpha}_k^T\bx
\]
where $\bm{\alpha}_k$ is an eigenvector of $\Sigma$ corresponding to the $k$th largest eigenvalue $\lambda_k$
\vfill
If, additionally, $\|\bm{\alpha}_k\|=\bm{\alpha}_k^T\bm{\alpha}=1$, then $\lambda_k=\Var z_k$
\end{frame}


\begin{frame}{Why is that?}
Let us start with
\[
\bm{\alpha}_1^T\bx
\]
\vfill
We want maximum variance, where $\bm{\alpha}_1=(\alpha_{11},\ldots,\alpha_{1p})$, i.e.,
\[
\bm{\alpha}_1^T\bx = \sum_{j=1}^p\alpha_{1j}x_j
\]
with the constraint that $\|\bm{\alpha}_1\|=1$
\vfill
We have
\[
\Var \bm{\alpha}_1^T\bx
=\bm{\alpha}_1^T\Sigma\bm{\alpha}_1
\]
\end{frame}

\begin{frame}{Objective}
We want to maximise $\Var \bm{\alpha}_1^T\bx$, i.e.,
\[
\bm{\alpha}_1^T\Sigma\bm{\alpha}_1
\]
under the constraint that $\|\bm{\alpha}_1\|=1$
\vfill
$\implies$ use \textbf{Lagrange multipliers}
\end{frame}


\begin{frame}{Maximisation using Lagrange multipliers}
\framesubtitle{(A.k.a. super-brief intro to multivariable calculus)}
We want the max of $f(x_1,\ldots,x_n)$ under the constraint $g(x_1,\ldots,x_n)=k$
\begin{enumerate}
\item Solve
\begin{align*}
\nabla f(x_1,\ldots,x_n) &= \lambda\nabla g(x_1,\ldots,x_n) \\
g(x_1,\ldots,x_n) &= k
\end{align*}
where $\nabla=(\frac{\partial}{\partial x_1},\ldots,\frac{\partial}{\partial x_n})$ is the \textbf{gradient operator}
\item Plug all solutions into $f(x_1,\ldots,x_n)$ and find maximum values (provided values exist and $\nabla g\neq \b0$ there)
\end{enumerate}
\vfill
$\lambda$ is the \textbf{Lagrange multiplier}
\end{frame}


\begin{frame}{The gradient}
\framesubtitle{(Continuing our super-brief intro to multivariable calculus)}
$f:\IR^n\to\IR$ function of several variables, $\nabla=\left(\frac{\partial}{\partial x_1},\ldots,\frac{\partial}{\partial x_n}\right)$ the gradient operator
\vfill
Then
\[
\nabla f = \left(
\frac{\partial}{\partial x_1}f,\ldots,
\frac{\partial}{\partial x_n}f
\right)
\]
\vfill
So $\nabla f$ is a \emph{vector-valued} function, $\nabla f:\IR^n\to\IR^n$; also written as
\[
\nabla f = f_{x_1}(x_1,\ldots,x_n)\be_1+\cdots f_{x_n}(x_1,\ldots,x_n)\be_n
\]
where $f_{x_i}$ is the partial derivative of $f$ with respect to $x_i$ and $\{\be_1,\ldots,\be_n\}$ is the standard basis of $\IR^n$
\end{frame}


\begin{frame}{Bear with me..}
\framesubtitle{(You may experience a brief period of discomfort)}
$\bm{\alpha}_1^T\Sigma\bm{\alpha}_1$ and $\|\bm{\alpha}_1\|^2=\bm{\alpha}_1^T\bm{\alpha_1}$ are functions of $\bm{\alpha}_1=(\alpha_{11},\ldots,\alpha_{1p})$
\vfill
In the notation of the previous slide, we want the max of 
\[
f(\alpha_{11},\ldots,\alpha_{1p}) := \bm{\alpha}_1^T\Sigma\bm{\alpha}_1
\]
under the constraint that
\[
g(\alpha_{11},\ldots,\alpha_{1p}) := \bm{\alpha}_1^T\bm{\alpha_1} = 1
\]
and with gradient operator
\[
\nabla = \left(
\frac{\partial}{\partial \alpha_{11}},
\ldots,
\frac{\partial}{\partial \alpha_{1p}}
\right)
\]
\end{frame}


\begin{frame}{Effect of $\nabla$ on $g$}
$g$ is easiest to see:
\begin{align*}
\nabla g(\alpha_{11},\ldots,\alpha_{1p}) &=
\left(
\frac{\partial}{\partial \alpha_{11}},
\ldots,
\frac{\partial}{\partial \alpha_{1p}}
\right) (\alpha_{11},\ldots,\alpha_{1p}) 
\begin{pmatrix}
\alpha_{11}\\ \vdots\\ \alpha_{1p}
\end{pmatrix} \\
&= \left(
\frac{\partial}{\partial \alpha_{11}},
\ldots,
\frac{\partial}{\partial \alpha_{1p}}
\right) 
\left(
\alpha_{11}^2+\cdots+\alpha_{1p}^2
\right) \\
&= \left(2\alpha_{11},\ldots,2\alpha_{1p}\right)\\
&= 2\bm{\alpha}_1
\end{align*}
\vfill
(And that's a general result: $\nabla\|\bx\|_2^2=2\bx$ with $\|\cdot\|_2$ the Euclidean norm)
\end{frame}

\begin{frame}{Effect of $\nabla$ on $f$}
Expand (write $\Sigma=[s_{ij}]$ and do not exploit symmetry)
\begin{align*}
\bm{\alpha}_1^T\Sigma\bm{\alpha}_1 &=
\left(\alpha_{11},\ldots,\alpha_{1p}\right)
\begin{pmatrix}
s_{11} & s_{12} & \cdots & s_{1p} \\
s_{21} & s_{22} & \cdots & s_{2p} \\
\vdots & \vdots & & \vdots \\
s_{p1} & s_{p2} & & s_{pp}
\end{pmatrix}
\begin{pmatrix}
\alpha_{11} \\ \alpha_{12} \\ \vdots \\ \alpha_{1p}
\end{pmatrix} \\
&=
\left(\alpha_{11},\ldots,\alpha_{1p}\right)
\begin{pmatrix}
s_{11}\alpha_{11}+s_{12}\alpha_{12}+\cdots+s_{1p}\alpha_{1p} \\
s_{21}\alpha_{11}+s_{22}\alpha_{12}+\cdots+s_{2p}\alpha_{1p} \\
\vdots \\
s_{p1}\alpha_{11}+s_{p2}\alpha_{12}+\cdots+s_{pp}\alpha_{1p}
\end{pmatrix} \\
&=
(s_{11}\alpha_{11}+s_{12}\alpha_{12}+\cdots+s_{1p}\alpha_{1p})\alpha_{11} \\
&\quad +
(s_{21}\alpha_{11}+s_{22}\alpha_{12}+\cdots+s_{2p}\alpha_{1p})\alpha_{12} \\
&\quad\;\;\vdots \\
&\quad +
(s_{p1}\alpha_{11}+s_{p2}\alpha_{12}+\cdots+s_{pp}\alpha_{1p})\alpha_{1p}
\end{align*}
\end{frame}

\begin{frame}
We have
\begin{align*}
\bm{\alpha}_1^T\Sigma\bm{\alpha}_1 &=
(s_{11}\alpha_{11}+s_{12}\alpha_{12}+\cdots+s_{1p}\alpha_{1p})\alpha_{11} \\
&\quad +
(s_{21}\alpha_{11}+s_{22}\alpha_{12}+\cdots+s_{2p}\alpha_{1p})\alpha_{12} \\
&\quad\;\;\vdots \\
&\quad +
(s_{p1}\alpha_{11}+s_{p2}\alpha_{12}+\cdots+s_{pp}\alpha_{1p})\alpha_{1p} 
\end{align*}
So
\begin{align*}
\frac{\partial}{\partial \alpha_{11}}
\bm{\alpha}_1^T\Sigma\bm{\alpha}_1  
&= 
(s_{11}\alpha_{11}+s_{12}\alpha_{12}+\cdots+s_{1p}\alpha_{1p})+s_{11}\alpha_{11} \\
&\quad +
s_{21}\alpha_{12} \\
&\quad\;\;\vdots \\
&\quad +
s_{p1}\alpha_{1p} \\
&= s_{11}\alpha_{11}+s_{12}\alpha_{12}+\cdots+s_{1p}\alpha_{1p} \\
&\quad+
s_{11}\alpha_{11}+s_{21}\alpha_{12}+\cdots+s_{p1}\alpha_{1p} \\
&= 2(s_{11}\alpha_{11}+s_{12}\alpha_{12}+\cdots+s_{1p}\alpha_{1p})
\end{align*}
(last equality stems from symmetry of $\Sigma$)
\end{frame}

\begin{frame}
In general, for $i=1,\ldots,p$,
\begin{align*}
\frac{\partial}{\partial \alpha_{1i}}
\bm{\alpha}_1^T\Sigma\bm{\alpha}_1  
&= s_{i1}\alpha_{11}+s_{i2}\alpha_{12}+\cdots+s_{ip}\alpha_{1p}\\
&\quad+s_{i1}\alpha_{11}+s_{2i}\alpha_{12}+\cdots+s_{pi}\alpha_{1p} \\
&= 2(s_{i1}\alpha_{11}+s_{i2}\alpha_{12}+\cdots+s_{ip}\alpha_{1p})
\end{align*}
(because of symmetry of $\Sigma$)
\vfill
As a consequence,
\[
\nabla \bm{\alpha}_1^T\Sigma\bm{\alpha}_1
=2\Sigma\bm{\alpha}_1
\]
\end{frame}

\begin{frame}
So solving
\[
\nabla f(x_1,\ldots,x_n) = \lambda\nabla g(x_1,\ldots,x_n) 
\]
means solving
\[
2\Sigma\bm{\alpha}_1 = \lambda 2\bm{\alpha}_1 
\]
i.e.,
\[
\Sigma\bm{\alpha}_1 = \lambda\bm{\alpha}_1 
\]
\vfill
$\implies$
$(\lambda,\bm{\alpha}_1)$ eigenpair of $\Sigma$, with $\bm{\alpha}_1$ having unit length
\end{frame}


\begin{frame}{Picking the right eigenvalue}
$(\lambda,\bm{\alpha}_1)$ eigenpair of $\Sigma$, with $\bm{\alpha}_1$ having unit length
\vfill
But which $\lambda$ to choose?
\vfill
Recall that we want $\Var \bm{\alpha}_1^T\bx=\bm{\alpha}_1^T\Sigma\bm{\alpha}_1$ maximal
\vfill
We have
\[
\Var \bm{\alpha}_1^T\bx 
= \bm{\alpha}_1^T\Sigma\bm{\alpha}_1 
= \bm{\alpha}_1^T(\Sigma\bm{\alpha}_1) 
= \bm{\alpha}_1^T(\lambda\bm{\alpha}_1) 
= \lambda(\bm{\alpha}_1^T\bm{\alpha}_1) = \lambda
\]
\vfill
$\implies$ we pick $\lambda=\lambda_1$, the largest eigenvalue (covariance matrix symmetric so eigenvalues real)
\end{frame}


\begin{frame}{What we have this far..}
The first principal component is $\bm{\alpha}_1^T\bx$ and has variance $\lambda_1$, where $\lambda_1$ the largest eigenvalue of $\Sigma$ and $\bm{\alpha}_1$ an associated eigenvector with $\|\bm{\alpha}_1\|=1$
\vfill
We want the second principal component to be \emph{uncorrelated} with $\bm{\alpha}_1^T\bx$ and to have maximum variance $\Var \bm{\alpha}_2^T\bx=\bm{\alpha}_2^T\Sigma\bm{\alpha}_2$, under the constraint that $\|\bm{\alpha}_2\|=1$
\vfill
$\bm{\alpha}_2^T\bx$ uncorrelated to $\bm{\alpha}_1^T\bx$ if $\cov(\bm{\alpha}_1^T\bx,\bm{\alpha}_2^T\bx)=0$
\end{frame}

\begin{frame}
We have
\begin{align*}
\cov(\bm{\alpha}_1^T\bx,\bm{\alpha}_2^T\bx) &= 
\bm{\alpha}_1^T\Sigma\bm{\alpha}_2 \\
&= \bm{\alpha}_2^T\Sigma^T\bm{\alpha}_1 \\
&= \bm{\alpha}_2^T\Sigma\bm{\alpha}_1 \quad\textrm{[$\Sigma$ symmetric]} \\
&= \bm{\alpha}_2^T(\lambda_1\bm{\alpha}_1) \\
&= \lambda \bm{\alpha}_2^T\bm{\alpha}_1
\end{align*}
\vfill
So $\bm{\alpha}_2^T\bx$ uncorrelated to $\bm{\alpha}_1^T\bx$ if $\bm{\alpha}_1\perp\bm{\alpha}_2$
\vfill
This is beginning to sound a lot like Gram-Schmidt, no?
\end{frame}

\begin{frame}{In short}
Take whatever covariance matrix is available to you (known $\Sigma$ or sample $S_X$) -- assume sample from now on for simplicity
\vfill
For $i=1,\ldots,p$, the $i$th principal component is
\[
z_i = \bv_i^T\bx
\]
where $\bv_i$ eigenvector of $S_X$ associated to the $i$th largest eigenvalue $\lambda_i$
\vfill
If $\bv_i$ is normalised, then $\lambda_i=\Var z_k$
\end{frame}


\begin{frame}{Covariance matrix}
$\Sigma$ the covariance matrix of the random variable, $S_X$ the sample covariance matrix
\vfill
$X\in\M_{mp}$ the data, then the (sample) covariance matrix $S_X$ takes the form
\[
S_X = \frac{1}{n-1}X^TX
\]
where the data is centred!
\vfill
Sometimes you will see $S_X=1/(n-1)XX^T$. This is for matrices with observations in columns and variables in rows. Just remember that you want the covariance matrix to have size the number of variables, not observations, this will give you the order in which to take the product
\end{frame}

\begin{frame}{A smaller 2D example}
    \framesubtitle{Hockey players at IIHF world championships 2001-2016}
    \begin{center}
        \includegraphics[height=0.85\textheight]{FIGS_slides/slides-08-hockey_players_raw}
    \end{center}
\end{frame}

\begin{frame}{Centre the data}
    \framesubtitle{Subtract the mean (our first -- simple -- change of basis)}
    \begin{center}
        \includegraphics[height=0.85\textheight]{FIGS_slides/slides-08-hockey_players_centred}
    \end{center}
\end{frame}



%%%%%%%%%%%%%%%%%%%
%%%%%%%%%%%%%%%%%%%
%%%%%%%%%%%%%%%%%%%
%%%%%%%%%%%%%%%%%%%
\section{What are clustering and classification ?}

\begin{frame}{Clustering vs classification}
    Clustering is partitioning an unlabelled dataset into groups of similar objects
    \vfill
    Classification sorts data into specific categories using a labelled dataset
\end{frame}

\begin{frame}{Clustering}
    From \href{https://en.wikipedia.org/wiki/Cluster_analysis}{Wikipedia}
    \begin{quote}
        \textbf{Cluster analysis} or \textbf{clustering} is the task of grouping a set of objects in such a way that objects in the same group (called a \textbf{cluster}) are more similar (in some sense) to each other than to those in other groups (clusters).
    \end{quote}
    \vfill
    There are a myriad of ways to do clustering, this is an extremely active field of research and application. See the Wikipedia page for leads
\end{frame}


\begin{frame}{Classification}
    From \href{https://en.wikipedia.org/wiki/Statistical_classification}{Wikipedia}
    \begin{quote}
        In statistics, \textbf{classification} is the problem of identifying which of a set of categories (sub-populations) an observation (or observations) belongs to. Examples are assigning a given email to the "spam" or "non-spam" class, and assigning a diagnosis to a given patient based on observed characteristics of the patient (sex, blood pressure, presence or absence of certain symptoms, etc.).
    \end{quote}
\end{frame}

%%%%%%%%%%%%%%%%%%%
%%%%%%%%%%%%%%%%%%%
%%%%%%%%%%%%%%%%%%%
%%%%%%%%%%%%%%%%%%%
\section{Support vector machines}

\begin{frame}{Support vector machines (SVM)}
    We are given a training dataset of $n$ points of the form
    \[ 
        (\bx_1, y_1), \ldots, (\bx_n, y_n)
    \]
    where $\bx_i\in\IR^p$ and $y_i=\{-1,1\}$. The value of $y_i$ indicates the class to which the point $\bx_i $ belongs
    \vfill
    We want to find a \textbf{surface} $\S$ in $\IR^p$ that divides the group of points into two subgroups
    \vfill
    Once we have this surface $\S$, any additional point that is added to the set can then be \emph{classified} as belonging to either one of the sets depending on where it is with respect to the surface $\S$
\end{frame}

\begin{frame}{Linear SVM}
    We are given a training dataset of $n$ points of the form
    \[ 
        (\bx_1, y_1), \ldots, (\bx_n, y_n)
    \]
    where $\bx_i\in\IR^p$ and $y_i=\{-1,1\}$. The value of $y_i$ indicates the class to which the point $\bx_i $ belongs
    \vfill
    \begin{quote}\textbf{Linear SVM --}
        Find the ``maximum-margin hyperplane'' that divides the group of points $\bx_i$ for which $y_i = 1$ from the group of points for which $y_i = -1$, which is such that the distance between the hyperplane and the nearest point $\bx_i$ from either group is maximized.
    \end{quote}
\end{frame}

\begin{frame}
    \begin{minipage}{0.7\textwidth}
        \includegraphics[height=\textheight]{FIGS_slides/SVM_margin}
    \end{minipage}
    \begin{minipage}{0.28\textwidth}
        Maximum-margin hyperplane and margins for an SVM trained with samples from two classes. Samples on the margin are the \textbf{support vectors}
    \end{minipage}
\end{frame}

\begin{frame}
    Any \textbf{hyperplane} can be written as the set of points $\mathbf{x}$ satisfying
    \[
        \bw^\mathsf{T} \bx - b = 0
    \]
    where $\bw$ is the (not necessarily normalized) \textbf{normal vector} to the hyperplane (if the hyperplane has equation $a_1z_1+\cdots+a_pz_p=c$, then $(a_1,\ldots,a_n)$ is normal to the hyperplane)
    % \vfill
    % This is much like \textbf{Hesse normal form}, except that $\mathbf{w}$ is not necessarily a unit vector
    \vfill
    The parameter $b/\|\bw\|$ determines the offset of the hyperplane from the origin along the normal vector $\bw$
    \vfill
    Remark: a hyperplane defined thusly is not a subspace of $\IR^p$ unless $b=0$. We can of course transform the data so that it is...
\end{frame}

\begin{frame}{Linearly separable points}
    Let $X_1$ and $X_2$ be two sets of points in $\IR^p$ 
    \vfill
    Then $X_1$ and $X_2$ are \textbf{linearly separable} if there exist $w_{1}, w_{2},..,w_{p}, k\in\IR$ such that 
    \begin{itemize}
        \item every point $x \in X_1$ satisfies $\sum^{p}_{i=1} w_{i}x_{i} > k$ 
        \item every point $x \in X_2$ satisfies $\sum^{p}_{i=1} w_{i}x_{i} < k$
    \end{itemize}
    where $x_{i}$ is the $i$th component of $x$
\end{frame}

\begin{frame}{Hard-margin SVM}
    If the training data is \textbf{linearly separable}, we can select two parallel hyperplanes that separate the two classes of data, so that the distance between them is as large as possible 
    \vfill
    The region bounded by these two hyperplanes is called the ``margin'', and the maximum-margin hyperplane is the hyperplane that lies halfway between them
    \vfill
    With a normalized or standardized dataset, these hyperplanes can be described by the equations
    \begin{itemize}
        \item $\mathbf{w}^\mathsf{T} \mathbf{x} - b = 1$ (anything on or above this boundary is of one class, with label 1) 
        \item $\mathbf{w}^\mathsf{T} \mathbf{x} - b = -1$ (anything on or below this boundary is of the other class, with label -1)
    \end{itemize}
\end{frame}

\begin{frame}
    Distance between these two hyperplanes is $2/\|\bw\|$
    \vfill
    $\Rightarrow$ to maximize the distance between the planes we want to minimize $\|\bw\|$
    \vfill
    The distance is computed using the distance from a point to a plane equation
    \vfill
    We must also prevent data points from falling into the margin, so we add the following constraint: for each $i$ either
    \[
        \mathbf{w}^\mathsf{T} \mathbf{x}_i - b \ge 1 \, , \text{ if } y_i = 1
    \]
    or
    \[
        \mathbf{w}^\mathsf{T} \mathbf{x}_i - b \le -1 \, , \text{ if } y_i = -1
    \]
    \vfill
    (Each data point must lie on the correct side of the margin)
\end{frame}

\begin{frame}
    This can be rewritten as
    \[
        y_i(\mathbf{w}^\mathsf{T} \mathbf{x}_i - b) \ge 1, \quad \text{ for all } 1 \le i \le n
    \]
    or
    \[
        y_i(\mathbf{w}^\mathsf{T} \mathbf{x}_i - b)-1\geq 0, \quad \text{ for all } 1 \le i \le n
    \]
    \vfill
    We get the optimization problem:
    \begin{quote}
        Minimize $\|\mathbf{w}\|$ subject to $y_i(\mathbf{w}^\mathsf{T} \mathbf{x}_i - b)-1 \ge 0$ for $i = 1, \ldots, n$
    \end{quote}
    \vfill
    The $\mathbf{w}$ and $b$ that solve this problem determine the classifier, $\mathbf{x} \mapsto \sgn(\mathbf{w}^\mathsf{T} \mathbf{x} - b)$ where $\sgn(\cdot)$ is the \textbf{sign function}.
\end{frame}

\begin{frame}   
    The maximum-margin hyperplane is completely determined by those $\bx_i$ that lie nearest to it
    \vfill
    These $\bx_i$ are the \textbf{support vectors}
\end{frame}

\begin{frame}{Writing the goal in terms of Lagrange multipliers}
    Recall that our goal is to
    \begin{quote}
        minimize $\|\mathbf{w}\|$ subject to $y_i(\mathbf{w}^\mathsf{T} \mathbf{x}_i - b)-1 \ge 0$ for $i = 1, \ldots, n$
    \end{quote}
    \vfill
    Using Lagrange multipliers $\lambda_1,\ldots,\lambda_n$, we have the function
    \[
        L_P:=F(\bw,b\lambda_1,\ldots,\lambda_n) =
        \frac 12\|\bw\|^2 -\sum_{i=1}^n \lambda_iy_i(\bx_i\bw+b)
        +\sum_{i=1}^n\lambda_i
    \]
    \vfill
    Note that we have as many Lagrange multipliers as there are data points. Indeed, there are that many inequalities that must be satisfied
    \vfill 
    The aim is to minimise $L_p$ with respect to $\bw$ and $b$ while the derivatives of $L_p$ w.r.t. $\lambda_i$ vanish and the $\lambda_i\geq 0$, $i=1,\ldots,n$
\end{frame}

\begin{frame}{Lagrange multipliers}
    We have already seen Lagrange multipliers, when we were studying PCA
    \vfill
\end{frame}

\begin{frame}{Maximisation using Lagrange multipliers (V1.0)}
    We want the max of $f(x_1,\ldots,x_n)$ under the constraint $g(x_1,\ldots,x_n)=k$
    \begin{enumerate}
    \item Solve
    \begin{align*}
    \nabla f(x_1,\ldots,x_n) &= \lambda\nabla g(x_1,\ldots,x_n) \\
    g(x_1,\ldots,x_n) &= k
    \end{align*}
    where $\nabla=(\frac{\partial}{\partial x_1},\ldots,\frac{\partial}{\partial x_n})$ is the \textbf{gradient operator}
    \item Plug all solutions into $f(x_1,\ldots,x_n)$ and find maximum values (provided values exist and $\nabla g\neq \b0$ there)
    \end{enumerate}
    \vfill
    $\lambda$ is the \textbf{Lagrange multiplier}
\end{frame}
    
    
\begin{frame}{The gradient}
    $f:\IR^n\to\IR$ function of several variables, $\nabla=\left(\frac{\partial}{\partial x_1},\ldots,\frac{\partial}{\partial x_n}\right)$ the gradient operator
    \vfill
    Then
    \[
    \nabla f = \left(
    \frac{\partial}{\partial x_1}f,\ldots,
    \frac{\partial}{\partial x_n}f
    \right)
    \]
    \vfill
    So $\nabla f$ is a \emph{vector-valued} function, $\nabla f:\IR^n\to\IR^n$; also written as
    \[
    \nabla f = f_{x_1}(x_1,\ldots,x_n)\be_1+\cdots f_{x_n}(x_1,\ldots,x_n)\be_n
    \]
    where $f_{x_i}$ is the partial derivative of $f$ with respect to $x_i$ and $\{\be_1,\ldots,\be_n\}$ is the standard basis of $\IR^n$
\end{frame}

\begin{frame}{Lagrange multipliers (V2.0)}
        However, the problem we were considering then involved a single multiplier $\lambda$
        \vfill
        Here we want $\lambda_1,\ldots,\lambda_n$
\end{frame}

\begin{frame}{Lagrange multiplier theorem}
    \begin{theorem}
        Let $f\colon\mathbb{R}^n \rightarrow \mathbb{R}$ be the objective function, $g\colon\mathbb{R}^n \rightarrow \mathbb{R}^c $ be the constraints function, both being $C^1$.
        Consider the optimisation problem
        \begin{align*}
            \text{maximize}\ f(x) \\
            \text{subject to}\ g(x) = 0                 
        \end{align*}
        Let $x^*$ be an optimal solution to the optimization problem, such that $\operatorname{rank} (Dg(x^*)) = c < n$, where $Dg(x^*)$ denotes the matrix of partial derivatives
        \[
            \left[{\partial g_j}/{\partial x_k}\right]  
        \]
        Then there exists a unique Lagrange multiplier $\lambda^* \in \mathbb{R}^c$ such that
        \[
            Df(x^*) = \lambda^{*T}Dg(x^*)
        \]
    \end{theorem}
\end{frame}

\begin{frame}{Lagrange multipliers (V3.0)}
    Here we want $\lambda_1,\ldots,\lambda_n$
    \vfill
    But we also are looking for $\lambda_i\geq 0$
    \vfill 
    So we need to consider the so-called Karush-Kuhn-Tucker (KKT) conditions
\end{frame}

\begin{frame}{Karush-Kuhn-Tucker (KKT) conditions}
    Consider the optimisation problem
    \begin{align*}
        \text{maximize}\ f(x) \\
        \text{subject to}& \quad g_i(x) \leq 0  \\
        &\quad h_i(x)=0               
    \end{align*}
    Form the Lagrangian
    \[
        L(\bx,\mu,\lambda) = f(\bx)+\mu^T\bg(\bx)+\lambda^T\bh(\bx)
    \]
    \begin{theorem}
        If $(\mathbf{x}^{\ast},\mathbf{\mu}^\ast)$ is a \emph{saddle point} of $L(\mathbf{x},\mathbf{\mu})$ in $\mathbf{x} \in \mathbf{X}$, $\mathbf{\mu} \geq \mathbf{0}$, then $\mathbf{x}^{\ast}$ is an optimal vector for the above optimization problem. Suppose that $f(\mathbf{x})$ and $g_i(\mathbf{x})$, $i = 1, \ldots, m$, are \emph{convex} in $\mathbf{x}$ and that there exists $\mathbf{x}_{0} \in \mathbf{X}$ such that $\mathbf{g}(\mathbf{x}_{0}) < 0$. Then with an optimal vector $\mathbf{x}^{\ast}$ for the above optimization problem there is associated a non-negative vector $\mathbf{\mu}^\ast$ such that $L(\mathbf{x}^{\ast},\mathbf{\mu}^\ast)$ is a saddle point of $L(\mathbf{x},\mathbf{\mu})$
    \end{theorem}
\end{frame}

\begin{frame}{KKT conditions}
    \begin{align*}
        \frac{\partial}{\partial w_\nu}L_P &=
        w_\nu-\sum_{i}^n\lambda_iy_ix_{i\nu}=0
        \qquad\nu=1,\ldots,p \\
        \frac{\partial}{\partial b}L_P &=
        -\sum_{i=1}^n \lambda_iy_i = 0 \\
        y_i(\bx_i^T\bw+b)-1 &\geq 0\qquad i=1,\ldots,n \\
        \lambda_i &\geq 0\qquad i=1,\ldots,n \\
        \lambda_i(y_i(\bx_i^T\bw+b)-1) &=0\qquad i=1,\ldots,n \\
    \end{align*}
\end{frame}


    
\begin{frame}{Soft-margin SVM}
    To extend SVM to cases in which the data are not linearly separable, the \textbf{hinge loss} function is helpful
    \[
        \max\left(0, 1 - y_i(\mathbf{w}^\mathsf{T} \mathbf{x}_i - b)\right)
    \]
    \vfill
    $y_i$ is the $i$th target (i.e., in this case, 1 or -1), and $\mathbf{w}^\mathsf{T} \mathbf{x}_i - b$ is the $i$-th output
    \vfill
    This function is zero if the constraint is satisfied, in other words, if $\mathbf{x}_i$ lies on the correct side of the margin
    \vfill 
    For data on the wrong side of the margin, the function's value is proportional to the distance from the margin
\end{frame}

\begin{frame}
   
    The goal of the optimization then is to minimize
    
    \[ 
        \lambda \lVert \mathbf{w} \rVert^2 +\left[\frac 1 n \sum_{i=1}^n \max\left(0, 1 - y_i(\mathbf{w}^\mathsf{T} \mathbf{x}_i - b)\right) \right]
    \]
    
    where the parameter $\lambda > 0$ determines the trade-off between increasing the margin size and ensuring that the $\mathbf{x}_i$ lie on the correct side of the margin
    \vfill
    Thus, for sufficiently small values of $\lambda$, it will behave similar to the hard-margin SVM, if the input data are linearly classifiable, but will still learn if a classification rule is viable or not
\end{frame}

\end{document}
