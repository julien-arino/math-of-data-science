\PassOptionsToPackage{landscape}{geometry}
\documentclass[aspectratio=43]{beamer}

\usepackage{bm}

\usepackage{xcolor}
\usepackage{colortbl}
\def\boldred#1{\color{red}\textbf{#1}}

\def\IC{\mathbb{C}}
\def\IF{\mathbb{F}}
\def\II{\mathbb{I}}
\def\IM{\mathbb{M}}
\def\IN{\mathbb{N}}
\def\IP{\mathbb{P}}
\def\IR{\mathbb{R}}
\def\IZ{\mathbb{Z}}

\def\ba{\mathbf{a}}
\def\bb{\mathbf{b}}
\def\bc{\mathbf{c}}
\def\be{\mathbf{e}}
\def\bh{\mathbf{h}}
\def\bi{\mathbf{i}}
\def\bj{\mathbf{j}}
\def\bk{\mathbf{k}}
\def\bn{\mathbf{n}}
\def\bp{\mathbf{p}}
\def\br{\mathbf{r}}
\def\bs{\mathbf{s}}
\def\bu{\mathbf{u}}
\def\bv{\mathbf{v}}
\def\bw{\mathbf{w}}
\def\bx{\mathbf{x}}
\def\by{\mathbf{y}}
\def\bz{\mathbf{z}}

\def\bB{\mathbf{B}}
\def\bD{\mathbf{D}}
\def\bF{\mathbf{F}}
\def\bG{\mathbf{G}}
\def\bN{\mathbf{N}}
\def\bR{\mathbf{R}}
\def\bS{\mathbf{S}}
\def\bT{\mathbf{T}}
\def\b0{\mathbf{0}}

\bmdefine{\bmu}{\bm{\mu}}

\def\A{\mathcal{A}}
\def\B{\mathcal{B}}
\def\C{\mathcal{C}}
\def\D{\mathcal{D}}
\def\E{\mathcal{E}}
\def\F{\mathcal{F}}
\def\G{\mathcal{G}}
\def\I{\mathcal{I}}
\def\L{\mathcal{L}}
\def\M{\mathcal{M}}
\def\P{\mathcal{P}}
\def\R{\mathcal{R}}
\def\S{\mathcal{S}}
\def\T{\mathcal{T}}
\def\U{\mathcal{U}}
\def\V{\mathcal{V}}

\def\nbOne{{\mathchoice {\rm 1\mskip-4mu l} {\rm 1\mskip-4mu l}
{\rm 1\mskip-4.5mu l} {\rm 1\mskip-5mu l}}}

\def\cov{\ensuremath{\mathsf{cov}}}
\def\Var{\ensuremath{\mathsf{Var}\ }}

\newtheorem{proposition}[theorem]{Proposition}
\newtheorem{property}[theorem]{Property}
\newtheorem{importantproperty}[theorem]{Property}
\newtheorem{importanttheorem}[theorem]{Theorem}
%\newtheorem{lemma}[theorem]{Lemma}



\setbeamertemplate{navigation symbols}{}
\setbeamertemplate{footline}
{%
	\quad\insertsection\hfill p. \insertpagenumber\quad\mbox{}\vskip2pt
}
\usecolortheme{orchid}
\setbeamertemplate{theorems}[numbered]

\makeatletter
\newlength\beamerleftmargin
\setlength\beamerleftmargin{\Gm@lmargin}
\makeatother

%%%%%%% 
%% Definitions in yellow boxes
\usepackage{etoolbox}
\setbeamercolor{block title}{use=structure,fg=structure.fg,bg=structure.fg!05!bg}
\setbeamercolor{block body}{parent=normal text,use=block title,bg=block title.bg!20!bg}

\BeforeBeginEnvironment{definition}{%
	\setbeamercolor{block title}{fg=black,bg=yellow!20!white}
	\setbeamercolor{block body}{fg=black, bg=yellow!05!white}
}
\AfterEndEnvironment{definition}{
	\setbeamercolor{block title}{use=structure,fg=structure.fg,bg=structure.fg!20!bg}
	\setbeamercolor{block body}{parent=normal text,use=block title,bg=block title.bg!50!bg, fg=black}
}
\BeforeBeginEnvironment{importanttheorem}{%
	\setbeamercolor{block title}{fg=black,bg=red!20!white}
	\setbeamercolor{block body}{fg=black, bg=red!05!white}
}
\AfterEndEnvironment{importanttheorem}{
	\setbeamercolor{block title}{use=structure,fg=structure.fg,bg=structure.fg!20!bg}
	\setbeamercolor{block body}{parent=normal text,use=block title,bg=block title.bg!50!bg, fg=black}
}
\BeforeBeginEnvironment{importantproperty}{%
	\setbeamercolor{block title}{fg=black,bg=red!50!white}
	\setbeamercolor{block body}{fg=black, bg=red!30!white}
}
\AfterEndEnvironment{importantproperty}{
	\setbeamercolor{block title}{use=structure,fg=structure.fg,bg=structure.fg!20!bg}
	\setbeamercolor{block body}{parent=normal text,use=block title,bg=block title.bg!50!bg, fg=black}
}


%%%%%%%%%%%%%%%%%
\usepackage{tikz}
\usetikzlibrary{shapes,arrows}
\usetikzlibrary{positioning}
\tikzstyle{cloud} = [draw, ellipse,fill=red!20, node distance=0.87cm,
minimum height=2em]
\tikzstyle{line} = [draw, -latex']
\usetikzlibrary{shapes.symbols,shapes.callouts,patterns}
\usetikzlibrary{calc,fit}
\usetikzlibrary{backgrounds}

\usetikzlibrary{decorations.pathmorphing,backgrounds,positioning,fit,petri}
\usetikzlibrary{automata}


\title{Graphs -- Introduction (theory)}
\date{}

\begin{document}
\DeclareFontShape{OT1}{cmss}{b}{n}{<->ssub * cmss/bx/n}{} 
\begin{frame}
	\titlepage
\end{frame}


%%%%%%%%%%%%%%%%%%%%%
%%%%%%%%%%%%%%%%%%%%%
%%%%%%%%%%%%%%%%%%%%%
%%%%%%%%%%%%%%%%%%%%%


\begin{frame}{Graphs versus networks}
Mostly a terminology difference:
\begin{itemize}
\item graphs in the mathematical world
\item networks elsewhere
\end{itemize}
\vfill
I will mostly say \emph{graphs} (this is a math course) but might oscillate
\vfill
I will be basing the theory part on Berge (Graphs and Hypergraphs, 1973) and Bang-Jensen \& Gutin (Digraphs: Theory, Algorithms and Applications, 2009)
\vfill
Graph theory's vocabulary is not homogeneous. As much as possible, I will point out alternate terms
\vfill
\end{frame}


\begin{frame}{Graph}
Intuitively: a graph is a set of points, and a set of arrows, with each arrow joining one point to another
\vfill
The points are called the \emph{vertices} of the graph and the arrows are the \emph{arcs} of the graph
\end{frame}

\begin{frame}
\begin{definition}[Graph]
A graph $G$ is a pair $(X,U)$, where
\begin{enumerate}
\item $X$ a set $\{x_1,x_2,\ldots,x_n\}$ of elements called \textbf{vertices}
\item $U$ a family $(u_1,u_2,\ldots,u_m)$ of elements of the Cartesian product $X\times X$, called \textbf{arcs}. This family is often be denoted by the set $U=\{1,2,\ldots,m\}$ of its indices.
An element $(x,y)$ of $X\times X$ can appear more than once in this family.
A graph in which no element of $X\times X$ appears more than $p$ times is called a $p$-graph
\end{enumerate}
\end{definition}
\vfill
\begin{definition}[Order of a graph]
The number of vertices in a graph is called the \textbf{order} of the graph
\end{definition}
\end{frame}

\begin{frame}
\begin{definition}[Loop]
An arc of $G$ of the form $(x,x)$ is called a \textbf{loop}
\end{definition}
\vfill
\begin{definition}[Arc endpoints]
For an arc $u=(x,y)$, vertex $x$ is the \textbf{initial endpoint}, and vertex $y$ is the \textbf{terminal endpoint}
\end{definition}
\vfill
\begin{definition}[Successors/Predecessors]
Vertex $y$ is a \textbf{successor} (resp. \textbf{predecessor}) of vertex $x$ if there is an arc with $y$ as its initial (resp. terminal) endpoint and $x$ as its termina1 (resp. initial) endpoint. The set of all successors (resp. predecessors) of $x$ is denoted $\Gamma^+_G(x)$ (resp. $\Gamma^-_G(x)$)
\end{definition}

\begin{definition}
The set of all neighbours of $x$ is denoted $\Gamma(x) = \Gamma^+_G(x)\cup\Gamma^-_G(x)$
\end{definition}
\end{frame}

\begin{frame}{Examples}
	\begin{figure}[htbp]
		\centering
		\def\skip{2.75cm}
		\begin{tikzpicture}[scale=0.75, 
		every node/.style={transform shape},
		auto,
		cloud/.style={minimum width={width("N-1")+2pt},
			draw, ellipse},
		connected/.style={dotted,-}]
		%% Vertices
		\node [cloud] at (0,0) (a) {$a$};
		\node [cloud] at (1*\skip,1*\skip) (b) {$b$};
		\node [cloud] at (1*\skip,-1*\skip) (c) {$c$};
		\node [cloud] at (2*\skip,0) (d) {$d$};
		%% Arcs
		\path [line, thick,bend right] (a) to node [midway, above] (TextNode) {2} (b);
		\path [line, thick] (a) to node [midway, above] (TextNode) {3} (b);
		\path [line, thick,bend left] (a) to node [midway, above] (TextNode) {4} (b);
		\path [line, thick, looseness=5] (b) to node [midway, above] (TextNode) {5} (b);
		\path [line, thick, looseness=15] (b) to node [midway, above] (TextNode) {6} (b);
		\path [line, thick] (c) to node [midway, left] (TextNode) {7} (b);
		\path [line, thick] (c) to node [midway, above] (TextNode) {1} (a);
		\path [line, thick, bend left] (d) to node [midway, above] (TextNode) {8} (c);
		\path [line, thick, bend right] (d) to node [midway, above] (TextNode) {9} (c);
		\path [line, thick, looseness=5] (d) to node [midway, above] (TextNode) {10} (d);
		\end{tikzpicture}
	\end{figure}
	\begin{figure}[htbp]
	\centering
	\def\skip{2.75cm}
	\begin{tikzpicture}[scale=0.75, 
	every node/.style={transform shape},
	auto,
	cloud/.style={minimum width={width("N-1")+2pt},
		draw, ellipse},
	connected/.style={dotted,-}]
	%% Vertices
	\node [cloud] at (0,0) (x1) {$x_1$};
	\node [cloud] at (1*\skip,1*\skip) (x2) {$x_2$};
	\node [cloud] at (2*\skip,1*\skip) (x3) {$x_3$};
	\node [cloud] at (1.5*\skip,0) (x4) {$x_4$};
	\node [cloud] at (2.5*\skip,0*\skip) (x5) {$x_5$};
	\node [cloud] at (2.5*\skip,1*\skip) (x6) {$x_6$};
	%% Arcs
	\path [line, thick,bend left] (x1) to node [midway, above] (TextNode) {1} (x2);
	\path [line, thick,bend left] (x2) to node [midway, below] (TextNode) {2} (x1);
	\path [line, thick, looseness=5] (x2) to node [midway, above] (TextNode) {3} (x2);
	\path [line, thick] (x2) to node [midway, above] (TextNode) {4} (x3);
	\path [line, thick] (x2) to node [midway, below] (TextNode) {5} (x4);
	\path [line, thick] (x4) to node [midway, below] (TextNode) {6} (x3);
	\path [line, thick, looseness=5, below] (x6) to node [midway, above] (TextNode) {7} (x6);
	\path [line, thick] (x6) to node [midway, right] (TextNode) {8} (x5);
	\end{tikzpicture}
\end{figure}
\end{frame}


\begin{frame}{Directed versus undirected graph}
\begin{definition}[Edge]
In $G=(X,U)$, each arc $u_i=(x,y)$ determines a continuous line
joining $x$ and $y$. Such a line with no specification of direction is an \textbf{edge}, and is denoted by $e_i=[x,y]$.
The family $(e_1,e_2,\ldots,e_m)$ of the edges of $G$ is denoted by its set of indices $E=\{1,2,\ldots,m\}$
\end{definition}
\vfill
Without direction, $G=(X,E)$, we talk of \textbf{undirected graph} (or \textbf{multigraph})
\vfill
An undirected graph in which there is no loops and no more than one edge joins any two vertices is \textbf{simple}
\vfill
In modern texts (e.g., B-J\&G), a \textbf{digraph} (or directed graph) has no loops
\end{frame}


\begin{frame}
\begin{definition}[Adjacent arcs/edges]
Two arcs (or two edges) are \textbf{adjacent} if they have at least one endpoint in common
\end{definition}

\begin{definition}[Multiplicity of a pair]
The \textbf{multiplicity} of a pair $x,y$ is the number $m_G^+(x,y)$ of arcs with initial endpoint $x$ and terminal endpoint $y$. Let
\begin{align*}
m_G^-(x,y) &= m_G^+(y,x) \\
m_G(x,y) &= m_G^+(x,y)+m_G^-(x,y)
\end{align*}
If $x\neq y$, then $m_G(x,y)$ is number of arcs with both $x$ and $y$ as endpoints. If $x=y$, then $m_G(x,y)$ equals twice the number of loops attached to vertex $x$. If $A,B\subset X$, $A\neq B$, let
\begin{align*}
m_G^+(A,B) &= \{u:u\in U, u=(x,y),x\in A,y\in B\} \\
m_G(A,B) &= m_G^+(A,B)+m_G^+(A,B)
\end{align*}
\end{definition}
\end{frame}


\begin{frame}
\begin{definition}[Arc incident to a vertex]
If a vertex $x$ is the initial endpoint of an arc $u$, which is not a loop, the arc $u$ is \textbf{incident out of vertex} $x$ 
\vskip0.2cm
The number of arcs incident out of $x$ plus the number of loops attached to $x$ is denoted $d_G^+(x)$ and is the \textbf{outer demi-degree} of $x$
\vskip0.2cm
An arc \textbf{incident into vertex} $x$ and the \textbf{inner demi-degree} $d_G^-(x)$ are defined similarly
\end{definition}
\vfill
\begin{definition}[Degree]
The \textbf{degree} of vertex $x$ is the number of arcs with $x$ as an endpoint, each loop being counted twice. The degree of $x$ is denoted $d_G(x)=d_G^+(x)+d_G^-(x)$
\vskip0.2cm
If each vertex has the same degree, the graph is \textbf{regular}
\end{definition}
\end{frame}


\begin{frame}
\begin{definition}[Arc incident out of $A\subset X$]
If the initial endpoint of an arc $u$ belongs to $A$, and if the terminal endpoint of arc $u$ does not belong to $A$, then $u$ is said to be incident out of $A$, and we write $u\in\omega^+(A)$
\vskip0.5cm
Similarly, we define an arc incident into $A$, and the set $\omega^-(A)$
\vskip0.5cm
Finally, the set of arcs incident to $A$ is denoted
\[
\omega(A) = \omega^+(A)\cup\omega^-(A)
\]
\end{definition}
\end{frame}


\begin{frame}
\begin{definition}[Symmetric graph]
If $m_G^+(x,y)=m_G^-(x,y)$ for all $x,y\in X$, the graph $G$ is
\textbf{symmetric}. A 1-graph $G=(X,U)$ is symmetric if, and only if,
\[
(x,y)\in U\implies (y,x)\in U
\]
\end{definition}
\vfill
\begin{definition}[Anti-symmetric graph]
If for each pair $(x,y)\in X\times X$,
\[
m_G^+(x,y)+m_G^-(x,y)\leq 1
\]
then the graph $G$ is \textbf{anti-symmetric}. A 1-graph $G=(X,U)$ is anti-symmetric if, and only if,
\[
(x,y)\in U\implies (y,x)\not\in U
\]
An anti-symmetric 1-graph without its direction is a simple graph
\end{definition}
\end{frame}


\begin{frame}
\begin{definition}[Complete graph]
A graph $G$ is \textbf{complete} if
\[
m_G(x,y)=m_G^+(x,y)+m_G^-(x,y)\geq 1
\]
for all $x,y\in X$ such that $x\neq y$
\vskip0.2cm
A 1-graph is complete if, and only if,
\[
(x,y)\not\in U\implies (y,x)\in U
\]
\end{definition}
\vfill
\begin{definition}[$n$-clique]
A simple, complete graph on $n$ vertices is called an $n$-\textbf{clique} and is often denoted $K_n$
\end{definition}
\end{frame}

\begin{frame}
\begin{definition}[Bipartite graph]
A graph is \textbf{bipartite} if its vertices can be partitioned into two sets $X_1$ and $X_2$, s.t. no two vertices in the same set are adjacent.
This graph may be written $G=(X_1,X_2,U)$
\end{definition}
\vfill
\begin{definition}[Complete bipartite graph]
If $\forall x_1\in X_1, \forall x_2\in X$, we have $m_G(x_1,x_2)\geq 1$, then $G=(X_1,X_2,U)$ is a \textbf{complete bipartite} graph
\vskip0.5cm
A simple, complete bipartite graph with $|X_1|=p$ and $|X_2|=q$ is often denoted $K_{p,q}$
\end{definition}
\end{frame}

\begin{frame}
\begin{definition}[Subgraph of $G$ generated by $A\subset X$]
The \textbf{subgraph} of $G$ generated by $A$ is the graph with $A$ as its vertex set and with all the arcs in $G$ that have both their endpoints in $A$. If $G=(X,\Gamma)$ is a 1-graph, then the subgraph generated by $A$ is the 1-graph $G_A=(A,\Gamma_A)$ where
\[
\Gamma_A(x)=\Gamma(x)\cap A\qquad (x\in A)
\]
\end{definition}
\vfill
\begin{definition}[Partial graph of $G$ generated by $V\subset U$]
\label{def:partial_graph}
The graph $(X,V)$ whose vertex set is $X$ and whose arc set is $V$. 
In other words, it is graph $G$ without the arcs $U-V$
\end{definition}
\vfill
\begin{definition}[Partial subgraph of $G$]
A partial subgraph of $G$ is the subgraph of a partial graph of $G$
\end{definition}
\end{frame}


\begin{frame}{Chain}
\begin{definition}[Chain of length $q>0$]
A \textbf{chain} is a sequence $\mu=(u_1,\ldots, u_q)$ of arcs of $G$ s.t. each arc in the sequence has one endpoint in common with its predecessor in the sequence and its other endpoint in common with its successor in the sequence
\vskip1cm 
The number of arcs in the sequence is the \textbf{length} of chain $\mu$
\vskip1cm
A chain that does not encounter the same vertex twice is \textbf{elementary}
\vskip1cm
A chain that does not use the same arc twice is \textbf{simple}
\end{definition}
\end{frame}


\begin{frame}{Path}
\begin{definition}[Path of length $q>0$]\label{def:path_length_q}
A \textbf{path} of length $q$ is a chain $\mu=(u_1, \ldots, u_i, \ldots, u_q)$
in which the terminal endpoint of arc $u_i$ is the initial endpoint of arc $u_{i+1}$ for all $i < q$. 
For a 1-graph, a path is completely determined by the sequence of vertices $x_1,x_2,\ldots$ that it encounters. Hence, we often write
\begin{align*}
\mu &=
((x_1,x_2), (x_2, x_3), \ldots) \\
&= [x_1, x_2, \ldots, x_k ,x_{k+1}] \\
&= \mu[x_1,x_{k+1}]
\end{align*}
Vertices $x_1$ and $x_k$ are the \textbf{initial} and \textbf{terminal endpoints} of path $\mu$.
Similarly, for a simple graph, a chain $\mu$ with endpoints $x$ and $y$ is determined by the sequence of its vertices, and we may write
\[
\mu=\mu[x,y] = [x,x_1,\ldots,y]
\]
\end{definition}
\end{frame}

\begin{frame}
\begin{definition}[Cycle]
A \textbf{cycle} is a chain such that
\begin{enumerate}
\item no arc appears twice in the sequence
\item the two endpoints of the chain are the same vertex
\end{enumerate}
\end{definition}
\vfill
\begin{definition}[Pseudo-cycle]
A \textbf{pseudo-cycle} is a chain $\mu=(u_1, u_2,\ldots,u_q)$ whose two endpoints are the same vertex and whose arcs are not necessarily distinct
\end{definition}
\vfill
\begin{definition}[Circuit]
A \textbf{circuit} is a cycle $\mu=(u_1, u_2,\ldots,u_q)$ such that for all $i<q$, the terminal endpoint of $u_i$ is the initial endpoint of $u_{i+1}$
\end{definition}
\end{frame}

\begin{frame}
\begin{definition}[Connected graph]
A \textbf{connected graph} is a graph that contains a chain $\mu[x,y]$ for each pair $x,y$ of distinct vertices
\end{definition}
\vfill
Denote $x\equiv y$ the relation  ``$x=y$, or $x\neq y$ and there exists a chain in $G$ connecting $x$ and $y$''. $\equiv$ is an equivalence relation since
\begin{enumerate}
	\item $x\equiv y$ \hfill[reflexivity]
	\item $x\equiv y\implies y\equiv x$ \hfill[symmetry]
	\item $x\equiv y, y\equiv z\implies x\equiv z$ \hfill[transitivity]
\end{enumerate}
\begin{definition}[Connected component of a graph]
The classes of the equivalence relation $\equiv$ partition $X$ into connected sub-graphs of $G$ called \textbf{connected components}
\end{definition}
\end{frame}

\begin{frame} 
\begin{definition}[Articulation set]
For a connected graph, a set $A$ of vertices is called an \textbf{articulation set} (or a \textbf{cutset}) if the subgraph of $G$ generated by $X-A$ is not connected
\end{definition}
\vfill
\begin{definition}[Stable set]
A set $S$ of vertices is called a stable set if no arc joins two distinct vertices in $S$
\end{definition}
\vfill
\begin{definition}[Matrix associated with $G$]
If $G$ has vertices $x_1,x_2,\ldots,x_n$, then the \textbf{matrix associated} with $G$ is 
\[
a_{ij}=m_G^+(x_i,x_j)
\]
\end{definition}
\vfill
\begin{definition}[Adjacency matrix]
The matrix $a_{ij}+a_{ji}$ is the \textbf{adjacency matrix} associated with $G$
\end{definition}
\end{frame}



\begin{frame}{More on cycles}
In a graph $G=(X,U)$, a \textbf{cycle} is a sequence of arcs
\[
\mu=(u_1,u_2,\ldots,u_q)
\]
such that
\begin{enumerate}
\item each arc $u_k$, where $1 < k < q$, has one endpoint in common with the
preceding arc $u_{k-1}$, and the other end point in common with the succeeding
arc $u_{k+1}$ (i.e., this sequence is a chain)
\item the sequence does not use the same arc twice
\item the initial vertex and terminal vertex of the chain are the same
\end{enumerate}
An \textbf{elementary cycle} is a cycle in which, in addition,
\begin{enumerate}\setcounter{enumi}{3}
\item no vertex is encountered more than once (except, of course, the initial
vertex which is also the terminal vertex)
\end{enumerate}
\end{frame}

\begin{frame}{Example}
	\begin{figure}[htbp]
		\centering
		\def\skip{2.75cm}
		\begin{tikzpicture}[scale=0.75, 
		every node/.style={transform shape},
		auto,
		cloud/.style={minimum width={width("N-1")+2pt},
			draw, ellipse},
		connected/.style={dotted,-}]
		%% Vertices
		\node [cloud] at (0,0) (a) {$a$};
		\node [cloud] at (1*\skip,0) (b) {$b$};
		\node [cloud] at (0.5*\skip,1*\skip) (c) {$c$};
		\node [cloud] at (1.5*\skip,1*\skip) (d) {$d$};
		%% Arcs
		\path [line, thick] (a) to node [midway, above] (TextNode) {1} (b);
		\path [line, thick, bend left] (a) to node [midway, above] (TextNode) {2} (c);
		\path [line, thick] (b) to node [midway, above] (TextNode) {6} (c);
		\path [line, thick] (b) to node [midway, above] (TextNode) {4} (d);
		\path [line, thick, bend left] (c) to node [midway, above] (TextNode) {3} (a);
		\path [line, thick] (c) to node [midway, above] (TextNode) {5} (d);
		\end{tikzpicture}
	\end{figure}
	\vfill
	Elementary cycles:
	\begin{itemize}
		\item $\bmu^1=(1,6,2)=[abca]$
		\item $\bmu^2=(1,6,3)=[abca]$
		\item $\bmu^3=(2,3)=[aca]$
		\item $\bmu^4=(1,4,5,2)=[abdca]$
		\item $\bmu^5=(6,5,4)=[acdb]$
		\item $\bmu^6=(1,4,5,3)=[abdca]$
	\end{itemize}
\end{frame}


\begin{frame}
Given a cycle $\mu$, denote $\mu^+$ the set of all arcs in $\mu$ that are in the direction that the cycle is traversed and $\mu^-$ the set of all the other arcs in $\mu$
\vfill
Number the arcs in $G$ as $1,2,\ldots,m$, then the cycle $\mu$ is the vector
\[
\bm{\mu}=(\mu_1,\ldots,\mu_m)
\]
where
\[
\mu_i = \begin{cases}
0 & \textrm{if }i\not\in\mu^+\cup\mu^- \\
+1 & \textrm{if }i\in\mu^+ \\
-1 & \textrm{if }i\in\mu^-
\end{cases}
\]
\end{frame}


\begin{frame}{Cocycles}
Let $A\subset X$ be nonempty and denote $\omega^+(A)$ the set of arcs that have only their initial endpoint in $A$ and $\omega^-(A)$ the set of arcs that have only their terminal endpoint in $A$. Let
\[
\omega(A)=\omega^+(A)\cup\omega^-(A)
\]
\vfill
A \textbf{cocycle} is a nonempty set of arcs of the form $\omega(A)$, partitioned into two sets $\omega^+(A)$ and $\omega^-(A)$
\vfill
An \textbf{elementary cocycle} is the set of arcs joining two connected subgraphs $A_1$ and $A_2$ s.t. 
\begin{itemize}
\item $A_1,A_2\neq\emptyset$
\item $A_1\cap A_2=\emptyset$
\item $A_1\cup A_2=C$, with $C$ a connected component of the graph
\end{itemize}
\end{frame}


\begin{frame}{A colouring lemma}
\begin{lemma}[Arc colouring Lemma]\label{lemma:arc_colouring_lemma}
Consider $G$ with arcs $1,\ldots,m$. Colour arc 1 black and arbitrarily colour the remaining arcs red, black or green. Then exactly one of the following holds true:
\begin{enumerate}
\item there is an elementary cycle containing arc 1 and only red and black
arcs with the property that all black arcs in the cycle have the same
direction
\item there is an elementary cocycle containing arc 1 and only green and black
arcs, with the property that all black arcs in the cocycle have the same
direction
\end{enumerate}
\end{lemma}
\end{frame}

\begin{frame}{Independent cycles and cycle bases}
Consider cycles $\bmu^1,\bmu^2,\ldots,\bmu^k$. The cycles are \textbf{independent} if
\begin{multline*}
c_1\bmu^1+c_2\bmu^2+\cdots+c_k\bmu^k=\b0 \\
\iff
c_1=c_2=\cdots=c_k=0
\end{multline*}
A \textbf{cycle basis} is an independent set $\{\bmu^1,\bmu^2,\ldots,\bmu^k\}$ of cycles such that any cycle $\bmu$ can be written as
\[
\bmu= c_1\bmu^1+c_2\bmu^2+\cdots+c_k\bmu^k
\]
for $c_1,\ldots,c_k\in\IR$
\vfill
The constant $k$ is the \textbf{cyclomatic number} of $G$, denoted $\nu(G)$
\end{frame}


\begin{frame}{Example}
\begin{figure}[htbp]
	\centering
	\def\skip{2.75cm}
	\begin{tikzpicture}[scale=0.75, 
	every node/.style={transform shape},
	auto,
	cloud/.style={minimum width={width("N-1")+2pt},
		draw, ellipse},
	connected/.style={dotted,-}]
	%% Vertices
	\node [cloud] at (0,0) (a) {$a$};
	\node [cloud] at (1*\skip,0) (b) {$b$};
	\node [cloud] at (0.5*\skip,1*\skip) (c) {$c$};
	\node [cloud] at (1.5*\skip,1*\skip) (d) {$d$};
	%% Arcs
	\path [line, thick] (a) to node [midway, above] (TextNode) {1} (b);
	\path [line, thick, bend left] (a) to node [midway, above] (TextNode) {2} (c);
	\path [line, thick] (b) to node [midway, above] (TextNode) {6} (c);
	\path [line, thick] (b) to node [midway, above] (TextNode) {4} (d);
	\path [line, thick, bend left] (c) to node [midway, above] (TextNode) {3} (a);
	\path [line, thick] (c) to node [midway, above] (TextNode) {5} (d);
	\end{tikzpicture}
\end{figure}
\vfill
Elementary cycles:
\begin{itemize}
\item $\bmu^1=(1,6,2)=[abca]$
\item $\bmu^2=(1,6,3)=[abca]$
\item $\bmu^3=(2,3)=[aca]$
\item $\bmu^4=(1,4,5,2)=[abdca]$
\item $\bmu^5=(6,5,4)=[acdb]$
\item $\bmu^6=(1,4,5,3)=[abdca]$
\end{itemize}
We have $\bmu^1-\bmu^2+\bmu^3=\b0$ 
\end{frame}

\begin{frame}{An important result}
\begin{importanttheorem}\label{th:cyclomatic_nb}
Let $G$ be a graph with $n$ vertices, $m$ arcs and $p$ connected components. Then the cyclomatic number of $G$ is
\[
\nu(G)=m-n+p
\]
\end{importanttheorem}
\end{frame}

\begin{frame}{Trees and forests}
\begin{definition}[Tree]
A \textbf{tree} is a connected graph without cycles
\end{definition}
(A tree is a special kind of 1-graph)
\vfill
\begin{definition}[Forest]
A \textbf{forest} is a graph whose connected components are trees
\end{definition}
(i.e., a forest is a graph without cycles)
\end{frame}


\begin{frame}
\begin{theorem}[Characterisation of trees]\label{th:characterisation_tree}
$H=(X,U)$ a graph of order $|X|=n>2$. TFAE and all characterise a tree (i.e., $H$ satisfying any of these is a tree):
\begin{enumerate}
\item $H$ connected and has no cycles
\item $H$ has $n-1$ arcs and no cycles
\item $H$ connected and has exactly $n-1$ arcs
\item $H$ has no cycles, and if an arc is added to $H$, exactly one cycle is created
\item $H$ connected, and if any arc is removed, the remaining graph is not
connected
\item Every pair of vertices of $H$ is connected by one and only one chain
\end{enumerate}
\end{theorem}
\end{frame}


\begin{frame}{Proof of Theorem~\ref{th:characterisation_tree}}
We make abundant use of Theorem~\ref{th:cyclomatic_nb}
\vfill
[$\mathbf{1\implies 2}$] 
Let $p$ be the number of connected components, $m$ the number of arcs and $\nu(H)$ the cyclomatic number. Since $H$ connected, $p=1$. Since $H$ has no cycles, $\nu(H)=m-n+p=0$

$\implies$ $m=n-p=n-1$
\vfill

[$\mathbf{2\implies 3}$] 
Assume $H$ has no cycles ($\nu(H)=0$) and has $n-1$ arcs ($m=n-1$). Then, since
\[
\nu(H)=m-n+p
\]
$p=\nu(H)-m+n=0-(n-1)-n=1$, i.e., $H$ is connected
\end{frame}

\begin{frame}{Proof of Theorem~\ref{th:characterisation_tree} (cont.)}
[$\mathbf{3\implies 4}$]
Assume $H$ connected ($p=1$) and contains exactly $n-1$ arcs ($m=n-1$). Then
\[
\nu(H)=m-n+p=(n-1)-n+1=0
\]
and $H$ has no cycles
\vskip0.5cm
Now add an arc, i.e., suppose $m=n$. Then $\nu(H)=m-n+p=n-n+1=1$ and there is one cycle in the new graph
\end{frame}


\begin{frame}{Proof of Theorem~\ref{th:characterisation_tree} (cont.)}
[$\mathbf{4\implies 5}$]
Assume $H$ has no cycles ($\nu(H)=0$) and that addition of an arc to $H$ creates exactly one cycle
\vskip0.5cm
Suppose $H$ not connected. Then there are two vertices, say $a$ and $b$, that are not connected and adding the arc $(a,b)$ does not create a cycle, a contradiction with ``addition of an arc to $H$ creates exactly one cycle'' 

$\implies$ $p=1$. Since $\nu(H)=0$, this implies that $m=n-1$
\vskip0.5cm
Now suppose we remove an arc. We obtain graph $H'$ with
\[
m'=n'-2\quad\text{and}\quad \nu(H')=0
\]
So
\[
p'=\nu(H')-m'+n'=2
\]
$\implies$ $H'$ not connected
\end{frame}

\begin{frame}{Proof of Theorem~\ref{th:characterisation_tree} (cont.)}
[$\mathbf{5\implies 6}$]
Assume $H$ connected and if any arc is removed, the remaining graph is not connected
\vskip0.5cm
Any vertices $a,b\in X$ are connected by a chain (since $H$ connected). That chain is unique: suppose there is a second chain connecting $a$ to $b$; then removing an arc from that chain does not disconnect the graph, since there is still the original chain connecting $a$ and $b$
\vfill
[$\mathbf{6\implies 1}$]
Assume every pair of vertices of $H$ is connected by one and only one chain

Now assume $H$ has a cycle. Then at least one pair of vertices would be connected by two distinct chains, a contradiction
\vfill
\flushright$\qed$
\end{frame}


\begin{frame}
\begin{definition}[Pendant vertex]
A vertex is \textbf{pendant} if it is adjacent to exactly one other vertex
\end{definition}
\vfill
\begin{theorem}
A tree of order $n\geq 2$ has at least two pendant vertices
\end{theorem}
\vfill
\textbf{Proof:} 
Suppose $H$ tree of order $n\geq 2$ with 0 or 1 pendant vertices
\vskip0.5cm
Consider a traveller traversing the graph edges starting from a pendant vertex (if there is one) or anywhere if there is no pendant vertex
\vskip0.5cm
If he does not allow himself to use same edge twice, he cannot go to the same vertex twice ($H$ has no cycle)
\vskip0.5cm
If he arrives at vertex $x$, he can depart $x$ using a new edge ($x$ is not a pendant vertex as there are 0 or 1 pendant vertex and if there is 1, that's where he started)
\vskip0.5cm
So the trip continues without end, which is impossible as $H$ finite $\qed$
\end{frame}

\begin{frame}
\begin{theorem}\label{th:partial_graph_tree_G_connected}
A graph $G=(X,U)$ has a partial graph that is a tree $\iff$ $G$ connected
\end{theorem}
\vfill
Recall that a partial graph is a graph generated by a subset of the arcs (Definition~\ref{def:partial_graph} slide~\pageref{def:partial_graph})
\vfill
\textbf{Proof:}
Suppose $G$ not connected. Then no partial graph of $G$ is connected $\implies$ there is no partial graph of $G$ that is a tree [we want to show $P\iff Q$, we start with $\wedge Q\implies \wedge P$ (which $\iff$ $P\implies Q$)]
\vskip1cm
Suppose $G$ connected. Look for an arc whose removal does not disconnect $G$
\begin{itemize}
\item if there is none, $G$ is a tree by Theorem~\ref{th:characterisation_tree}(5)
\item if there is, remove it and look for another one, etc. When no more arcs can be removed, the remaining graph is a tree with vertex set $X$ $\qed$
\end{itemize}
\end{frame}

\begin{frame}{Spanning tree}
The procedure in the proof of Theorem~\ref{th:partial_graph_tree_G_connected} gives a \textbf{spanning tree}
\vfill
Can also build a spanning tree as follows:
\begin{itemize}
\item Consider any arc $u_0$
\item Find arc $u_1$ that does not form a cycle with $u_0$
\item Find arc $u_2$ that does not form a cycle with $\{u_0,u_1\}$
\item Continue
\item When you cannot continue anymore, you have a spanning tree
\end{itemize}
\end{frame}


\begin{frame}{Strongly connected graphs}
Recall that paths are defined in Definition~\ref{def:path_length_q} (slide~\pageref{def:path_length_q})
\vfill
$G=(X,U)$ connected. 
A \textbf{path of length 0} is any sequence $\{x\}$ consisting of a single vertex $x\in X$
\vfill
For $x,y\in X$, let $x\equiv y$ be the relation ``there is a path $\mu_1[x,y]$ from $x$ to $y$ as well as a path $\mu_2[y,x]$ from $y$ to $x$''. This is an equivalence relation (it is reflexive, symmetric and transitive)
\vfill
\begin{definition}[Strong components]\label{def:strong_components}
Sets of the form 
\[
A(x_0)=\{x:x\in X, x\equiv x_0\}	
\]
are equivalence classes; they partition $X$ and are the \textbf{strongly connected components} of $G$	
\end{definition}
\vfill
\begin{definition}[Strongly connected graph]
\label{def:strongly_connected_graph}
$G$ \textbf{strongly connected} if it has a single strong component
\end{definition}
\end{frame}


\begin{frame}
\begin{theorem}
$G$ connected graph with $\geq 1$ arc. TFAE
\begin{enumerate}
\item $G$ strongly connected
\item Every arc lies on a circuit
\item $G$ contains no cocircuits
\end{enumerate}
\end{theorem}
\vfill
\textbf{Proof:}
[$\mathbf{1\implies 2}$]
$(x,y)$ an arc of $G$; there is a path from $y$ to $x$ ($G$ strongly connected), so arc $(x,y)$ is contained in a circuit of $G$

[$\mathbf{2\implies 3}$]
Suppose $G$ has a cocircuit containing arc $(x,y)$; then $G$ cannot have
a circuit containing this arc by the Arc Colouring Lemma (Lemma~\ref{lemma:arc_colouring_lemma} slide~\pageref{lemma:arc_colouring_lemma}) with
all arcs coloured black. This contradicts (2)

[$\mathbf{3\implies 1}$]
Assume $G$ connected graph without cocircuits, but $G$ not strongly connected.
Since $G$ not strongly connected, it has $>1$ strongly
connected component. 
Since $G$ connected, there exist two distinct strongly connected components that are joined by an arc $(a,b)$. Arc $(a,b)$ is not contained in any circuit because otherwise $a$ and $b$ would be in the same strongly connected component. 
By Lemma~\ref{lemma:arc_colouring_lemma}, arc $(a,b)$ is contained in some co-
circuit. This contradicts (3)\qed
\end{frame}


\begin{frame}
\begin{theorem}
$G$ graph with $\geq 1$ arc. TFAE
\begin{enumerate}
\item $G$ is a graph without circuits
\item Each arc is contained in a cocircuit
\end{enumerate}
\end{theorem}
\vfill
\begin{theorem}
If $G$ is a strongly connected graph of order $n$, then $G$ has a cycle basis of $\nu(G)$ circuits
\end{theorem}
\end{frame}

\begin{frame}
\begin{definition}[Node, anti-node, branch]
$G=(X,U)$ strongly connected without loops and $>1$ vertex. For each $x\in X$, there is a path from it and a path to it so $x$ has at least 2 incident arcs. Specifically,
\begin{itemize}
\item $x\in X$ with $>2$ incident arcs is a \textbf{node}
\item $x\in X$ with $2$ incident arcs is an \textbf{anti-node}
\end{itemize}
A path whose only nodes are its endpoints is a \textbf{branch}
\end{definition}
\vfill
\begin{definition}[Minimally connected graph]
$G$ is \textbf{minimally connected} if it is strongly connected and removal of any arc destroys strong-connectedness
\end{definition}
A minimally connected graph is 1-graph without loops
\vfill
\begin{definition}[Contraction]
$G=(X,U)$. The \textbf{contraction} of the set $A\subset X$ of vertices consists in replacing $A$ by a single vertex $a$ and replacing each arc into (resp. out of) $A$ by an arc with same index into (resp. out of) $a$
\end{definition}
\end{frame}


\begin{frame}
\begin{theorem}
$G$ minimally connected, $A\subset X$ generating a strongly connected subgraph of $G$. Then the contraction of $A$ gives a minimally connected graph
\end{theorem}
\vfill
\textbf{Proof:}
First, show that contraction of $A$ yields a 1-graph. 
If this were not the case, there would exist $x\not\in A$ and $a,a'\in A$ s.t. $(x,a),(x,a')\in U$ (or, with $(a,x),(a',x)\in U$ but this would not
change the proof). 
If one of these arcs is removed, the graph remains strongly
connected. Thus, $G$ is not minimally connected, a contradiction

\vskip1cm
Now show that the contraction of $A$ yields a graph $G'$ that is minimally connected. Clearly, $G'$ strongly connected. If an arc $u$ is removed, the remaining graph is not strongly connected, since the graph
$(X,U-\{u\})$ not strongly connected \qed
\end{frame}

\begin{frame} 
\begin{theorem}
$G$ a minimally connected graph, $G'$ be the minimally connected graph obtained by the contraction of an elementary circuit of $G$.	Then
\[
\nu(G)=\nu(G')+1
\]
\end{theorem}
\vfill
\begin{theorem}
$G$ minimally connected of order $n\geq 2$ $\implies$ $G$ has $\geq 2$ anti-nodes
\end{theorem}
\vfill
\begin{theorem}
$G=(X,U)$. Then the graph $C'$ obtained by contracting each strongly connected component of $G$ contains no circuits
\end{theorem}
\end{frame}


\begin{frame}{Arborescences}
\begin{definition}[Root]
Vertex $a\in X$ in $G=(X,U)$ is a \textbf{root} if all vertices of $G$ can be reached by paths \emph{starting} from $a$
\end{definition}
Not all graphs have roots
\vfill
\begin{definition}[Quasi-strong connectedness]
$G$ is \textbf{quasi-strongly connected} if $\forall x,y\in X$, exists $z\in X$ (denoted $z(x,y)$ to emphasize dependence on $x,y$) from which there is a path to $x$ and a path to $y$
\end{definition}
Strongly connected $\implies$ quasi-strongly connected (take $z(x,y)=x$); converse not true

Quasi-strongly connected $\implies$ connected
\vfill
\begin{definition}[Arborescence]
An \textbf{arborescence} is a tree that has a root
\end{definition}
\vfill
\begin{lemma}
$G=(X,U)$ has a root $\iff$ $G$ quasi-strongly connected
\end{lemma}
\end{frame}

\begin{frame}
\begin{theorem}
$H$ graph of order $n>1$. TFAE (and all characterise an arborescence)
\begin{enumerate}
\item $H$ quasi-strongly connected without cycles
\item $H$ quasi-strongly connected with $n-1$ arcs
\item $H$ tree having a root $a$
\item $\exists a\in X$ s.t. all other vertices are connected with $a$ by 1 and only 1 path from $a$
\item $H$ quasi-strongly connected and loses quasi-strong connectedness if any arc is removed
\item $H$ quasi-strongly connected and $\exists a\in X$ s.t.
\begin{align*}
d_H^-(a)&=0 \\
d_H^-(x)&=1\qquad \forall x\neq a
\end{align*}
\item $H$ has no cycles and $\exists a\in X$ s.t.
\begin{align*}
d_H^-(a)&=0 \\
d_H^-(x)&=1\qquad \forall x\neq a
\end{align*}
\end{enumerate}
\end{theorem}
\end{frame}


\begin{frame}
\begin{theorem}
$G$ has a partial graph that is an arborescence $\iff$ $G$ quasi-strongly connected
\end{theorem}
\vfill
\begin{theorem}
$G=(X,E)$ simple connected graph and $x_1\in X$. It is possible to direct all edges of $E$ so that the resulting graph $G_0=(X,U)$ has a spanning tree $H$ s.t.
\begin{enumerate}
	\item $H$ is an arborescence with root $x_1$
	\item The cycles associated with $H$ are circuits
	\item The only elementary circuits of $G_0$ are the cycles associated with $H$
\end{enumerate}
\end{theorem}
\end{frame}


\begin{frame}{Counting trees}
\begin{proposition}
$X$ a set with $n$ distinct objects, $n_1,\ldots,n_p$ nonnegative integers s.t. $n_1+\cdots+n_p=n$. The number of ways to place the $n$ objects into $p$ boxes $X_1,\ldots,X_p$ containing $n_1,\ldots,n_p$ objects respectively is 
\[
	\begin{pmatrix}
		n\\ n_1,\ldots,n_p
	\end{pmatrix}
	=
	\frac{n!}{n_1!\cdots n_p!}
\]
\end{proposition}
\vfill
\begin{proposition}[Multinomial formula]
Let $a_1,\ldots,a_p\in\IR$ be $p$ real numbers, then
\begin{multline*}
	(a_1+\cdots+a_p)^n= \\
	\sum_{n_1,\ldots,n_p\geq 0}
	\begin{pmatrix}
		n\\ n_1,\ldots,n_p
	\end{pmatrix}
	(a_1)^{n_1}\cdots(a_p)^{n_p}	
\end{multline*}
\end{proposition}
(Recall the binomial formula, where $p=2$)
\end{frame}


\begin{frame}
\begin{theorem}
Denote $T(n;d_1,\ldots,d_n)$ the number of distinct trees $H$ with vertices $x_1,\ldots,x_n$ and with degrees $d_H(x_1)=d_1,\ldots,d_H(x_n)=d_n$. Then
\[
	T(n;d_1,\ldots,d_n) = 
	\begin{pmatrix}
		n-2 \\
		d_1-1,\ldots,d_n-1
	\end{pmatrix}
\]
\end{theorem}
\vfill
\begin{theorem}
The number of different trees with vertices $x_1,\ldots,x_n$ is 
$n^{n-2}$
\end{theorem}
\vfill
There is a whole industry of similar results (as well as for arborescences), but we will stop here. The main point is that we are talking about a large number of possibilities..
\end{frame}


\begin{frame}{Matrices associated to a graph}
	There are multiple matrices associated to a graph
	\vfill
	We review some of them now
	\vfill
	We also review some of their uses
	\vfill
	The branch of graph theory that studies the properties of matrices derived from graphs and uses of these matrices in determining graph properties if \emph{spectral graph theory}
\end{frame}


\begin{frame}{Adjacency matrix}
We have already seen adjacency matrices, let us recall the definition here
\begin{definition}[Adjacency matrix]
	$G$ a 1-graph, then the \textbf{adjacency matrix} $A=[a_{ij}]$ is defined as follows
	\[
		a_{ij} = \begin{cases}
			1 & \text{if arc } (i,j)\in U\\
			0 & \text{otherwise}
		\end{cases}
	\]
\end{definition}
\vfill
We often write $A(G)$ and, reciprocally, if $A$ is an adjacency matrix, $G(A)$ the corresponding graph
\vfill
$G$ undirected $\implies$ $A(G)$ symmetric
\vfill
$A(G)$ has nonzero diagonal entries if $G$ is not simple
\end{frame}

\begin{frame}{Adjacency matrix -- Multigraph case}
\begin{definition}
	$G$ an $\ell$-graph, then the adjacency matrix $A=[a_{ij}]$ is defined as follows
	\[
		a_{ij} = \begin{cases}
			k & \text{if arc there are $k$ arcs } (i,j)\in U\\
			0 & \text{otherwise}
		\end{cases}
	\]
	with $k\leq \ell$
\end{definition}
\vfill
$G$ undirected $\implies$ $A(G)$ symmetric
\vfill
$A(G)$ has nonzero diagonal entries if $G$ is not simple
\end{frame}

\begin{frame}{Weighted adjacency matrices}
	Sometimes, adjacency matrices (typically for 1-graphs) have real entries, usually positive
	\vfill
	This means that the arcs/edges have been given a weight
\end{frame}

\begin{frame}{Spectrum of a graph}
	We will come back to this later, but for now..
	\vfill
	\begin{definition}[Spectrum of a graph]
		The \textbf{spectrum} of a graph $G$ is the spectrum (set of eigenvalues) of its associated adjacency matrix $A(G)$
	\end{definition}
	\vfill
	This is regardless of the type of adjacency matrix or graph
\end{frame}


\begin{frame}{Incidence matrix}
	\begin{definition}[Incidence matrix -- Undirected case]
		$G=(X,E)$ an undirected graph of order $n$ with $p$ edges. The \textbf{incidence} matrix of $G$ is an $n\times p$ matrix with vertices as rows and edges as columns and where $B=[b_{ij}]$ satisfies
		\[
			b_{ij} = \begin{cases}
				1 & \text{if edge }j\text{ is incident to vertex }i \\
				0 & \text{otherwise}
			\end{cases}
		\]
	\end{definition}
	\vfill
	\begin{definition}[Incidence matrix -- Directed case]
		$G=(X,U)$ a directed graph of order $n$ with $p$ arcs. The \textbf{incidence} matrix of $G$ is an $n\times p$ matrix with vertices as rows and edges as columns and where $B=[b_{ij}]$ satisfies
		\[
			b_{ij} = \begin{cases}
				1 & \text{if arc }j\text{ ``enters'' vertex }i \\
				-1 & \text{if arc }j\text{ ``leaves'' vertex }i \\
				0 & \text{otherwise}
			\end{cases}
		\]
	\end{definition}
\end{frame}



\begin{frame}{Degree matrix}
	\begin{definition}[Degree matrix]
		The \textbf{degree} matrix $D=[d_{ij}]$ for $G$ is a $n \times n$ diagonal matrix defined as
		\[
			d_{ij} = \begin{cases}
				d_G(v_i) & \text{if } i = j \\
				0 & \text{otherwise}
			\end{cases}
		\]
	\end{definition}
\vfill
In an undirected graph, this means that each loop increases the degree of a vertex by two
\vfill
In a directed graph, the term ``degree'' may refer either to indegree (the number of incoming edges at each vertex) or outdegree (the number of outgoing edges at each vertex)
\end{frame}

\begin{frame}{Laplacian matrix}
\begin{definition}
	$G=(X,U)$ a simple graph with $n$ vertices. The \textbf{Laplacian} matrix is 
	\[
		L=D(G)-A(G)
	\]
	where $D(G)$ is the degree matrix and $A(G)$ is the adjacency matrix
\end{definition}
$G$ simple graph $\implies$ $A(G)$ only contains 1 or 0 and its diagonal elements are all 0
\vfill
For directed graphs, either the indegree or outdegree is used, depending on the application
\vfill
Elements of $L$ are given by
\[
\ell_{ij}=\begin{cases}
	d_G(v_{i}) & \text{if } i=j\\
	-1 & \text{if } i\neq j \text{ and } v_{i} 
	\text{ is adjacent to }v_{j}\\
	0 & \text{otherwise}
\end{cases}
\]
\end{frame}

\begin{frame}{Counting paths}
	To count paths between vertices $x$ and $y$ in a graph, we use the adjacency matrix
	\vfill
	\begin{theorem}
		$G$ a graph and $A(G)$ its adjacency matrix. Denote $P=[p_{ij}]$ the matrix $P=A^k$. Then $p_{ij}$ is the number of distinct paths of length $k$ from $i$ to $j$ in $G$
	\end{theorem}
	\vfill
	This provides an interesting connection with linear algebra
	\begin{definition}[Irreducible matrix]
		A matrix $A\in\M_n$ is \textbf{reducible} if $\exists P\in\M_n$, permutation matrix, s.t. $P^TAP$ can be written in block triangular form. If no such $P$ exists, $A$ is \textbf{irreducible}
	\end{definition}
	\vfill
	\begin{theorem}
		$A$ irreducible $\iff$ $G(A)$ strongly connected
	\end{theorem}
\end{frame}


\begin{frame}{Perron-Frobenius theorem}
Assignment 7 had the following:
\vfill
$A=[a_{ij}]\in\M_n(\IR)$ \textbf{nonnegative} if $a_{ij}\geq 0$ $\forall i,j=1,\ldots,n$; $\bv\in\IR^n$ nonnegative if $v_i\geq 0$ $\forall i=1,\ldots,n$.  \textbf{Spectral radius} of $A$
\[
\rho(A) = \max_{\lambda\in\mathsf{Sp}(A)}\{|\lambda|\}
\]
$\mathsf{Sp}(A)$ the \textbf{spectrum} of $A$
\vfill
\begin{theorem}[PF -- Nonnegative case]
$0\leq A\in\M_n(\IR)$. Then $\exists\bv\geq\b0$ s.t. 
\[
	A\bv=\rho(A)\bv
\]
\end{theorem}
\vfill

\begin{theorem}[PF -- Irreducible case]
	\label{th:PF}
	Let $0\leq A\in \M_n(\IR)$ irreducible. Then $\exists\bv>\b0$ s.t.
	\[
		A\bv=\rho(A)\bv
	\]
	$\rho(A)>0$ and with algebraic multiplicity 1
	\vskip0.5cm
	No nonnegative eigenvector is associated to any other eigenvalue of $A$
\end{theorem}
\end{frame}



\begin{frame}{Primitive matrices}
\begin{definition}
$0\leq A\in\M_n(\IR)$ \textbf{primitive} (with \textbf{primitivity index} $k\in \IN_{+}^{*}$) if $\exists k\in \IN_{+}^{*}$ s.t.
\[
A^k>0,
\]
with $k$ the smallest integer for which this is true.
$A$ \textbf{imprimitive} if it is not primitive
\end{definition}
\vfill
$A$ primitive $\implies$ $A$ irreducible; converse false
\vfill
\begin{theorem}
	$A\in\M_n(\IR)$ irreducible and $\exists i=1,\ldots,n$ s.t. $a_{ii}>0$ $\implies$ $A$ primitive
\end{theorem}
\vfill
Here $d$ is the index of imprimitivity (i.e., the number of eigenvalues that have the same modulus as $\lambda_p=\rho(A)$). If $d=1$, then $A$ is primitive. We have that $d=\mathsf{gcd}$ of all the lengths of closed walks in $G(A)$
\end{frame}

\begin{frame}
\begin{minipage}{0.49\textwidth}
	\begin{tikzpicture}[auto, %node distance = 2cm, auto,
		cloud/.style={minimum width={width("N-1")+2pt},
		draw, ellipse}]
		\node [cloud] (S1) at (0,0) {$1$};
		\node [cloud] (S2) at (2,0) {$2$};
		\node [cloud] (S3) at (1,-2) {$3$};
		%% Recoveries
		\path [line, thick] (S1) to [bend left = 20] (S2);
		\path [line, thick] (S2) to [bend left = 20] (S3);
		\path [line, thick] (S3) to [bend left = 20] (S1);
	\end{tikzpicture}
\end{minipage}
\begin{minipage}{0.49\textwidth}
Adjacency matrix
\[
	A=\begin{pmatrix}
	0&1& 0\\
	0&0&1\\
	1&0&0
	\end{pmatrix}
\]
\end{minipage}
\vskip0.2cm
Closed walks in $G(A)$ (lengths): $1\to 1$ (3), $2\to 2$ (3), $2\to 2$ (3)
$\implies$ $\gcd = 3$ $\implies$ $d=3$ (here, all eigenvalues have modulus $1$)
\vfill
\begin{minipage}{0.49\textwidth}
\begin{tikzpicture}[auto, %node distance = 2cm, auto,
cloud/.style={minimum width={width("N-1")+2pt},
draw, ellipse}]
\node [cloud] (S1) at (0,0) {$1$};
\node [cloud] (S2) at (2,0) {$2$};
\node [cloud] (S3) at (1,-2) {$3$};
%% Loops
\path [line, thick, distance=30] (S1) to (S1);
%% Recoveries
\path [line, thick] (S1) to [bend left = 20] (S2);
\path [line, thick] (S2) to [bend left = 20] (S3);
\path [line, thick] (S3) to [bend left = 20] (S1);
\end{tikzpicture}
\end{minipage}
\begin{minipage}{0.49\textwidth}
\[
A= \begin{pmatrix}
1&1&0\\
0&0&1\\
1&0&0
\end{pmatrix}
\]
\end{minipage}
\vskip0.2cm
Closed walk $1\to 1$ has length 1 $\implies$
gcd of lengths of closed walks is 1 $\implies$ $A$ primitive
\end{frame}

\begin{frame}
\begin{theorem}
$0\leq A\in \M_n$. $A$ primitive $\implies$ $A^k>0$ for some $0<k\leq (n-1)n^n$
\end{theorem}
\vfill
\begin{theorem}
$A\geq 0$ primtive. Suppose the shortest simple directed cycle in $G(A)$ has length $s$, then primitivity index is $\leq n+s(n-1)$
\end{theorem}
\vfill
\begin{theorem}
$0\leq A\in \M_n$ primitive $\iff$ $A^{n^2-2n+2}>0$
\end{theorem}
\vfill
\begin{theorem}
	$0\leq A\in \M_n$ irreducible. $A$ has $d$ positive entries on the diagonal $\implies$ primitivity index $\leq 2n-d-1$
\end{theorem}
\end{frame}


\begin{frame}
\begin{theorem}\label{th:PF_allCases}
	$0\leq A\in\M_n$, $\lambda_P=\rho(A)$ the Perron root of $A$, $\bv_P$ and $\bw_P$ the corresponding right and left Perron vectors of $A$, respectively, $d$ the index of imprimitivity of $A$ (with $d=1$ when $A$ is primitive) and $\lambda_j\in\sigma(A)$ the spectrum of $A$, with $j=2,\ldots,n$ unless otherwise specified (assuming $\lambda_1=\lambda_P$) 
	\begin{center}
		\begin{tikzpicture}[scale=0.75, every node/.style={transform shape},
			auto,
		cloud/.style={minimum width={width("N-1")+2pt},
			draw, rectangle}]
		\node [cloud] at (5,0) (nonneg) {\sc Nonnegative};
		\node [cloud] (reduc) at (2,-3) [text width=3cm] {{\sc Reducible}\\ \noindent\begin{itemize}
			\item $\lambda_P\geq 0$
			\item $\bw_P\geq 0$
			\item $\bv_P\geq 0$
			\item $\lambda_P \geq | \lambda_j |$
			\end{itemize}
		};
		\node [cloud] (irred) at (6,-4) {\sc Irreducible};	
		\node [cloud] (notprim) at (3,-8) [text width=3cm] {{\sc Imprimitive}\\ \begin{itemize}
			\item $\lambda_P> 0$
			\item $\bw_P> 0$
			\item $\bv_P> 0$
			\item $\lambda_P=|\lambda_j|$, $j=2,\ldots,d$
			\item $\lambda_P > |\lambda_j|$, $j>d$
			\end{itemize}
		};	
		\node [cloud] (prim) [text width=3cm] at (8,-8) {{\sc Primitive}\\
			\begin{itemize}
			\item $\lambda_P> 0$
			\item $\bw_P> 0$
			\item $\bv_P> 0$
			\item $\lambda_P>|\lambda_j|$, $j\neq P$ 
			\end{itemize}
		};	
		%% Links
		\path [line, thick] (nonneg) to (reduc);
		\path [line, thick] (nonneg) to (irred);
		\path [line, thick] (irred) to (notprim);
		\path [line, thick] (irred) to (prim);
		\end{tikzpicture}
	\end{center}
	\end{theorem}	
\end{frame}

\end{document}
